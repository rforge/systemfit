
%       $Id$    

%%%%%%%%%%%%%%%%%%%%%%%%%%%%%%%%%%%%%%%%%%%%%%%%%%
%\section{Other issues and tools}\label{sec:Other}
%%%%%%%%%%%%%%%%%%%%%%%%%%%%%%%%%%%%%%%%%%%%%%%%%%

%%%%%%%%%%%%%%%%%%%%%%%%%%%%%%%%%%%%%%%%%%%%%%%%%%
\subsection{Residual covariance matrix}\label{sec:residcov}

Since the true residuals of the estimated equations are generally not known,
the true covariance matrix of the residuals cannot be determined.
Thus, this covariance matrix must be calculated from the
\emph{estimated} residuals. 
Generally, the estimated covariance matrix of the residuals
($\widehat{\Sigma} = \left[ \widehat{\sigma}_{ij} \right]$)
can be calculated from the residuals of a first-step OLS or 2SLS estimation.
The following formula is often applied:
\begin{equation}
   \widehat{\sigma}_{ij} = \frac{ \widehat{u}_i' \widehat{u}_j }{ T }
   \label{eq:rcov0}
\end{equation}
where $T$ is the number of observations in each equation.
However, in finite samples this estimator is biased,
because it is not corrected for degrees of freedom.
The usual single-equation procedure to correct for degrees of freedom
cannot always be applied, because the number of regressors in each equation
might differ.
Two alternative approaches to calculate the residual covariance
matrix are
\begin{equation}
   \widehat{\sigma}_{ij} = \frac{ \widehat{u}_i' \widehat{u}_j }
   { \sqrt{ \left( T - K_i \right) \cdot \left( T - K_j \right) } }
   \label{eq:rcov1}
\end{equation}
and
\begin{equation}
   \widehat{\sigma}_{ij} = \frac{ \widehat{u}_i' \widehat{u}_j }
   { T - \max \left( K_i , K_j \right) }
   \label{eq:rcov3}
\end{equation}
where $K_i$ and $K_j$ are the number of regressors in equation
$i$ and $j$, respectively.
However, these formulas yield unbiased estimators only if $K_i = K_j$
\citep[p.\ 469]{judge85}. 
% Greene (2003, p. 344) says that the second is unbiased if i = j or K_i = K_j,
% whereas the first is unbiased only if i = j. 
% However, if K_i = K_j the first and the second are equal.
% Why is the first biased if K_i = K_j ???????????


A further approach to obtain the estimated residual covariance
matrix is \citep[p.\ 309]{zellner62c}
\begin{eqnarray}
   \widehat{\sigma}_{ij} & = & 
   \frac{ \widehat{u}_i' \widehat{u}_j } 
   { T - K_i - K_j + tr \left[ X_i \left( X_i' X_i \right)^{-1}
   X_i' X_j \left( X_j' X_j \right)^{-1} X_j' \right] }
   \label{eq:rcov2} \\
   & = &
   \frac{ \widehat{u}_i' \widehat{u}_j } 
   { T - K_i - K_j + tr \left[ \left( X_i' X_i \right)^{-1}
   X_i' X_j \left( X_j' X_j \right)^{-1} X_j' X_i \right] }
\end{eqnarray} 
This yields an unbiased estimator for all elements of $\widehat{\Sigma}$,
but even if $\widehat{\Sigma}$ is an unbiased estimator of $\Sigma$, 
its inverse $\widehat{\Sigma}^{-1}$ is not an unbiased estimator 
of $\Sigma^{-1}$ \citep[p.\ 322]{theil71}.
Furthermore, the covariance matrix calculated by (\ref{eq:rcov2})
is not necessarily positive semidefinite \citep[p.\ 322]{theil71}. 
Hence, �it is doubtful whether [this formula] is really superior to 
[(\ref{eq:rcov0})]� \citep[p.\ 322]{theil71}.


The WLS, SUR, W2SLS and 3SLS parameter estimates are consistent
if the estimated residual covariance matrix is calculated
using the residuals from a first-step OLS or 2SLS estimation.
There exists also an alternative slightly different approach.%
\footnote{
For instance, this approach is applied by
the command �TSCS� of the software LIMDEP that carries out SUR estimations
in which all coefficient vectors are constrained to be equal
\citep{greene06}.
}
This alternative approach uses the residuals of a first-step OLS or 2SLS estimation
to apply a WLS or W2SLS estimation on a second step.
Then, it calculates the residual covariance matrix
from the residuals of the second-step estimation
to estimates the model by SUR or 3SLS in a third step.
If no cross-equation restrictions are imposed,
the parameter estimates of OLS and WLS as well as 2SLS and W2SLS are identical.
Hence, in this case both approaches generate the same results.

It is also possible to iterate WLS, SUR, W2SLS and 3SLS estimations.
At each iteration the residual covariance matrix is calculated
from the residuals of the previous iteration.
If equation (\ref{eq:rcov0}) is applied to calculate the estimated
residual covariance matrix,
an iterated SUR estimation converges to maximum
likelihood \citep[p.\ 345]{greene03}.

In some uncommon cases,
for instance in pooled estimations,
where the coefficients are restricted to be equal in all equations,
the means of the residuals of each equation are not equal to zero
$( \overline{ \widehat{u} }_i \neq 0 )$.
Therefore, it might be argued
that the residual covariance matrix should be calculated
by subtracting the means from the residuals
and substituting $\widehat{u}_i - \overline{ \widehat{u} }_i$
for $\widehat{u}_i$ in (\ref{eq:rcov0}--\ref{eq:rcov2}).


%%%%%%%%%%%%%%%%%%%%%%%%%%%%%%%%%%%%%%%%%%%%%%%%%%
\subsection{Degrees of freedom}
\label{sec:degreesOfFreedom}

To our knowledge the question about how to determine the degrees
of freedom for single-parameter $t$ tests is not comprehensively
discussed in the literature.
While sometimes the degrees of freedom of the entire system
(total number of observations in all equations minus
total number of estimated parameters)
are applied,
in other cases the degrees of freedom of each single equation
(number of observations in the equations minus
number of estimated parameters in the equation)
are used.
Asymptotically, this distinction does not make a difference.
However, in many empirical applications, the number of observations
of each equation is rather small, and
therefore, it matters.

If a system of equations is estimated by an unrestricted OLS and
the covariance matrix of the parameters is calculated
by~(\ref{eq:olsCovSingleSigma}),
the estimated parameters and their standard errors are identical
to an equation-wise OLS estimation.
In this case, it is reasonable to use the degrees of freedom of
each single equation,
because this yields the same $p$ values as the equation-wise
OLS estimation.

In contrast, if a system of equations is estimated with many
cross-equation restrictions and
the covariance matrix of an OLS estimation is calculated
by~(\ref{eq:olsCovSameSigma}),
the system estimation is similar to a single equation estimation.
Therefore, in this case, it seems to be reasonable to use the degrees
of freedom of the entire system.



%%%%%%%%%%%%%%%%%%%%%%%%%%%%%%%%%%%%%%%%%%%%%%%%%%
\subsection{Goodness of fit}

The goodness of fit of each single equation can be measured by the
traditional $R^2$ values:
\begin{equation}
   R_i^2 = 1 - \frac{ \widehat{u}_i' \widehat{u}_i }
   { ( y_i - \overline{y_i} )' ( y_i - \overline{y_i} ) }
\end{equation}
where $R_i^2$ is the $R^2$ value of the $i$th equation
and $\overline{y_i}$ is the mean value of $y_i$.

The goodness of fit of the whole system can be measured by the
McElroy's $R^2$ value \citep{mcelroy77}: 
% also: \citep[p.\ 345]{greene03}
\begin{equation}
   R_*^2 = 1 - \frac{ \widehat{u}' \widehat{ \Omega }^{-1} \widehat{u} }
   { y' \left( \widehat{ \Sigma }^{-1} \otimes
   \left( I - \frac{i i'}{T} \right) \right) y }
\end{equation}
where $T$ is the number of observations in each equation,
$I$ is an $T \times T$ identity matrix and 
$i$ is a column vector of $T$ ones.


%%%%%%%%%%%%%%%%%%%%%%%%%%%%%%%%%%%%%%%%%%%%%%%%%%
\subsection{Testing linear restrictions}
\label{sec:testingRestrictions}

Linear restrictions can be tested by an $F$ test, Wald test or
likelihood ratio (LR) test.

The $F$ statistic for systems of equations is
\begin{equation}
F = \frac{
   ( R \hat{\beta} - q )'
   ( R ( X' ( \hat{\Sigma} \otimes I )^{-1} X )^{-1} R' )^{-1}
   ( R \hat{\beta} - q ) /
   j
}{
   \hat{u}' ( \Sigma \otimes I )^{-1} \hat{u} /
   ( M \cdot T - K )
}
\end{equation}
where $j$ is the number of restrictions,
$M$ is the number of equations,
$T$ is the number of observations per equation,
$K$ is the total number of estimated coefficients, and
$\hat{\Sigma}$ is the estimated residual covariance matrix
used in the estimation.
Under the null hypothesis, $F$ has an $F$ distribution
with $j$ and $M \cdot T - K$ degrees of freedom
\citep[p.\ 314]{theil71}.

The Wald statistic for systems of equations is
\begin{equation}
W =
   ( R \hat{\beta} - q )'
   ( R \widehat{Cov} [ \hat{\beta} ] R' )^{-1}
   ( R \hat{\beta} - q )
\end{equation}
Asymptotically, $W$ has a $\chi^2$
distribution with $j$ degrees of freedom
under the null hypothesis
\citep[p.\ 347]{greene03}.

The LR statistic for systems of equations is
\begin{equation}
LR = T \cdot \left(
   log \left| \hat{ \hat{ \Sigma } }_r \right|
   - log \left| \hat{ \hat{ \Sigma } }_u \right|
   \right)
\end{equation}
where $T$ is the number of observations per equation, and 
$\hat{\hat{\Sigma}}_r$ and $\hat{\hat{\Sigma}}_u$ are
the residual covariance matrices calculated by formula (\ref{eq:rcov0})
of the restricted and unrestricted estimation, respectively.
Asymptotically, $LR$ has a $\chi^2$
distribution with $j$ degrees of freedom
under the null hypothesis
\citep[p.\ 349]{greene03}.



%%%%%%%%%%%%%%%%%%%%%%%%%%%%%%%%%%%%%%%%%%%%%%%%%%
\subsection{Hausman test}
\label{sec:hausman}

\citet{hausman78} developed a test for misspecification.
The null hypothesis of the test is that all exogenous variables are
uncorrelated with all disturbance terms.
Under this hypothesis both the 2SLS and the 3SLS estimator are consistent
but only the 3SLS estimator is (asymptotically) efficient.
Under the alternative hypothesis the 2SLS estimator is consistent
but the 3SLS estimator is inconsistent.
The Hausman test statistic is,
\begin{equation}
  m = \left( \hat{\beta}_{2SLS} - \hat{\beta}_{3SLS} \right)^{'}
      \left( \Cov \left[ \hat{\beta}_{2SLS} \right] -
             \Cov \left[ \hat{\beta}_{3SLS} \right] \right)
      \left( \hat{\beta}_{2SLS} - \hat{\beta}_{3SLS} \right)
\label{eq:hausman}
\end{equation}
where $\hat{\beta}_{2SLS}$ and $\Cov \left[ \hat{\beta}_{2SLS} \right]$ are the estimated
coefficient and covariance matrix from 2SLS estimation, and
$\hat{\beta}_{3SLS}$ and $\Cov \left[ \hat{\beta}_{3SLS} \right]$ are the estimated
coefficients and covariance matrix from 3SLS estimation.
Under the null hypothesis this test statistic has a
$\chi^2$ distribution with degrees of freedom equal to the number of
estimated parameters.




%%% Local Variables: 
%%% mode: latex
%%% TeX-master: "systemfit"
%%% End: 
