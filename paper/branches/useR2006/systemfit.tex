\documentclass[12pt]{scrartcl}
\usepackage[T1]{fontenc}
\usepackage[latin1]{inputenc}

\usepackage{geometry}
\geometry{a4paper, tmargin=2.5cm, bmargin=2.5cm, lmargin=2.5cm,
   rmargin=2.5cm, headheight=0.1cm, headsep=0.1cm, footskip=1cm }

\usepackage{setspace}
\onehalfspacing

\usepackage{natbib}
\usepackage{url}
\usepackage{csquotes}
\MakeOuterQuote{�}

\newcommand{\pkg}[1]{{\normalfont\fontseries{b}\selectfont #1}}
\let\proglang=\textsf

\hyphenation{sys-tem-fit}
\clubpenalty=10000
\widowpenalty=10000

\renewcommand{\title}[1]{\LARGE{\textbf{ #1 }}}
\renewcommand{\author}[1]{\Large{ #1 }}


\begin{document}

\begin{center}

\title{
systemfit: A Package to Estimate\\
Simultaneous Equation Systems in \proglang{R}
}

\vspace{6mm}

\author{
Arne Henningsen and Jeff D.\ Hamann
}

\end{center}


Many theoretical models that are econometrically estimated
consist of more than one equation.
In this case, the disturbance terms of these equations are likely
to be contemporaneously correlated,
because some unconsidered factors
that influence the disturbance term in one equation
probably influence the disturbance terms in other equations
of this model, too.
Ignoring this contemporaneous correlation
and estimating these equations separately
leads to inefficient parameter estimates.
However, estimating all equations simultaneously,
taking the covariance structure of the residuals into account,
leads to efficient estimates.
This estimation procedure is generally called
�Seemingly Unrelated Regression� (SUR) \citep{zellner62}.
Another reason to estimate an equation system simultaneously are
cross-equation parameter restrictions.%
\footnote{
Especially the economic theory suggests many cross-equation parameter
restrictions (e.g.\ the symmetry restriction in demand models).
}
These restrictions can be tested and/or imposed only in a simultaneous
estimation approach.

Furthermore, these models can contain variables
that appear on the left-hand side in one equation
and on the right-hand side of another equation.
Ignoring the endogeneity of these variables can lead to inconsistent
parameter estimates.
This simultaneity bias can be circumvented by applying
a �Two-Stage Least Squares� (2SLS) or �Three-Stage Least Squares� (3SLS)
estimation of the equation system.

The \pkg{systemfit} package provides the capability to estimate
linear equation systems in \proglang{R}
\citep{r-project}.
Although linear equation systems can be estimated
with several other statistical and econometric software packages,
\pkg{systemfit} has several advantages.
First, all estimation procedures are publicly available in the source code.
Second, the estimation algorithms can be easily modified to meet specific
requirements.
Third, the (advanced) user can control estimation details generally
not available in other software packages by overriding reasonable defaults.

The \pkg{systemfit} package has been tested on a variety of datasets and
has produced satisfactory for a few years.
On the useR! conference,
we would like to present some of the basic features of the
\pkg{systemfit} package,
some of the many details that can be controlled by the user,
and the statistical tests for parameter restrictions and
consistency of 3SLS estimation that are included in the package.
While the \pkg{systemfit} package performs the basic fitting methods,
more sophisticated methods are still missing.
We hope to implement missing functionalities
in the near future
--- maybe with the help of other useRs.



\bibliographystyle{jss}
\bibliography{systemfit} % a subset of my big bibtex file
%\bibliography{/home/suapm095/Documents/Literatur/arne}

\end{document}
