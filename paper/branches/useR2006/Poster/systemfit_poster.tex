\documentclass[portrait,a0b,final]{a0poster}
\input{systemfit_poster_commands}

\begin{document}
% set background color (gradient)
\psframe[fillstyle=gradient, gradbegin=white, gradend=whiteblue,
   gradmidpoint=1.0](-3,3)(1.1\textwidth,-1.1\textheight)

\begin{center}
% \fbox{
\begin{postercolumn}{1}
% main header
\posterbox{0.96\textwidth}{}{linewidth=2mm, framearc=0.3,
   linecolor=lightblue, fillstyle=gradient, gradangle=0,
   gradbegin=white, gradend=whiteblue, gradmidpoint=1.0,
   framesep=1em}{
% left useR logo
\begin{minipage}[c][9cm][c]{0.2\textwidth}
  \begin{center}
    \includegraphics[width=12cm]{useR-large}
  \end{center}
\end{minipage}
% title etc
\begin{minipage}[c][13cm][c]{0.59\textwidth}
  \begin{center}
    {\VeryHuge \textbf{systemfit}}\\[12mm]
    {\Huge Simultaneous Equation Systems in R}\\[8mm]
    {\Huge Arne Henningsen and Jeff D.\ Hamann}\\[8mm]
    {\LARGE University of Kiel (Germany) and Forest Informatics, Inc. (Corvallis, OR, USA)}
  \end{center}
\end{minipage}
% right useR logo
\begin{minipage}[c][9cm][c]{0.2\textwidth}
  \begin{center}
    \includegraphics[width=12cm]{useR-large}
  \end{center}
\end{minipage}
}
\end{postercolumn}

\vspace{15mm}

% \fbox{
\begin{postercolumn}{0.49}
% Features
\posterbox{0.92\textwidth}{}{linewidth=2mm, framearc=0.1,
   linecolor=lightblue, fillstyle=gradient, gradangle=0,
   gradbegin=white, gradend=whiteblue, gradmidpoint=1.0,
   framesep=1em}{
{\boxheaderfontsize \textbf{Features}}
{\standardfontsize
\begin{itemize}
\item several estimation methods, e.g.:
\begin{itemize}
\item Ordinary Least Squares (OLS)
\item Two-Stage Least Squares (2SLS)
\item Seemingly Unrelated Regression (SUR)
\item Three-Stage Least Squares (3SLS)
\end{itemize}
\item imposition of linear restrictions by two different methods
\item instrumental variables: the same for all equations or different for each equation
\item several different formulas to calculate the residual covariance matrix
\item several different formulas for the 3SLS estimation
\item $t$ tests: degrees of freedom of the whole system
or degrees of freedom of the single equation
\item OLS/2SLS estimation: different $\sigma^2$s for each single equation
or the same $\sigma^2$ for all equations
\item wrapper function for (classical) panel-like data in long format
(\texttt{systemfitClassic})
\item testing linear hypothesis using the F-, Wald-, and LR-statistic
\item Hausman test for the consistency of the 3SLS estimator
\end{itemize}
}}

\end{postercolumn}
% }
\hfill
% \fbox{
\begin{postercolumn}{0.49}
% Advantages
\posterbox{0.92\textwidth}{}{linewidth=2mm, framearc=0.1,
   linecolor=lightblue, fillstyle=gradient, gradangle=0,
   gradbegin=white, gradend=whiteblue, gradmidpoint=1.0,
   framesep=1em}{
{\boxheaderfontsize \textbf{Advantages of \texttt{systemfit}}}
{\standardfontsize
\begin{itemize}
\item all estimation procedures are publicly available in the source code
\item estimation algorithms can be easily modified to meet specific
requirements
\item the (advanced) user can control many estimation details generally
not available in other software packages
\item novices can safely ignore these estimation details
because they have reasonable defaults
\item reliability of \texttt{systemfit} has been demonstrated by comparing its estimation
results with results published in the literature
\end{itemize}
}}

\vspace{15mm}

% Future
\posterbox{0.92\textwidth}{}{linewidth=2mm, framearc=0.1,
   linecolor=lightblue, fillstyle=gradient, gradangle=0,
   gradbegin=white, gradend=whiteblue, gradmidpoint=1.0,
   framesep=1em}{
{\boxheaderfontsize \textbf{Plans for the Future}}
{\standardfontsize
\begin{itemize}
\item estimation with unbalanced data sets
\item estimation methods: LIML, FIML,  and GMM
\item fitting equation systems with serially correlated and
heteroscedastic disturbances
\item spatial econometric methods
\item improving the function \texttt{nlsystemfit}
to estimate systems of non-linear estimations
\end{itemize}
}}

\end{postercolumn}
% }

\vspace{15mm}

\begin{postercolumn}{1}
\posterbox{0.96\textwidth}{}{linewidth=2mm, framearc=0.3,
   linecolor=lightblue, fillstyle=gradient, gradangle=0,
   gradbegin=white, gradend=whiteblue, gradmidpoint=1.0,
   framesep=1em}{
\begin{spacing}{1.5}
{\Large
\textbf{Arne Henningsen} $\cdot$
Department of Agricultural Economics $\cdot$
University of Kiel $\cdot$
Olshausenstr.~40 $\cdot$
D-24098 Kiel (Germany) $\cdot$
Phone~+49-431-880-4445 $\cdot$
Fax~+49-431-880-1397 $\cdot$
\url{ahenningsen@agric-econ.uni-kiel.de} $\cdot$
\url{http://www.uni-kiel.de/agrarpol/ahenningsen/}\\[5mm]
\textbf{Jeff D. Hamann} $\cdot$
Forest Informatics, Inc. $\cdot$
PO Box 1421 $\cdot$
Corvallis, Oregon USA 97339-1421 $\cdot$
Phone~+1-541-754-1428 $\cdot$
Fax~+1-541-752-0288 $\cdot$
\url{jeff.hamann@forestinformatics.com} $\cdot$
\url{http://www.forestinformatics.com}
}
\end{spacing}
}
\end{postercolumn}
% }
\end{center}
\end{document}
