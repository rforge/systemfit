\documentclass{beamer}
%\documentclass[notes=show,handout]{beamer}
%\documentclass[handout]{beamer}

\usetheme[width=1.5cm]{PaloAlto}
% \usecolortheme{dolphin}
\logo{\pgfimage[width=1.5cm]{useR}}
\usepackage[english]{babel}
\usepackage[latin1]{inputenc}
%\usepackage[T1]{fontenc}
\usepackage{fancyvrb}
% \usepackage{beamerthemesplit}
\usepackage{csquotes}
\MakeOuterQuote{�}

\makeatletter
\setbeamertemplate{note page}[plain]
\setbeamerfont{note page}{size=\footnotesize}
\hyphenpenalty = 5000
\binoppenalty = 10000
\relpenalty = 5000
\newcommand{\myhrule}{\vspace{1ex} \hrule \vspace{1ex}}
%\setbeamersize{text margin left=0.5cm}
\makeatother




\title[systemfit]{
   \textbf{systemfit}\\ Simultaneous Equation Systems in R}
\author{Arne Henningsen and Jeff D.\ Hamann}
\date{University of Kiel (Germany) and\\Forest Informatics, Inc. (Corvallis, OR, USA)}

\begin{document}
\frame{ \titlepage }
\note{
\begin{itemize}
\item Welcome!
\item authors
\item package systemfit: Estimating simultaneous equation systems in R
\end{itemize}
}

% ============== Introduction ==============================
\section{Introduction}

% ----------------- Motivation ----------------------------
\subsection{Motivation}
\frame{
   \frametitle{Motivation}
\begin{itemize}
\item many theoretical models consist of more than one equation
   \begin{itemize}
   \item contemporaneous correlation of disturbance terms (likely)
   \item simultaneous estimation of all equations as
      �Seemingly Unrelated Regression� (SUR) leads to efficient results
   \end{itemize}
\item theoretically derived cross-equation parameter restrictions\\
   \begin{itemize}
   \item simultaneous estimation of all equations required
   \end{itemize}
\item endogeneity of some variables\\
   \begin{itemize}
   \item estimation using �Two-Stage Least Squares� (2SLS)
      or �Three-Stage Least Squares� (3SLS) required
   \end{itemize}
\end{itemize}
}
\note{
\begin{itemize}
\item ?
\end{itemize}
}

% ----------------- Outline ----------------------------
\subsection{Outline}
\frame{
   \frametitle{Outline}
\begin{itemize}
\item Introduction
\item Features of systemfit
\item Examples
\item Advantages
\item Plans for the Future
\end{itemize}
}
\note{
\begin{itemize}
\item ?
\end{itemize}
}

% ============== Features ==============================
\section{Features}

% ----------------- Estimation Methods ----------------------------
\subsection{Methods}
\frame{
   \frametitle{Estimation Methods}
\begin{itemize}
\item Ordinary Least Squares (OLS)
\item Two-Stage Least Squares (2SLS)
\item Seemingly Unrelated Regression (SUR)
\item Three-Stage Least Squares (3SLS)
\item \ldots
\end{itemize}
}
\note{
\begin{itemize}
\item 2SLS often used for single-equations
\end{itemize}
}

% ----------------- Estimation Control ----------------------------
\subsection{Estimation Control}
\frame{
   \frametitle{Estimation Control}
\begin{itemize}
\item imposition of linear restrictions
\item instrumental variables
\item formulas for the residual covariance matrix
\item formulas for 3SLS estimation
\item degrees of freedom for $t$ tests
\item homogenous residual variance in OLS/2SLS estimations
\end{itemize}
}
\note{
\begin{itemize}
\item imposition of linear restrictions by two different methods
\item instrumental variables: the same for all equations or different for each equation
\item several different formulas to calculate the residual covariance matrix
\item several different formulas for the 3SLS estimation
\item $t$ tests: degrees of freedom of the whole system
   or degrees of freedom of the single equation
\item OLS/2SLS estimation: different $\sigma^2$s for each single equation
   or the same $\sigma^2$ for all equations
\end{itemize}
}

% ----------------- Other Tools ----------------------------
\subsection{Other Tools}
\frame{
   \frametitle{Other Tools}
\begin{itemize}
\item wrapper function for (classical) panel-like data in long format:
   \texttt{systemfitClassic}
\item testing linear hypotheses using the F-, Wald-, and LR-statistic
\item Hausman test for the consistency of the 3SLS estimator
\end{itemize}
}
\note{
\begin{itemize}
\item ?
\end{itemize}
}


% ======================= Example ======================================
\section{Example}

% ----------------- Commands ----------------------------
\subsection{Commands}
\begin{SaveVerbatim}{specSystem}
eqDemand <- consump ~ price + income
eqSupply <- consump ~ price + farmPrice + trend
system <- list(demand=eqDemand, supply=eqSupply)
\end{SaveVerbatim}
% \end{Verbatim}
\begin{SaveVerbatim}{estSystem}
fitsur <- systemfit( "SUR", system )
\end{SaveVerbatim}
% \end{Verbatim}
\begin{SaveVerbatim}{printResults}
summary( fitsur )
\end{SaveVerbatim}
% \end{Verbatim}
\frame{
   \frametitle{Example: Commands}
\begin{itemize}
\item from Kmenta (1986): Elements of Econometrics, p.~685
\item specification of the equation system:\\[2mm]
   \BUseVerbatim{specSystem}
\item estimation using method �SUR�:
   \BUseVerbatim{estSystem}
\item printing summary results:\\
   \BUseVerbatim{printResults}
\end{itemize}
}
\note{
\begin{itemize}
\item ?
\end{itemize}
}

% ----------------- Output ----------------------------
\subsection{Output}
\begin{SaveVerbatim}{output}
???
\end{SaveVerbatim}
% \end{Verbatim}
\frame{
   \frametitle{Example: Output}
\BUseVerbatim{output}
}
\note{
\begin{itemize}
\item ?
\end{itemize}
}


% ============== Advantages ==============================
\section{Advantages}
\frame{
   \frametitle{Advantages}
\begin{itemize}
\item all estimation procedures are publicly available
\item estimation algorithms can be easily modified
\item many estimation details can be controlled by the user
\item novices can safely ignore these estimation details
\item reliability of \texttt{systemfit} has been demonstrated
\end{itemize}
}
\note{
\begin{itemize}
\item all estimation procedures are publicly available in the source code
\item estimation algorithms can be easily modified to meet specific
requirements
\item the (advanced) user can control many estimation details generally
not available in other software packages
\item novices can safely ignore these estimation details
because they have reasonable defaults
\item reliability of \texttt{systemfit} has been demonstrated by comparing its estimation
results with results published in the literature
\end{itemize}
}


% ============== Future ==============================
\section{Future}

% ----------------- General Plans for the Future ----------------------------
\subsection{General}
\frame{
   \frametitle{Plans for the Future}
\begin{itemize}
\item estimation with unbalanced data sets
\item estimation methods: LIML, FIML,  and GMM
\item fitting equation systems with serially correlated and
heteroscedastic disturbances
\item spatial econometric methods
\item improving the function \texttt{nlsystemfit}
to estimate systems of non-linear estimations
\end{itemize}
}
\note{
\begin{itemize}
\item ?
\end{itemize}
}

% ----------------- Arguments ----------------------------
\subsection{Arguments}
\frame{
   \frametitle{User Interface: Arguments}
Arguments of \texttt{systemfit}:\\[3mm]
\begin{minipage}{0.22\textwidth}
\begin{itemize}
\item method
\item eqns
\item eqnlabels
\item inst
\item data
\item R.restr
\end{itemize}
\end{minipage}
\begin{minipage}{0.32\textwidth}
\begin{itemize}
\item q.restr
\item TX
\item maxiter
\item tol
\item rcovformula
\item centerResiduals
\end{itemize}
\end{minipage}
\begin{minipage}{0.40\textwidth}
\begin{itemize}
\item formula3sls
\item probdfsys
\item single.eq.sigma
\item solvetol
\item saveMemory
\item (more in the future)
\end{itemize}
\end{minipage}\\[5mm]
\textbf{Too many?}
}
\note{
\begin{itemize}
\item ?
\end{itemize}
}

\frame{
   \frametitle{Arguments}
Reducing arguments:\\[3mm]
\begin{minipage}{0.25\textwidth}
\begin{itemize}
\item method
\item eqns
\item inst
\item data
\end{itemize}
\end{minipage}
\begin{minipage}{0.55\textwidth}
\begin{itemize}
\item R.restr
\item q.restr
\item TX
\item \textbf{control} (like in \texttt{optim})
\end{itemize}
\end{minipage}\\[5mm]
However: This would break existing code!
}
\note{
\begin{itemize}
\item ?
\end{itemize}
}

% ==================== The End ===========================
\section{ }
\frame{
   \frametitle{The End . . .}

\begin{center}
{\LARGE \emph{Thank you for your attention!}}
\end{center}
}
\note{
   Thank you for your attention!
}

% ------------ End -------------

\end{document}
