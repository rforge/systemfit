%%%%%%%%%%%%%%%%%%%%%%%%%%%%%%%%%%%%%%%%%%%%%%%%%
\section{Source code}\label{sec:code}
%%%%%%%%%%%%%%%%%%%%%%%%%%%%%%%%%%%%%%%%%%%%%%%%%%

The \pkg{systemfit} package includes functions
to estimate systems of equations
(\code{systemfit}, \code{systemfitClassic})
and to test hypotheses in these systems
(\code{ftest.systemfit}, \code{wald\-test.sys\-tem\-fit},
\code{lrtest.systemfit}, \code{hausman.systemfit}).
Furthermore, this package provides some helper functions
e.g. to extract the estimated coefficients (\code{coef.systemfit})
or to calculate predicted values (\code{predict.systemfit}).

The source code of the \pkg{systemfit} is publicly available
for download from �CRAN� (The Comprehensive R Archive Network,
\url{http://cran.r-project.org/src/contrib/Descriptions/systemfit.html}).
Since the whole package has more than 2,100 lines of code,
it is not presented in this article.
Furthermore, the code corresponds exactly to the procedures and formulas
described in the previous section.


%%%%%%%%%%%%%%%%%%%%%%%%%%%%%%%%%%%%%%%%%%%%%%%%
\subsection[Basic function systemfit]{The basic function \code{systemfit}}
The basic functionality of this package is provided by the
function \code{systemfit}.
This function estimates the equation system as described
in sections~\ref{sec:Estimation}.
If parameter restrictions are provided, the formulas in
section~\ref{sec:Restrictions} are applied.
Furthermore, the user can control several details of the estimation.
For instance, the
formula to calculate the residual covariance matrix
(see section~\ref{sec:residcov}),
the degrees of freedom for the t-tests
(see section~\ref{sec:degreesOfFreedom}), or
the formula for the 3SLS estimation
(see sections~\ref{sec:Estimation} and~\ref{sec:Restrictions})
can be specified by the user.
The \code{systemfit} function returns many objects
that users might be interest in.
A complete list is available in the documentation of this function
that is included in the package.


%%%%%%%%%%%%%%%%%%%%%%%%%%%%%%%%%%%%%%%%%%%%%%%%
\subsection[Wrapper function systemfitClassic]{The wrapper function
   \code{systemfitClassic}}
Furthermore, the \pkg{systemfit} package includes the function
\code{systemfitClassic}.
This is a wrapper function for \code{systemfit}
that can be applied to (classical) panel-like data in long format
if the regressors are the same for all equations.
The data are reshaped and the formulas are modified to enable
an estimation using the standard \code{systemfit} function.
The user can specify whether the coefficients should be restricted
to be equal in all equations.


%%%%%%%%%%%%%%%%%%%%%%%%%%%%%%%%%%%%%%%%%%%%%%%%
\subsection{Statistical tests}
The statistical tests described in
sections~\ref{sec:testingRestrictions} and~\ref{sec:hausman}
are implemented as specified in these sections.
The functions \code{ftest.systemfit}, \code{waldtest.systemfit} and
\code{lrtest.systemfit} test linear restrictions on the
estimated parameters.
On the other hand, the function \code{hausman.systemfit}
tests the consistency of the 3SLS estimator.
All functions return the empirical test statistic,
the degree(s) of freedom, and the p-value.


%%%%%%%%%%%%%%%%%%%%%%%%%%%%%%%%%%%%%%%%%%%%%%%%
\subsection{Efficiency of the code}
We have followed \cite{bates04} to make the code faster and more stable.
First, if a formula contains an inverse of a matrix
that is post-multiplied by a vector,
we use \code{solve(A, b)} instead of \code{solve(A) \%*\% b}.
Second, we calculate crossproducts
by \code{crossprod(X)} or \code{crossprod(X, y)}
instead of \code{t(X) \%*\% X} or \code{t(X) \%*\% y},
respectively.

The matrix $\Omega^{-1}$ that is used to compute the estimated
coefficients and their covariance matrix is of size
$( G \cdot T ) \times ( G \cdot T )$
(see sections~\ref{sec:Estimation} and~\ref{sec:Restrictions}).
In case of large data sets, this matrix $\Omega^{-1}$ gets really huge
and needs a lot of memory.
Therefore, we use the following transformation and compute $X' \Omega^{-1}$
by deviding the $X$ matrix into submatrices,
doing some calculations with these submatrices,
adding up some of these submatrices, and
finally putting the submatrices together:
\begin{equation}
X' \Omega^{-1}
%= X' \left( \Sigma^{-1} \otimes I \right)
= \sum_{i=1} \left[ \begin{array}{c}
   \sigma^{1i} {X^1} \\
   \sigma^{2i} {X^2} \\
   \vdots \\
   \sigma^{Gi} {X^G} \\
\end{array} \right]'
\end{equation}
where $\sigma^{ij}$ are the elements of the matrix $\Sigma^{-1}$,
and $X^i$ is a submatrix of $X$ that contains the rows
that belong to the $i$'s equation.

