%%%%%%%%%%%%%%%%%%%%%%%%%%%%%%%%%%%%%%%%%%%%%%%%%%
\subsection{Estimation with only exogenous regressors}
\label{sec:Estimation-ols-wls-sur}
%%%%%%%%%%%%%%%%%%%%%%%%%%%%%%%%%%%%%%%%%%%%%%%%%%

If all regressors are exogenous,
the system of equations~(\ref{eq:model})
can be consistently estimated by Ordinary Least Squares (OLS),
Weighted Least Squares (WLS), and Seemingly Unrelated Regression (SUR).
These estimators can be obtained by
\begin{equation}
   \bHat = \left( X' \OHat^{-1} X \right)^{-1} X' \OHat^{-1} y .
   \label{eq:ols-wls-sur}
\end{equation}
The covariance matrix of these estimators
can be estimated by
\begin{equation}
   \COVHat \left[ \bHat \right] = \left( X' \OHat^{-1} X \right)^{-1} .
   \label{eq:cov-ols-wls-sur}
\end{equation}

%%%%%%%%%%%%%%%%%%%%%%%%%%%%%%%%%%%%%%%%%
\subsubsection{Ordinary least squares (OLS)}

The Ordinary Least Squares (OLS) estimator is based on the assumption
that the disturbance terms are not contemporaneously correlated
$(\sigma_{ij} = 0 \; \forall \; i \neq j)$
and have the same variance in each equation
$( \sigma_i^2 = \sigma_j^2 \, \forall \, i, j$,
where $\sigma_i^2 = \E [ u_{it}^2 ])$.
In this case, $\OHat$ in equation~(\ref{eq:ols-wls-sur})
is equal to $I_{G \cdot T}$ and thus, cancels out.
The OLS estimator is efficient,
as long as the disturbances are not contemporaneously correlated.

If the whole system is treated as one single equation,
$\OHat$ in equation~(\ref{eq:cov-ols-wls-sur}) is $\sHat^2 I_{G \cdot T}$,
where $\sHat^2$ is the estimated variance of the disturbances.
This assumes that the disturbances of all equations have the
same variance $(\E [ u_{it}^2 ] = \sigma^2 \, \forall \, i)$.
If the disturbance terms of the individual equations
are allowed to have different variances,
$\OHat$ in equation~(\ref{eq:cov-ols-wls-sur}) is $\SHat \otimes I_T$,
where $\sHat_{ij} = 0 \; \forall \; i \neq j$ and
$\sHat_{ii} = \sHat_i^2$ is the estimated variance
of the disturbance term in the $i$th equation.

If no cross-equation parameter restrictions are imposed, the simultaneous 
OLS estimation of the system leads to the same parameter estimates 
as an equation-wise OLS estimation.
The covariance matrix of the parameters from an equation-wise
OLS estimation is equal to the covariance matrix obtained
by equation~(\ref{eq:cov-ols-wls-sur})
with $\OHat$ equal to $\SHat \otimes I_T$.

%%%%%%%%%%%%%%%%%%%%%%%%%%%%%%%%%%%%%%%%%%%%%%%
\subsubsection{Weighted least squares (WLS)}

The Weighted Least Squares (WLS) estimator allows for different variances
of the disturbance terms in the different equations
$( \sigma_i^2 \neq \sigma_j^2 \, \forall \, i \neq j)$.
In this case, $\OHat$ in equations~(\ref{eq:ols-wls-sur})
and~(\ref{eq:cov-ols-wls-sur})
is $\SHat \otimes I_T$,
where $\sHat_{ij} = 0 \; \forall \; i \neq j$ and
$\sHat_{ii} = \sHat_i^2$ is the estimated variance
of the disturbance terms in the $i$th equation.
Theoretically, $\sHat_{ii}$ should be the variance of
the (true) disturbances $( \sigma_{ii} )$.
However, they are not known in most empirical applications.
Therefore, true variances are generally replaced by estimated variances
$( \sHat_{ii} )$
that are calculated from the residuals of a first-step OLS estimation
(see section~\ref{sec:residcov}).

The WLS estimator is (asymptotically) efficient only
if the disturbance terms are not contemporaneously correlated.
If no cross-equation parameter restrictions are imposed,
the parameter estimates are equal to OLS estimates.

%%%%%%%%%%%%%%%%%%%%%%%%%%%%%%%%%%%%%%%%%%%%%%%
\subsubsection{Seemingly unrelated regression (SUR)}

If the disturbances are contemporaneously correlated, a Generalized
Least Squares (GLS) estimation leads to efficient parameter estimates.
In this case, the GLS estimator is generally called �Seemingly Unrelated Regression�
(SUR) estimator \citep{zellner62}.
However, the true covariance matrix of the disturbance terms
is generally unknown.
The textbook solution for this problem is
a Feasible Generalized Least Squares (FGLS) estimation.
As the FGLS estimator is based on
an estimated covariance matrix of the disturbance terms,
it is only asymptotically efficient.
In case of an SUR estimator, $\OHat$ in equations~(\ref{eq:ols-wls-sur})
and~(\ref{eq:cov-ols-wls-sur})
is $\SHat \otimes I_T$,
where $\SHat$ is the estimated covariance matrix of the disturbance terms.

It should be noted that while an unbiased OLS or WLS estimation requires only that
the regressors and the disturbance terms of each single 
equation are uncorrelated
$( \E \left[ u_i ' X_i \right] = 0 \; \forall \; i )$,
a consistent SUR estimation requires that all disturbance terms and all 
regressors are uncorrelated
$( \E \left[ u_i ' X_j \right] = 0 \; \forall \; i, j )$.


%%%%%%%%%%%%%%%%%%%%%%%%%%%%%%%%%%%%%%%%%%%%%%%%%%
\subsection{Estimation with endogenous regressors}
\label{sec:Estimation-2sls-w2sls-3sls}
%%%%%%%%%%%%%%%%%%%%%%%%%%%%%%%%%%%%%%%%%%%%%%%%%%

If the regressors of one or more equations are correlated
with the disturbances ($\E \left[ u_i' X_i \right] \neq 0$),
OLS, WLS, and SUR estimates are biased.
This can be circumvented by a
two-stage least squares (2SLS),
weighted two-stage least squares (W2SLS),
or a three-stage least squares (3SLS) estimation
with instrumental variables (IV).
The instrumental variables for each equation $H_i$ 
can be either different or identical for all equations.
They must not be correlated with
the disturbance terms of the corresponding equation 
($\E \left[ u_i' H_i \right] = 0$).

At the first stage new ('fitted') regressors are obtained by
\begin{equation}
   \XHat_i = H_i \left( H_i' H_i \right)^{-1} H_i' X .
\end{equation}
Then, these 'fitted' regressors are used
to obtain unbiased 2SLS, W2SLS, or 3SLS estimates of $\beta$ by
\begin{equation}
   \bHat
   = \left( \XHat' \OHat^{-1} \XHat \right)^{-1} \XHat' \OHat^{-1} y .
   \label{eq:2sls-w2sls-3sls}
\end{equation}
An estimator of the covariance matrix of the estimated parameters is
\begin{equation}
   \COVHat \left[ \bHat \right]
   = \left( \XHat' \OHat^{-1} \XHat \right)^{-1} .
   \label{eq:cov-2sls-w2sls-3sls}
\end{equation}

%%%%%%%%%%%%%%%%%%%%%%%%%%%%%%%%%%%%%%%%%%%%%%%
\subsubsection{Two-stage least squares (2SLS)}

The Two-Stage Least Squares (2SLS) estimator
is based on the same assumptions as the OLS estimator.
Accordingly, $\OHat$ in equation~(\ref{eq:2sls-w2sls-3sls})
is equal to $I_{G \cdot T}$ and thus, cancels out.
Like for the OLS estimator,
the whole system can be treated either as one single equation
with $\OHat$ in equation~(\ref{eq:cov-2sls-w2sls-3sls})
equal to $\sHat^2 I_{G \cdot T}$,
or the disturbance terms of the individual equations
are allowed to have different variances
with $\OHat$ in equation~(\ref{eq:cov-2sls-w2sls-3sls})
equal to $\SHat \otimes I_T$,
where $\sHat_{ij} = 0 \; \forall \; i \neq j$ and
$\sHat_{ii} = \sHat_i^2$.

%%%%%%%%%%%%%%%%%%%%%%%%%%%%%%%%%%%%%%%%%%%%%%%
\subsubsection{Weighted two-stage least squares (W2SLS)}

The Weighted Two-Stage Least Squares (W2SLS) estimator allows
for different variances of the disturbance terms in the different equations.
Hence, $\OHat$ in equations~(\ref{eq:2sls-w2sls-3sls})
and~(\ref{eq:cov-2sls-w2sls-3sls})
is $\SHat \otimes I_T$,
where $\sHat_{ij} = 0 \; \forall \; i \neq j$ and
$\sHat_{ii} = \sHat_i^2$.
If no cross-equation parameter restrictions are imposed,
the parameter estimates are equal to the 2SLS estimates.

%%%%%%%%%%%%%%%%%%%%%%%%%%%%%%%%%%%%%%%%%%%%%%%
\subsubsection{Three-stage least squares (3SLS)}

If the disturbances are contemporaneously correlated,
a Feasible Generalized Least Squares (FGLS) version of the two-stage least squares
estimation leads to consistent and asymptotically more efficient estimates.
This estimation procedure is generally called �Three-stage Least
Squares� (3SLS) \citep{zellner62b}.
The standard 3SLS estimator and its covariance matrix are obtained by
equations~(\ref{eq:2sls-w2sls-3sls}) and~(\ref{eq:cov-2sls-w2sls-3sls})
with $\OHat$ equal to $\SHat \otimes I_T$,
where $\SHat$ is the estimated covariance matrix of the disturbance terms.

While an unbiased 2SLS or W2SLS estimation requires only that
the instrumental variables and the disturbance terms of each single 
equation are uncorrelated
$( \E \left[ u_i' H_i \right]) = 0 \; \forall \; i )$,
\cite{schmidt90} points out that this estimator is only consistent 
if all disturbance terms and all instrumental variables are uncorrelated
$( \E \left[ u_i' H_j \right]) = 0 \; \forall \; i, j )$.

Since there might be occasions where this cannot be avoided,
\cite{schmidt90} analyses other approaches to obtain 3SLS estimators.
One of these approaches based on instrumental variable estimation
(3SLS-IV) is
\begin{equation}
   \bHat_{3SLS-IV} = \left( \XHat' \OHat^{-1} X
   \right)^{-1} \XHat' \OHat^{-1} y .
   \label{eq:3slsIv}
\end{equation}
An estimator of the covariance matrix of the estimated 3SLS-IV parameters is
\begin{equation}
   \COVHat \left[ \bHat_{3SLS-IV} \right] = \left( \XHat' \OHat^{-1}
   X \right)^{-1} .
\end{equation}
Another approach based on the Generalized Method of Moments (GMM)
estimator (3SLS-GMM) is
\begin{equation}
   \bHat_{3SLS-GMM} = \left( X' H \left( H' \OHat H \right)^{-1}
   H' X \right)^{-1} X' H \left( H' \OHat H \right)^{-1} H' y
   \label{eq:3slsGmm}
\end{equation}
with
\begin{equation}
   H =
   \left[ \begin{array}{cccc}
      H_1 & 0 & \cdots & 0\\
      0 & H_2 & \cdots & 0\\
      \vdots & \vdots & \ddots & \vdots\\
      0 & 0 & \cdots & H_G
   \end{array}\right] .
\end{equation}
An estimator of the covariance matrix of the estimated 3SLS-GMM parameters is
\begin{equation}
   \COVHat \left[ \bHat_{3SLS-GMM} \right] =
   \left( X' H \left( H' \OHat H \right)^{-1} H' X \right)^{-1} .
\end{equation}
A fourth approach developed by \cite{schmidt90} himself is
\begin{equation}
   \bHat_{3SLS-Schmidt} = \left( \XHat' \OHat^{-1} \XHat
   \right)^{-1} \XHat' \OHat^{-1}
   H \left( H' H \right)^{-1} H' y .
   \label{eq:3slsSchmidt}
\end{equation}
An estimator of the covariance matrix of these estimated parameters is
\begin{align}
   \COVHat \left[ \bHat_{3SLS-Schmidt} \right] = &
   \left( \XHat' \OHat^{-1}  \XHat \right)^{-1}
   \XHat' \OHat^{-1} H \left( H' H \right)^{-1} H' \OHat H \\
   & \left( H' H \right)^{-1} H' \OHat^{-1} \XHat
   \left( \XHat' \OHat^{-1}  \XHat \right)^{-1} .
   \nonumber
\end{align}
The econometrics software \proglang{EViews} uses
\begin{equation}
   \bHat_{3SLS-EViews} = \bHat_{2SLS} +
   \left( \XHat' \OHat^{-1} \XHat \right)^{-1}
   \XHat' \OHat^{-1} \left( y - X \bHat_{2SLS} \right) ,
   \label{eq:3slsEViews}
\end{equation}
where $\bHat_{2SLS}$ is the two-stage least squares estimator
as defined above.
\proglang{EViews} uses the standard 3SLS formula~(\ref{eq:cov-2sls-w2sls-3sls}) to
calculate an estimator of the covariance matrix of the estimated parameters.


If the same instrumental variables are used in all equations 
($H_1 = H_2 = \ldots = H_G$), 
all the above mentioned approaches lead to identical parameter estimates.
However, if this is not the case, the results depend on the 
method used \citep{schmidt90}.
The only reason to use different instruments for different equations
is a correlation of the instruments of one equation with the
disturbance terms of another equation.
Otherwise, one could simply use all instruments in every equation
\citep{schmidt90}.
In this case, only the 3SLS-GMM (\ref{eq:3slsGmm})
and the 3SLS estimator developed by \cite{schmidt90} 
(\ref{eq:3slsSchmidt}) are consistent.



%%% Local Variables: 
%%% mode: latex
%%% TeX-master: "systemfit"
%%% End: 
