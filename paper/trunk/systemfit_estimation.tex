%%%%%%%%%%%%%%%%%%%%%%%%%%%%%%%%%%%%%%%%%%%%%%%%%%
\subsection{Estimation}\label{sec:Estimation}
%%%%%%%%%%%%%%%%%%%%%%%%%%%%%%%%%%%%%%%%%%%%%%%%%%


%%%%%%%%%%%%%%%%%%%%%%%%%%%%%%%%%%%%%%%%%
\subsubsection{Ordinary least squares (OLS)}

The Ordinary Least Squares (OLS) estimator of the system 
is obtained by
\begin{equation}
   \bHat_{OLS} = \left( X'X \right)^{-1} X'y .
   \label{eq:ols}
\end{equation}
These estimates are efficient only if the disturbance terms are not
contemporaneously correlated, which means
$\sigma_{ij} = 0 \; \forall \; i \neq j$.
If the whole system is treated as one single equation,
an estimator of the covariance matrix of the estimated parameters is
\begin{equation}
   \COVHat \left[ \bHat_{OLS} \right] = \sHat^2 \left( X'X \right)^{-1} ,
   \label{eq:olsCovSameSigma}
\end{equation}
where $\sHat^2$ is the estimated variance of the disturbances.
This assumes that the disturbances of all equations have the
same variance $(\E [ u_{it}^2 ] = \sigma^2 \, \forall \, i)$.

If the disturbance terms of the individual equations
are allowed to have different variances,
an estimator of the covariance matrix of the estimated parameters is
\begin{equation}
   \COVHat \left[ \bHat_{OLS} \right] = \left( X' \OHat^{-1} X \right)^{-1} ,
   \label{eq:olsCovSingleSigma}
\end{equation}
where $\OHat = \SHat \otimes I$,
$\sHat_{ij} = 0 \; \forall \; i \neq j$ and
$\sHat_{ii} = \sHat_i^2$ is the estimated variance
of the disturbance term in the $i$th equation.

If no cross-equation parameter restrictions are imposed, the simultaneous 
OLS estimation of the system leads to the same parameter estimates 
as an equation-wise OLS estimation.
The covariance matrix of the parameters from an equation-wise
OLS estimation is equal to the covariance matrix obtained
by equation (\ref{eq:olsCovSingleSigma}).


%%%%%%%%%%%%%%%%%%%%%%%%%%%%%%%%%%%%%%%%%%%%%%%
\subsubsection{Weighted least squares (WLS)}

The Weighted Least Squares (WLS) estimator allows for different variances
of the disturbance terms in the different equations
$( \sigma_i^2 \neq \sigma_j^2 \, \forall \, i \neq j$,
where $\sigma_i^2 = \E [ u_{it}^2 ])$.
The WLS estimator can be obtained by
\begin{equation}
   \bHat_{WLS} = \left( X' \OHat^{-1} X \right)^{-1} X' \OHat^{-1} y ,
\end{equation}
where $\OHat = \SHat \otimes I$,
$\sHat_{ij} = 0 \; \forall \; i \neq j$ and
$\sHat_{ii} = \sHat_i^2$ is the variance
of the (estimated) residuals in the $i$th equation.
Theoretically, it would be better to use the variances of
the (true) disturbances $( \sigma_{ii} )$.
However, they are not known in most empirical applications.
Therefore, true variances are generally replaced by estimated variances
$( \sHat_{ii} )$
that are calculated from the residuals of a first-step OLS estimation
(see section~\ref{sec:residcov}).

Like the OLS estimates, these estimates are only efficient
if the disturbance terms are not contemporaneously correlated.
An estimator of the covariance matrix of the estimated parameters is
\begin{equation}
   \COVHat \left[ \bHat_{WLS} \right] = \left( X' \OHat^{-1} X \right)^{-1} .
\end{equation}
If no cross-equation parameter restrictions are imposed,
the parameter estimates are equal to the OLS estimates.

%%%%%%%%%%%%%%%%%%%%%%%%%%%%%%%%%%%%%%%%%%%%%%%
\subsubsection{Seemingly unrelated regression (SUR)}

When the disturbances are contemporaneously correlated, a Generalized 
Least Squares (GLS) estimation leads to efficient parameter estimates.
In this case, the GLS estimator is generally called �Seemingly Unrelated Regression�
(SUR) estimator \citep{zellner62}.
Since the true covariance matrix of the disturbance terms
is generally unknown,
a Feasible Generalized Least Squares (FGLS) estimation,
which is based on an estimated covariance matrix of the disturbance terms,
is the textbook solution for this problem.
It should be noted that while an unbiased OLS or WLS estimation requires only that
the regressors and the disturbance terms of each single 
equation are uncorrelated
$( \E \left[ u_i ' X_i \right] = 0 \; \forall \; i )$,
a consistent SUR estimation requires that all disturbance terms and all 
regressors are uncorrelated
$( \E \left[ u_i ' X_j \right] = 0 \; \forall \; i, j )$.

The SUR estimator can be obtained by
\begin{equation}
   \bHat_{SUR} = \left( X' \OHat^{-1} X \right)^{-1} X' \OHat^{-1} y ,
   \label{eq:sur}
\end{equation}
where $\OHat = \SHat \otimes I$ and
$\SHat$ is the estimated covariance matrix of the disturbance terms.
An estimator of the covariance matrix of the estimated parameters is
\begin{equation}
   \COVHat \left[ \bHat_{SUR} \right] = \left( X' \OHat^{-1} X \right)^{-1} .
   \label{eq:surCov}
\end{equation}


%%%%%%%%%%%%%%%%%%%%%%%%%%%%%%%%%%%%%%%%%%%%%%%
\subsubsection{Two-stage least squares (2SLS)}

If the regressors of one or more equations are correlated 
with the disturbances ($\E \left[ u_i' X_i \right] \neq 0$),
OLS estimates are biased.
This can be circumvented by an instrumental variable (IV)
two-stage least squares (2SLS) estimation.
The instrumental variables for each equation $H_i$ 
can be either different or identical for all equations.
The instrumental variables of each equation may not be correlated with 
the disturbance terms of the corresponding equation 
($\E \left[ u_i' H_i \right] = 0$).

At the first stage new ('fitted') regressors are obtained by
\begin{equation}
   \XHat_i = H_i \left( H_i' H_i \right)^{-1} H_i' X .
\end{equation}
At the second stage, the unbiased two-stage least squares estimates
of $\beta$ are obtained by
\begin{equation}
   \bHat_{2SLS} = \left( \XHat' \XHat \right)^{-1}
   \XHat' y .
   \label{eq:beta2sls}
\end{equation}
If the whole system is treated as one single equation,
an estimator of the covariance matrix of the estimated parameters is
\begin{equation}
   \COVHat \left[ \bHat_{2SLS} \right] = \sHat^2 \left( \XHat'
   \XHat \right)^{-1} ,
   \label{eq:2slsCovSameSigma}
\end{equation}
where $\sHat^2$ is the estimated variance of the disturbance terms.
If the disturbance terms of the individual equations
are allowed to have different variances, 
an estimator of the covariance matrix of the estimated parameters is
\begin{equation}
   \COVHat \left[ \bHat_{2SLS} \right] = \left( \XHat' \OHat^{-1}
   \XHat \right)^{-1} ,
   \label{eq:2slsCovSingleSigma}
\end{equation}
where $\OHat = \SHat \otimes I$,
$\sHat_{ij} = 0 \; \forall \; i \neq j$ and
$\sHat_{ii} = \sHat_i^2$ is the estimated variance
of the disturbance term in the $i$th equation.


%%%%%%%%%%%%%%%%%%%%%%%%%%%%%%%%%%%%%%%%%%%%%%%
\subsubsection{Weighted two-stage least squares (W2SLS)}

The Weighted Two-Stage Least Squares (W2SLS) estimator of the system 
is obtained by
\begin{equation}
   \bHat_{W2SLS} = \left( \XHat' \OHat^{-1} \XHat
   \right)^{-1} \XHat' \OHat^{-1} y ,
\end{equation}
where $\OHat = \SHat \otimes I$,
$\sHat_{ij} = 0 \; \forall \; i \neq j$ and
$\sHat_{ii} = \sHat_i^2$ is the estimated variance
of the disturbance term in the $i$th equation.
An estimator of the covariance matrix of the estimated parameters is
\begin{equation}
   \COVHat \left[ \bHat_{W2SLS} \right] = \left( \XHat' \OHat^{-1}
   \XHat \right)^{-1} .
\end{equation}


%%%%%%%%%%%%%%%%%%%%%%%%%%%%%%%%%%%%%%%%%%%%%%%
\subsubsection{Three-stage least squares (3SLS)}

If the regressors are correlated with the disturbances 
($\E \left[ u_i' X_i \right] \neq 0$) and
the disturbances are contemporaneously correlated, 
a Feasible Generalized Least Squares (FGLS) version of the two-stage least squares
estimation leads to consistent and asymptotically efficient estimates.
This estimation procedure is generally called �Three-stage Least
Squares� (3SLS) \citep{zellner62b}.

The standard 3SLS estimator can be obtained by
\begin{equation}
   \bHat_{3SLS} = \left( \XHat' \OHat^{-1} \XHat
   \right)^{-1} \XHat' \OHat^{-1} y ,
   \label{eq:3slsGls}
\end{equation}
where $\OHat = \SHat \otimes I$ and
$\SHat$ is the estimated covariance matrix of the disturbance terms.
An estimator of the covariance matrix of the estimated parameters is
\begin{equation}
   \COVHat \left[ \bHat_{3SLS} \right] = \left( \XHat' \OHat^{-1}
   \XHat \right)^{-1} .
   \label{eq:cov3sls}
\end{equation}
While an unbiased 2SLS or W2SLS estimation requires only that
the instrumental variables and the disturbance terms of each single 
equation are uncorrelated
$( \E \left[ u_i' H_i \right]) = 0 \; \forall \; i )$,
\cite{schmidt90} points out that this estimator is only consistent 
if all disturbance terms and all instrumental variables are uncorrelated
$( \E \left[ u_i' H_j \right]) = 0 \; \forall \; i, j )$
with
\begin{equation}
   H =
   \left[ \begin{array}{cccc}
      H_1 & 0 & \cdots & 0\\
      0 & H_2 & \cdots & 0\\
      \vdots & \vdots & \ddots & \vdots\\
      0 & 0 & \cdots & H_G
   \end{array}\right] .
\end{equation}
Since there might be occasions where this cannot be avoided,
\cite{schmidt90} analyses other approaches to obtain 3SLS estimators.
One of these approaches based on instrumental variable estimation
(3SLS-IV) is
\begin{equation}
   \bHat_{3SLS-IV} = \left( \XHat' \OHat^{-1} X
   \right)^{-1} \XHat' \OHat^{-1} y .
   \label{eq:3slsIv}
\end{equation}
An estimator of the covariance matrix of the estimated 3SLS-IV parameters is
\begin{equation}
   \COVHat \left[ \bHat_{3SLS-IV} \right] = \left( \XHat' \OHat^{-1}
   X \right)^{-1} .
\end{equation}
Another approach based on the Generalized Method of Moments (GMM)
estimator (3SLS-GMM) is
\begin{equation}
   \bHat_{3SLS-GMM} = \left( X' H \left( H' \OHat H \right)^{-1}
   H' X \right)^{-1} X' H \left( H' \OHat H \right)^{-1} H' y .
   \label{eq:3slsGmm}
\end{equation}
An estimator of the covariance matrix of the estimated 3SLS-GMM parameters is
\begin{equation}
   \COVHat \left[ \bHat_{3SLS-GMM} \right] =
   \left( X' H \left( H' \OHat H \right)^{-1} H' X \right)^{-1} .
\end{equation}
A fourth approach developed by \cite{schmidt90} himself is
\begin{equation}
   \bHat_{3SLS-Schmidt} = \left( \XHat' \OHat^{-1} \XHat
   \right)^{-1} \XHat' \OHat^{-1}
   H \left( H' H \right)^{-1} H' y .
   \label{eq:3slsSchmidt}
\end{equation}
An estimator of the covariance matrix of these estimated parameters is
\begin{equation}
   \COVHat \left[ \bHat_{3SLS-Schmidt} \right] =
   \left( \XHat' \OHat^{-1}  \XHat \right)^{-1}
   \XHat' \OHat^{-1} H \left( H' H \right)^{-1} H' \OHat
   H \left( H' H \right)^{-1} H' \OHat^{-1} \XHat
   \left( \XHat' \OHat^{-1}  \XHat \right)^{-1} .
\end{equation}
The econometrics software \proglang{EViews} uses
\begin{equation}
   \bHat_{3SLS-EViews} = \bHat_{2SLS} +
   \left( \XHat' \OHat^{-1} \XHat \right)^{-1}
   \XHat' \OHat^{-1} \left( y - X \bHat_{2SLS} \right) ,
   \label{eq:3slsEViews}
\end{equation}
where $\bHat_{2SLS}$ is the two-stage least squares estimator
as defined by (\ref{eq:beta2sls}).
\proglang{EViews} uses the standard 3SLS formula (\ref{eq:cov3sls}) to
calculate an estimator of the covariance matrix of the estimated parameters.


If the same instrumental variables are used in all equations 
($H_1 = H_2 = \ldots = H_G$), 
all the above mentioned approaches lead to identical parameter estimates.
However, if this is not the case, the results depend on the 
method used \citep{schmidt90}.
The only reason to use different instruments for different equations
is a correlation of the instruments of one equation with the
disturbance terms of another equation.
Otherwise, one could simply use all instruments in every equation
\citep{schmidt90}.
In this case, only the 3SLS-GMM (\ref{eq:3slsGmm})
and the 3SLS estimator developed by \cite{schmidt90} 
(\ref{eq:3slsSchmidt}) are consistent.



%%% Local Variables: 
%%% mode: latex
%%% TeX-master: "systemfit"
%%% End: 
