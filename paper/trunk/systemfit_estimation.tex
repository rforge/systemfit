%%%%%%%%%%%%%%%%%%%%%%%%%%%%%%%%%%%%%%%%%%%%%%%%%%
\section{Estimation}\label{sec:Estimation}
%%%%%%%%%%%%%%%%%%%%%%%%%%%%%%%%%%%%%%%%%%%%%%%%%%

Consider a system of $G$ equations, where the $i$th equation is of
the form 
\begin{equation}
   y_{i} = \mathbf{X}_i \beta_i + u_i, \quad i = 1, 2, \ldots, G
\end{equation}
where $y_i$ is a vector of the dependent variable,
$\mathbf{X}_i$ is a matrix of the exogenous variables,
$\beta_i$ is the coefficient vector and
$u_i$ is a vector of the disturbance terms of the $i$th equation.

We can write the �stacked� system as
\begin{equation}
   \left[ \begin{array}{c}
      y_1 \\ y_2\\ \vdots\\ y_G
   \end{array} \right] = 
   \left[ \begin{array}{cccc}
      X_1 & 0 & \cdots & 0\\
      0 & X_2 & \cdots & 0\\
      \vdots & \vdots & \ddots & \vdots\\
      0 & 0 & \cdots & X_G
   \end{array}\right] \cdot 
   \left[ \begin{array}{c}
      \beta_1 \\ \beta_2 \\ \vdots\\ \beta_G
   \end{array} \right] +
   \left[ \begin{array}{c}
      u_1 \\ u_2 \\ \vdots\\ u_G
   \end{array} \right]
\end{equation}

or more simply as
\begin{equation}
   y = X \beta + u 
\end{equation}   

We assume that there is no correlation of the disturbance terms 
across observations:
\begin{equation}
   E \left( u_{it} \, u_{jt^*} \right) = 0 
   \; \forall \; t \neq t^*
\end{equation}
where $i$ and $j$ indicate the equation number 
and $t$ and $t^*$ denote the observation number.

However, we explicitly allow for contemporaneous correlation:
\begin{equation}
   E \left( u_{it} \, u_{jt} \right) = \sigma_{ij}
\end{equation}

Thus, the covariance matrix of the total system is
\begin{equation}
   E \left( u \, u' \right) = \Omega = \Sigma \otimes I
\end{equation}
where $\Sigma = \left[ \sigma_{ij} \right]$ and $I$ is an
identity matrix.


%%%%%%%%%%%%%%%%%%%%%%%%%%%%%%%%%%%%%%%%%
\subsection{Ordinary least squares (OLS)}

The Ordinary Least Squares (OLS) estimator of the system 
is obtained by

\begin{equation}
   \widehat{\beta} = \left( X'X \right)^{-1} X'y
\end{equation}

These estimates are efficient only if the disturbance terms are not 
contemporaneously correlated, which means

\begin{equation}
   \sigma_{ij} = 0 \; \forall \; i \neq j
\end{equation}

If the whole system is treated as one single equation, 
the variance-covariance matrix of the estimated parameters is

\begin{equation}
   Var \left[ \widehat{\beta} \right] = \sigma^2 \left( X'X \right)^{-1}
\end{equation}

with $\sigma^2 = E \left( u' u \right)$.
This assumes that the disturbances of all equations have the
same variance.

If the disturbance terms of the individual equations
are allowed to have different variances,
the variance-covariance matrix of the estimated parameters is

\begin{equation}
   Var \left[ \widehat{\beta} \right] = \left( X' \Omega^{-1} X \right)^{-1}
   \label{eq:OLSvcm2}
\end{equation}
with $\Omega = \Sigma \otimes I$,
$\sigma_{ij} = 0 \; \forall \; i \neq j$ and
$\sigma_{ii} = E \left( u_i' u_i \right)$.

If no cross-equation parameter restrictions are imposed, the simultaneous 
OLS estimation of the system leads to the same parameter estimates 
as an equation-wise OLS estimation.
The variance-covariance matrix of the parameters from an equation-wise
OLS estimation is equal to the variance-covariance matrix obtained
by equation (\ref{eq:OLSvcm2}).


%%%%%%%%%%%%%%%%%%%%%%%%%%%%%%%%%%%%%%%%%%%%%%%
\subsection{Weighted least squares (WLS)}

The Weighted Least Squares (WLS) estimator of the system 
is obtained by
\begin{equation}
   \widehat{\beta} = \left( X' \Omega^{-1} X \right)^{-1} X' \Omega^{-1} y
\end{equation}
with $\Omega = \Sigma \otimes I$, 
$\sigma_{ij} = 0 \; \forall \; i \neq j$ and
$\sigma_{ii} = E \left( u_i' u_i \right)$.

Like the OLS estimates these estimates are only efficient 
if the disturbance terms are not contemporaneously correlated.

The variance-covariance matrix of the estimated parameters is
\begin{equation}
   Var \left[ \widehat{\beta} \right] = \left( X' \Omega^{-1} X \right)^{-1}
\end{equation}

If no cross-equation parameter restrictions are imposed, 
the parameter estimates are equal to the OLS estimates.

%%%%%%%%%%%%%%%%%%%%%%%%%%%%%%%%%%%%%%%%%%%%%%%
\subsection{Seemingly unrelated regression (SUR)}

When the disturbances are contemporaneously correlated, a Generalized 
Least Squares (GLS) estimation leads to efficient parameter estimates.
In this case the GLS is generally called �Seemingly Unrelated Regression�
(SUR) (see \citealp{zellner62}). 
Note, while an unbiased OLS or WLS estimation requires only that 
the regressors and the disturbance terms of each single 
equation are uncorrelated,
a consistent SUR estimation requires that all disturbance terms and all 
regressors are uncorrelated:
\begin{equation}
   E \left( u | X \right) = 0
\end{equation}

The SUR estimator can be obtained by:%
\footnote{To calculate $\Omega^{-1}$ it is not necessary to invert 
the (huge) $\Omega$ matrix, 
because it is sufficient to invert the (small) $\Sigma$ matrix and
calculate it by $\Omega^{-1} = \Sigma^{-1} \otimes I$.}

\begin{equation}
   \widehat{\beta} = \left( X' \Omega^{-1} X \right)^{-1} X' \Omega^{-1} y
\end{equation}
with $\Omega = \Sigma \otimes I$ and
$\sigma_{ij} = E \left( u_i' u_j \right)$.

And the variance-covariance matrix of the estimated parameters is
\begin{equation}
   Var \left[ \widehat{\beta} \right] = \left( X' \Omega^{-1} X \right)^{-1}
\end{equation}


%%%%%%%%%%%%%%%%%%%%%%%%%%%%%%%%%%%%%%%%%%%%%%%
\subsection{Two-Stage least squares (2SLS)}

If the regressors of one or more equations are correlated 
with the disturbances ($E \left( u_i | X_i \right) \neq 0$), 
the estimated coefficients are biased.
This can be circumvented by an instrumental variable (IV) estimation.
The instrumental variables for each equation $H_i$ 
can be either different or identical for all equations.
The instrumental variables of each equation may not be correlated with 
the disturbance terms of the corresponding equation 
($E \left( u_i | H_i \right) = 0$).

At the first stage new ('fitted') regressors are obtained by
\begin{equation}
   \widehat{X_i} = H_i \left( H_i' H_i \right)^{-1} H_i' X
\end{equation}

At the second stage the unbiased two-stage least squares estimates
of $\beta$ are obtained by:
\begin{equation}
   \widehat{\beta} = \left( \widehat{X}' \widehat{X} \right)^{-1} 
   \widehat{X}' y 
   \label{eq:beta2sls}
\end{equation}

If the whole system is treated as one single equation, 
the variance-covariance matrix of the estimated parameters is
\begin{equation}
   Var \left[ \widehat{\beta} \right] = \sigma^2 \left( \widehat{X}'
   \widehat{X} \right)^{-1}
\end{equation}
with $\sigma^2 = E \left( u' u \right)$.

If the disturbance terms of the individual equations 
are allowed to have different variances, 
the variance-covariance matrix of the estimated parameters is
\begin{equation}
   Var \left[ \widehat{\beta} \right] = \left( \widehat{X}' \Omega^{-1} 
   \widehat{X} \right)^{-1}
\end{equation}
with $\Omega = \Sigma \otimes I$, 
$\sigma_{ij} = 0 \; \forall \; i \neq j$ and
$\sigma_{ii} = E \left( u_i' u_i \right)$.


%%%%%%%%%%%%%%%%%%%%%%%%%%%%%%%%%%%%%%%%%%%%%%%
\subsection{Weighted two-stage least squares (W2SLS)}

The Weighted Two-Stage Least Squares (W2SLS) estimator of the system 
is obtained by
\begin{equation}
   \widehat{\beta} = \left( \widehat{X}' \Omega^{-1} \widehat{X} 
   \right)^{-1} \widehat{X}' \Omega^{-1} y
\end{equation}
with $\Omega = \Sigma \otimes I$, 
$\sigma_{ij} = 0 \; \forall \; i \neq j$ and
$\sigma_{ii} = E \left( u_i' u_i \right)$.

The variance-covariance matrix of the estimated parameters is
\begin{equation}
   Var \left[ \widehat{\beta} \right] = \left( \widehat{X}' \Omega^{-1} 
   \widehat{X} \right)^{-1}
\end{equation}


%%%%%%%%%%%%%%%%%%%%%%%%%%%%%%%%%%%%%%%%%%%%%%%
\subsection{Three-Stage least squares (3SLS)}

If the regressors are correlated with the disturbances 
($E \left( u | X \right) \neq 0$) and 
the disturbances are contemporaneously correlated, 
an Instrumental Variable (IV) Generalized Least Squares (GLS) 
estimation leads to consistent and efficient estimates.
This estimation procedure is generally called �Three-stage Least
Squares� (3SLS).

The standard 3SLS estimator can be obtained by:
\begin{equation}
   \widehat{\beta} = \left( \widehat{X}' \Omega^{-1} \widehat{X} 
   \right)^{-1} \widehat{X}' \Omega^{-1} y
\end{equation}
with $\Omega = \Sigma \otimes I$ and
$\sigma_{ij} = E \left( u_i' u_j \right)$.

Its variance-covariance matrix is:
\begin{equation}
   Var \left[ \widehat{\beta} \right] = \left( \widehat{X}' \Omega^{-1} 
   \widehat{X} \right)^{-1}
   \label{eq:cov3sls}
\end{equation}

While an unbiased 2SLS or W2SLS estimation requires only that 
the instrumental variables and the disturbance terms of each single 
equation are uncorrelated,
\cite{schmidt90} points out that this estimator is only consistent 
if all disturbance terms and all instrumental variables are uncorrelated:
\begin{equation}
   E \left( u | H \right) = 0
\end{equation}
with
\begin{equation}
   H =
   \left[ \begin{array}{cccc}
      H_1 & 0 & \cdots & 0\\
      0 & H_2 & \cdots & 0\\
      \vdots & \vdots & \ddots & \vdots\\
      0 & 0 & \cdots & H_G
   \end{array}\right] \cdot 
\end{equation}

Since there might be occasions where this cannot be avoided, 
\cite{schmidt90} analyses other approaches to obtain 3SLS estimators:

One of these approaches is based on instrumental variable estimation
(IV-3SLS):
\begin{equation}
   \widehat{\beta} = \left( \widehat{X}' \Omega^{-1} X 
   \right)^{-1} \widehat{X}' \Omega^{-1} y
\end{equation}

The variance-covariance matrix of this IV-3SLS estimator is:
\begin{equation}
   Var \left[ \widehat{\beta} \right] = \left( \widehat{X}' \Omega^{-1} 
   X \right)^{-1}
\end{equation}


Another approach is based on the Generalized Method of Moments (GMM) 
estimator (GMM-3SLS):
\begin{equation}
   \widehat{\beta} = \left( X' H \left( H' \Omega H \right)^{-1}
   H' X \right)^{-1} X' H \left( H' \Omega H \right)^{-1} H' y
   \label{eq:beta3slsGMM}
\end{equation}

The variance-covariance matrix of the GMM-3SLS estimator is:
\begin{equation}
   Var \left[ \widehat{\beta} \right] = 
   \left( X' H \left( H' \Omega H \right)^{-1} H' X \right)^{-1}
\end{equation}


A fourth approach developed by \cite{schmidt90} himself is:
\begin{equation}
   \widehat{\beta} = \left( \widehat{X}' \Omega^{-1} \widehat{X} 
   \right)^{-1} \widehat{X}' \Omega^{-1} 
   H \left( H' H \right)^{-1} H' y 
   \label{eq:beta3slsSchmidt}
\end{equation}

The variance-covariance matrix of this estimator is:
\begin{equation}
   Var \left[ \widehat{\beta} \right] = 
   \left( \widehat{X}' \Omega^{-1}  \widehat{X} \right)^{-1} 
   \widehat{X}' \Omega^{-1} H \left( H' H \right)^{-1} H' \Omega 
   H \left( H' H \right)^{-1} H' \Omega^{-1} \widehat{X}
   \left( \widehat{X}' \Omega^{-1}  \widehat{X} \right)^{-1}
\end{equation}


The econometrics software EViews uses following approach:
\begin{equation}
   \widehat{\beta} = \widehat{\beta}_{2SLS} + 
   \left( \widehat{X}' \Omega^{-1} \widehat{X} \right)^{-1} 
   \widehat{X}' \Omega^{-1} \left( y - X \widehat{\beta}_{2SLS} \right)
\end{equation}
with $\widehat{\beta}_{2SLS}$ is the two-stage least squares estimator
(equation \ref{eq:beta2sls}).
EViews uses the standard 3SLS formula (\ref{eq:cov3sls}) to
calculate the variance-covariance matrix of the 3SLS estimator.


If the same instrumental variables are used in all equations 
($H_1 = H_2 = \ldots = H_G$), 
all the above mentioned approaches lead to identical parameter estimates.
However, if this is not the case, the results depend on the 
equation used \citep{schmidt90}. 
The only reason to use different instruments in the different equations 
is that the instruments of one equation are correlated with the
disturbance terms in another equation.
Otherwise, one would simply use every instrument in every equation
\citep{schmidt90}.
In this case, only the GMM-3SLS (\ref{eq:beta3slsGMM})
and the 3SLS estimator developed by \cite{schmidt90} 
(\ref{eq:beta3slsSchmidt}) are consistent.



%%% Local Variables: 
%%% mode: latex
%%% TeX-master: "systemfit"
%%% End: 
