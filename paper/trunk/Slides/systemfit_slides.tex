\documentclass{beamer}
%\documentclass[notes=show,handout]{beamer}
%\documentclass[handout]{beamer}

\usetheme[width=1.5cm]{PaloAlto}
% \usecolortheme{dolphin}
\logo{\pgfimage[width=1.5cm]{useR}}
\usepackage[english]{babel}
\usepackage[latin1]{inputenc}
%\usepackage[T1]{fontenc}
\usepackage{fancyvrb}
% \usepackage{beamerthemesplit}
\usepackage{csquotes}
\MakeOuterQuote{�}

\makeatletter
\setbeamertemplate{note page}[plain]
\setbeamerfont{note page}{size=\footnotesize}
\hyphenpenalty = 5000
\binoppenalty = 10000
\relpenalty = 5000
\newcommand{\myhrule}{\vspace{1ex} \hrule \vspace{1ex}}
%\setbeamersize{text margin left=0.5cm}
\makeatother




\title[systemfit]{
   \textbf{systemfit}\\ Simultaneous Equation Systems in R}
\author[Arne Henningsen and Jeff D.\ Hamann]{
   Arne Henningsen (University of Kiel, Germany)\\[0.4em]
   Jeff D.\ Hamann (Forest Informatics, Inc., Corvallis, USA)}
\date{useR!, Vienna, June 16, 2006}

\begin{document}
\frame{ \titlepage }
\note{
\begin{itemize}
\item Welcome to my presentation about the R package systemfit.
\item The package can be used to estimate simultaneous equation systems in R.
\item It has been written by me, Arne Henningsen,
   and by my co-author Jeff Hamann.
\end{itemize}
}

% ============== Introduction ==============================
\section{Introduction}

% ----------------- Motivation ----------------------------
\subsection{Motivation}
\frame{
   \frametitle{Motivation}
Many theoretical models consist of more than one equation
   \begin{itemize}
   \item contemporaneous correlation of disturbance terms (likely)
   \item simultaneous estimation of all equations as
      �Seemingly Unrelated Regression� (SUR) leads to efficient results
   \end{itemize}
\medskip
Theoretically derived cross-equation parameter restrictions\\
   \begin{itemize}
   \item simultaneous estimation of all equations required
   \end{itemize}
\medskip
Endogeneity of some variables
   \begin{itemize}
   \item estimation using �Two-Stage Least Squares� (2SLS)
      or �Three-Stage Least Squares� (3SLS) required
   \end{itemize}
\bigskip
$\Rightarrow$ All these models can be estimated by \texttt{systemfit}
}
\note{
\begin{itemize}
\item Many theoretical models that are econometrically estimated
   consist of more than one equation.
\item For instance, economists like me model the supply
   and the demand of a certain good.
\item Or forest engineers like my co-author Jeff model the growth of trees
   and take the height as well as the diameter of the trees into account.
\item The disturbance terms of the equations of an equation system
   are likely contemporaneously correlated.
\item For instance, if you have a specific tree with a below average
   diameter growth,
   the growth of the height will probably be also below average.
\item A standard Ordinary Least Squares regression of each equation separately
   ignores this correlation and leads to inefficient estimates.
\item However, efficient estimates can be obtained,
   if all equations are estimated simultaneously
   using Zellner's �Seemingly Unrelated Regression� method.
\item Moreover, if the underlying theory suggests cross-equation
   parameter restrictions,
   a simultaneous estimation of all equations is required, too.
\item Finally, it might be that some variables are the endogenous variable
    in some equations and a regressor in other equations.
\item In this case, �Two-Stage Least Squares� or �Three-Stage Least Squares�
   regressions have to be applied to obtain unbiased estimates.
\item All these models can be estimated by \texttt{systemfit}.
\end{itemize}
}

% ----------------- Outline ----------------------------
\subsection{Outline}
\frame{
   \frametitle{Outline}
\begin{itemize}
\setlength{\itemsep}{0.75em}
\item Introduction
\item Features of systemfit
\item Example
% \item Advantages
\item Plans for the Future
\end{itemize}
}
\note{
\begin{itemize}
\item After this introduction, I will present the features of systemfit.
\item Then, I will demonstrate the usage of systemfit by a simple example.
% \item Thereafter, I will highlight some advantages of systemfit compared to
%    other software packages.
\item And finally, I will present our future plans to extend and improve systemfit.
\end{itemize}
}

% ============== Features ==============================
\section{Features}

% ----------------- Estimation Methods ----------------------------
\subsection{Methods}
\frame{
   \frametitle{Estimation Methods}
\begin{itemize}
\setlength{\itemsep}{0.75em}
\item Ordinary Least Squares (OLS)
\item Two-Stage Least Squares (2SLS)
\item Seemingly Unrelated Regression (SUR)
\item Three-Stage Least Squares (3SLS)
\item \ldots
\end{itemize}
}
\note{
\begin{itemize}
\item The most important features of \texttt{systemfit}
   are -- of course -- the estimation methods.
\item It provides
   \begin{itemize}
   \item Ordinary Least Squares
   \item Two-Stage Least Squares
   \item Seemingly Unrelated Regression
   \item Three-Stage Least Squares
   \item and some other less important methods
   \end{itemize}
\item Although the �Two-Stage Least Squares� estimator is designed
   for estimating equation systems,
   it can be applied to single equations, too.
\item I have heard from several people -- among them Jeff Racine --
   that they generally use systemfit to fit single-equation models
   by �Two-Stage Least Squares�.
\end{itemize}
}

% ----------------- Estimation Control ----------------------------
\subsection{Estimation Control}
\frame{
   \frametitle{Estimation Control}
\begin{itemize}
\setlength{\itemsep}{0.75em}
\item imposition of linear restrictions
\item instrumental variables
\item iteration of FGLS estimation
\item formulas for the residual covariance matrix
\item formulas for 3SLS estimation
% \item degrees of freedom for $t$ tests
% \item homogenous residual variance in OLS/2SLS estimations
\item \ldots
\end{itemize}
}
\note{
\begin{itemize}
\item Besides the estimation method,
   the user can control many further details of the estimation.
\item For instance, she or he can impose linear restrictions
   by two different methods.
\item Instrumental variables can be provided
   either by specifying the same variables for all equations or
   by specifying different variables for each equation.
\item Furthermore, it can be indicated whether feasible generalized least squares
   estimations should be iterated or not.
\item The literature suggests several different formulas to calculate
   the residual covariance matrix and the Three-Stage Least Squares estimator.
\item To my knowledge, systemfit is the only statistical software
   where the user can decide which formula should be applied.
\item There exist some further details that can be controlled by the user.
\item If you are interested in these, please don't hesitate to ask me or
   take a look into the documentation.
\end{itemize}
}

% ----------------- Other Tools ----------------------------
\subsection{Other Tools}
\frame{
   \frametitle{Other Tools}
\begin{itemize}
\setlength{\itemsep}{0.75em}
\item \texttt{systemfitClassic}: wrapper function for (classical)
   panel-like data in long format
\item testing linear hypotheses using the F-, Wald-, and LR-statistic
\item Hausman test for the consistency of the 3SLS estimator
\end{itemize}
}
\note{
\begin{itemize}
\item The package systemfit does not just provide the function systemfit
   to estimate equation systems,
   but also a few other functions.
\item For instance, the package includes the wrapper function systemfitClassic
   that can be applied to panel-like data in long format.
\item Then, there are functions to test linear hypotheses using the
   F-, Wald-, and Likelihood-ratio-statistic.
\item Finally, the consistency of the Three-Stage Least Squares estimator
   can be tested by a Hausman test.
\end{itemize}
}


% ======================= Example ======================================
\section{Example}

% ----------------- Commands ----------------------------
\subsection{Commands}
\begin{SaveVerbatim}{specSystem}
eqDemand <- consump ~ price + income
eqSupply <- consump ~ price + farmPrice + trend
eqSystem <- list(demand=eqDemand, supply=eqSupply)
\end{SaveVerbatim}
% \end{Verbatim}
\begin{SaveVerbatim}{estSystem}
fitsur <- systemfit("SUR", eqSystem, data=Kmenta)
\end{SaveVerbatim}
% \end{Verbatim}
\begin{SaveVerbatim}{printResults}
summary( fitsur )
\end{SaveVerbatim}
% \end{Verbatim}
\frame{
   \frametitle{Example: Commands}
\begin{itemize}
\setlength{\itemsep}{0.75em}
\item from Kmenta (1986): Elements of Econometrics, p.~685
\item specification of the equation system:\\[2mm]
   {\small \BUseVerbatim{specSystem}}
\item estimation using method �SUR�:
   {\small \BUseVerbatim{estSystem}}
\item printing summary results:\\
   {\small \BUseVerbatim{printResults}}
\end{itemize}
}
\note{
\begin{itemize}
\item Now, I will shortly demonstrate how to use systemfit.
\item I have selected an example from Kmenta's textbook �Elements of
   Econometrics�.
\item In this example, the demand and the supply of food in the U.S.\ is
   analyzed.
\item The demand and the supply equation are specified first.
\item The consumed food quantity is a function of the food price and
   the income of the consumers.
\item The produced quantity equals the consumed quantity and depends
   on the food price, the average price of farm products
   and a time trend to account for technical progress.
\item In the third line, these two equations are combined into a list
   named �eqSystem�.
\item Then the equation system is estimated by �Seemingly Unrelated
   Regression�
\item Of course, the food price is endogenous and therefore,
   the system should be estimated by �Three-Stage Least Squares�
   using instrumental variables.
   However, I want to keep this example as simple as possible.
\item Finally, summary results of the regression are printed.
\end{itemize}
}

% ----------------- Output ----------------------------
\subsection{Output}
\begin{SaveVerbatim}{outputSystem}
systemfit results
method: SUR

        N DF      SSR     MSE    RMSE       R2   Adj R2
demand 20 17  65.6829 3.86370 1.96563 0.755019 0.726198
supply 20 16 104.0584 6.50365 2.55023 0.611888 0.539117

[...]

The correlations of the residuals
         demand   supply
demand 1.000000 0.982348
supply 0.982348 1.000000

The determinant of the residual covariance matrix: 0.879285
OLS R-squared value of the system: 0.683453
McElroy's R-squared value for the system: 0.788722
\end{SaveVerbatim}
% \end{Verbatim}
\frame{
   \frametitle{Results of the Entire System}
{\small \BUseVerbatim{outputSystem}}
}
\note{
\begin{itemize}
\item First, the general results of the entire system are presented.
\item For instance, the number of observations, \texttt{N},
   the degrees of freedom, \texttt{DF},
   as well as R-squared and adjusted R-squared values
   are printed for each equation.
\item Besides some other results, the correlation matrix of the residuals
   is presented.
\item A coefficient of correlation of about 0.98 shows that the residuals
   of the two equations are highly correlated.
\item This indicates that OLS results that do not take this correlation
   into account are rather inefficient.
\item Finally, the determinant of the residual covariance matrix,
   the OLS R-squared value and McElroy's R-squared value of
   the equation system are printed.
\end{itemize}
}

\begin{SaveVerbatim}{outputEq}
SUR estimates for 'demand' (equation 1)
Model Formula: consump ~ price + income

             Estimate Std. Error   t value Pr(>|t|)
(Intercept) 99.332894   7.514452 13.218913        0 ***
price       -0.275486   0.088509 -3.112513 0.006332  **
income        0.29855   0.041945  7.117605    2e-06 ***
---
Signif. codes:  0 '***' 0.001 '**' 0.01 '*' 0.05 '.' 0.1 ' ' 1

Residual standard error: 1.96563 on 17 degrees of freedom
Number of observations: 20 Degrees of Freedom: 17
SSR: 65.682902 MSE: 3.8637 Root MSE: 1.96563
Multiple R-Squared: 0.755019 Adjusted R-Squared: 0.726198
\end{SaveVerbatim}
% \end{Verbatim}
\frame{
   \frametitle{Results of a Single Equation}
{\small \BUseVerbatim{outputEq}}
}
\note{
\begin{itemize}
\item Summary results of each equation are printed then.
\item The results of the demand equation are presented on this slide.
\item The output is pretty similar to the output of �summary.lm�.
\item For instance, the estimated coefficients, their standard errors,
   the t-values, and the marginal levels of significance are shown.
\end{itemize}
}


% ============== Advantages ==============================
% \section{Advantages}
% \frame{
%    \frametitle{Advantages}
% \begin{itemize}
% \item all estimation procedures are publicly available
% \item estimation algorithms can be easily modified
% \item many estimation details can be controlled by the user
% \item novices can safely ignore these estimation details
% \item reliability of \texttt{systemfit} has been demonstrated
% \end{itemize}
% }
% \note{
% \begin{itemize}
% \item all estimation procedures are publicly available in the source code
% \item estimation algorithms can be easily modified to meet specific
% requirements
% \item the (advanced) user can control many estimation details generally
% not available in other software packages
% \item novices can safely ignore these estimation details
% because they have reasonable defaults
% \item reliability of \texttt{systemfit} has been demonstrated by comparing its estimation
% results with results published in the literature
% \end{itemize}
% }


% ============== Future ==============================
\section{Future}

% ----------------- General Plans for the Future ----------------------------
\subsection{General}
\frame{
   \frametitle{Plans for the Future}
\begin{itemize}
\setlength{\itemsep}{0.75em}
\item estimation with unbalanced data sets
\item estimation methods: LIML, FIML,  and GMM
\item fitting equation systems with serially correlated and
heteroscedastic disturbances
\item spatial econometric methods
\item simplify specification of parameter restrictions
\item improving the function \texttt{nlsystemfit}
to estimate systems of non-linear equations
\item \ldots
\end{itemize}
}
\note{
\begin{itemize}
\item Although systemfit has many features and produces
   reliable results,
   there are still many things that can be added to the package.
\item For instance, we plan to allow for an estimation with
   unbalanced data sets -- these are datasets with unequal numbers
   of observations for each equation.
\item Then we would like to add the estimation methods
   �Limited Information Maximum Likelihood�,
   �Full Information Maximum Likelihood�, and
   �Generalized Method of Moments�.
\item Furthermore, we plan to add estimation procedures
   that account for serially correlated and heteroscedastic disturbances
   as well as for spatial data.
\item And we would like to simplify the specification of parameter
   restrictions.
\item Finally, we would like to improve the function \texttt{nlsystemfit}
   to estimate systems of non-linear equations.
\item And there are of other things that can be improved.
\item If you are interested,
   you are invited to help us improving systemfit.
\end{itemize}
}

% ----------------- Arguments ----------------------------
\subsection{Arguments}
\frame{
   \frametitle{User Interface: Arguments}
Arguments of \texttt{systemfit}:\\[3mm]
\begin{minipage}{0.22\textwidth}
\begin{itemize}
\item method
\item eqns
\item eqnlabels
\item inst
\item data
\item R.restr
\end{itemize}
\end{minipage}
\begin{minipage}{0.32\textwidth}
\begin{itemize}
\item q.restr
\item TX
\item maxiter
\item tol
\item rcovformula
\item centerResiduals
\end{itemize}
\end{minipage}
\begin{minipage}{0.40\textwidth}
\begin{itemize}
\item formula3sls
\item probdfsys
\item single.eq.sigma
\item solvetol
\item saveMemory
\item (more in the future)
\end{itemize}
\end{minipage}\\[5mm]
\textbf{Too many?}
}
\note{
\begin{itemize}
\item At the end of my presentation,
   I would like to ask a question to the users of systemfit.
\item As I have indicated before,
   the user can control many estimation details of systemfit.
\item However, this has also an important drawback:
\item the systemfit function has a lot of arguments.
\item And the number of arguments will increase when we add further features
   in the future.
\item Although most arguments have reasonable default values
   and can be ignored by unexperienced users,
   the long list of arguments might confuse many newbies.
\end{itemize}
}

\frame{
   \frametitle{Arguments}
Reducing arguments?\\[3mm]
\begin{minipage}{0.25\textwidth}
\begin{itemize}
\item method
\item eqns
\item inst
\item data
\end{itemize}
\end{minipage}
\begin{minipage}{0.55\textwidth}
\begin{itemize}
\item R.restr
\item q.restr
\item TX
\item \textbf{control} (like in \texttt{optim})
\end{itemize}
\end{minipage}\\[5mm]
However: This would break existing code!
}
\note{
\begin{itemize}
\item Therefore, we consider whether we should reduce the number
   of arguments.
\item All arguments that are generally not used by unadvanced users,
   could be subsumed into a list of control parameters
   like in the function �optim�.
\item However: This would break existing code!
\item I would like to know what you think about this issue.
\item Do you think we should reduce the arguments
   or keep all arguments as they are?
\item Or do you have a better solution to make the function more concise?
\end{itemize}
}

% ==================== The End ===========================
\section{ }
\frame{
   \frametitle{The End . . .}

\begin{center}
{\LARGE \emph{Thank you for your attention!}}
\end{center}
}
\note{
   Thank you very much for your attention!
}

% ------------ End -------------

\end{document}
