%%%%%%%%%%%%%%%%%%%%%%%%%%%%%%%%%%%%%%%%%%%%%%%%%%
\section{Statistical background}\label{sec:statistics}
%%%%%%%%%%%%%%%%%%%%%%%%%%%%%%%%%%%%%%%%%%%%%%%%%%

In this section we give a short overview of the statistical background
on that the \pkg{systemfit} package is based on.
More detailed descriptions of simultaneous equations systems
are available for instance in
\citet[chapter~7]{theil71}
\citet[part~4]{judge82}
\citet[part~5]{judge85}
\citet{srivastava87},
\citet[chapters 14--15]{greene03}, and
\citet[chapter~10]{zivot06}.
After introducing notations and assumptions,
we provide the formulas to estimate systems of linear equations.
We then demonstrate how to impose linear restrictions on parameters.
Finally, we present additional relevant issues about estimation of
equation systems.

Consider a system of $G$ equations, where the $i$th equation is of
the form 
\begin{equation}
   y_{i} = X_i \beta_i + u_i, \quad i = 1, 2, \ldots, G ,
\end{equation}
where $y_i$ is a vector of the dependent variable,
$X_i$ is a matrix of the exogenous variables,
$\beta_i$ is the coefficient vector and
$u_i$ is a vector of the disturbance terms of the $i$th equation.

We can write the �stacked� system as
\begin{equation}
   \left[ \begin{array}{c}
      y_1 \\ y_2\\ \vdots\\ y_G
   \end{array} \right] = 
   \left[ \begin{array}{cccc}
      X_1 & 0 & \cdots & 0\\
      0 & X_2 & \cdots & 0\\
      \vdots & \vdots & \ddots & \vdots\\
      0 & 0 & \cdots & X_G
   \end{array}\right]
   \left[ \begin{array}{c}
      \beta_1 \\ \beta_2 \\ \vdots\\ \beta_G
   \end{array} \right] +
   \left[ \begin{array}{c}
      u_1 \\ u_2 \\ \vdots\\ u_G
   \end{array} \right]
   \label{eq:model-array}
\end{equation}

or more simply as
\begin{equation}
   y = X \beta + u .
   \label{eq:model-matrices}
\end{equation}   

We assume that there is no correlation of the disturbance terms 
across observations, so that
\begin{equation}
   \E \left[ u_{it} \, u_{js} \right] = 0
   \; \forall \; t \neq s ,
\end{equation}
where $i$ and $j$ indicate the equation number 
and $t$ and $s$ denote the observation number where the number of
observations is the same for all equations. 

However, we explicitly allow for contemporaneous correlation, i.e.\
\begin{equation}
   \E \left[ u_{it} \, u_{jt} \right] = \sigma_{ij} .
\end{equation}

Thus, the covariance matrix of all disturbances is
\begin{equation}
   \E \left[ u \, u' \right] = \Omega = \Sigma \otimes I ,
\end{equation}
where $\Sigma = \left[ \sigma_{ij} \right]$ is the (contemporaneous)
disturbance covariance matrix and $I$ is an identity matrix.


%%%%%%%%%%%%%%%%%%%%%%%%%%%%%%%%%%%%%%%%%%%%%%%%%%
\subsection{Estimation}\label{sec:Estimation}
%%%%%%%%%%%%%%%%%%%%%%%%%%%%%%%%%%%%%%%%%%%%%%%%%%


%%%%%%%%%%%%%%%%%%%%%%%%%%%%%%%%%%%%%%%%%
\subsubsection{Ordinary least squares (OLS)}

The Ordinary Least Squares (OLS) estimator of the system 
is obtained by
\begin{equation}
   \widehat{\beta}_{OLS} = \left( X'X \right)^{-1} X'y
   \label{eq:ols}
\end{equation}
These estimates are efficient only if the disturbance terms are not
contemporaneously correlated, which means
$\sigma_{ij} = 0 \; \forall \; i \neq j$.
If the whole system is treated as one single equation,
the covariance matrix of the estimated parameters is
\begin{equation}
   \Cov \left[ \widehat{\beta}_{OLS} \right] = \sigma^2 \left( X'X \right)^{-1}
   \label{eq:olsCovSameSigma}
\end{equation}
with $\sigma^2 = E \left( u' u \right)$.
This assumes that the disturbances of all equations have the
same variance.

If the disturbance terms of the individual equations
are allowed to have different variances,
the covariance matrix of the estimated parameters is
\begin{equation}
   \Cov \left[ \widehat{\beta}_{OLS} \right] = \left( X' \Omega^{-1} X \right)^{-1}
   \label{eq:olsCovSingleSigma}
\end{equation}
with $\Omega = \Sigma \otimes I$,
$\sigma_{ij} = 0 \; \forall \; i \neq j$ and
$\sigma_{ii} = E \left( u_i' u_i \right)$.

If no cross-equation parameter restrictions are imposed, the simultaneous 
OLS estimation of the system leads to the same parameter estimates 
as an equation-wise OLS estimation.
The covariance matrix of the parameters from an equation-wise
OLS estimation is equal to the covariance matrix obtained
by equation (\ref{eq:olsCovSingleSigma}).


%%%%%%%%%%%%%%%%%%%%%%%%%%%%%%%%%%%%%%%%%%%%%%%
\subsubsection{Weighted least squares (WLS)}

The Weighted Least Squares (WLS) estimator of the system 
is obtained by
\begin{equation}
   \widehat{\beta}_{WLS} = \left( X' \Omega^{-1} X \right)^{-1} X' \Omega^{-1} y
\end{equation}
with $\Omega = \Sigma \otimes I$, 
$\sigma_{ij} = 0 \; \forall \; i \neq j$ and
$\sigma_{ii} = E \left( u_i' u_i \right)$.
Like the OLS estimates these estimates are only efficient
if the disturbance terms are not contemporaneously correlated.
The covariance matrix of the estimated parameters is
\begin{equation}
   \Cov \left[ \widehat{\beta}_{WLS} \right] = \left( X' \Omega^{-1} X \right)^{-1}
\end{equation}
If no cross-equation parameter restrictions are imposed,
the parameter estimates are equal to the OLS estimates.

%%%%%%%%%%%%%%%%%%%%%%%%%%%%%%%%%%%%%%%%%%%%%%%
\subsubsection{Seemingly unrelated regression (SUR)}

When the disturbances are contemporaneously correlated, a Generalized 
Least Squares (GLS) estimation leads to efficient parameter estimates.
In this case, the GLS is generally called �Seemingly Unrelated Regression�
(SUR) \citep{zellner62}.
It should be noted that while an unbiased OLS or WLS estimation requires only that 
the regressors and the disturbance terms of each single 
equation are uncorrelated
$( E \left[ u_i | X_i \right] = 0 \; \forall \; i )$,
a consistent SUR estimation requires that all disturbance terms and all 
regressors are uncorrelated
$( E \left[ u | X \right] = 0 )$.

The SUR estimator can be obtained by:
\begin{equation}
   \widehat{\beta}_{SUR} = \left( X' \Omega^{-1} X \right)^{-1} X' \Omega^{-1} y
   \label{eq:sur}
\end{equation}
with $\Omega = \Sigma \otimes I$ and
$\sigma_{ij} = E \left( u_i' u_j \right)$.
And the covariance matrix of the estimated parameters is
\begin{equation}
   \Cov \left[ \widehat{\beta}_{SUR} \right] = \left( X' \Omega^{-1} X \right)^{-1}
   \label{eq:surCov}
\end{equation}


%%%%%%%%%%%%%%%%%%%%%%%%%%%%%%%%%%%%%%%%%%%%%%%
\subsubsection{Two-stage least squares (2SLS)}

If the regressors of one or more equations are correlated 
with the disturbances ($E \left( u_i | X_i \right) \neq 0$), 
the estimated coefficients are biased.
This can be circumvented by an instrumental variable (IV)
two-stage least squares (2SLS) estimation.
The instrumental variables for each equation $H_i$ 
can be either different or identical for all equations.
The instrumental variables of each equation may not be correlated with 
the disturbance terms of the corresponding equation 
($E \left( u_i | H_i \right) = 0$).

At the first stage new ('fitted') regressors are obtained by
\begin{equation}
   \widehat{X_i} = H_i \left( H_i' H_i \right)^{-1} H_i' X
\end{equation}
At the second stage the unbiased two-stage least squares estimates
of $\beta$ are obtained by:
\begin{equation}
   \widehat{\beta}_{2SLS} = \left( \widehat{X}' \widehat{X} \right)^{-1} 
   \widehat{X}' y 
   \label{eq:beta2sls}
\end{equation}
If the whole system is treated as one single equation,
the covariance matrix of the estimated parameters is
\begin{equation}
   \Cov \left[ \widehat{\beta}_{2SLS} \right] = \sigma^2 \left( \widehat{X}'
   \widehat{X} \right)^{-1}
   \label{eq:2slsCovSameSigma}
\end{equation}
with $\sigma^2 = E \left( u' u \right)$.
If the disturbance terms of the individual equations
are allowed to have different variances, 
the covariance matrix of the estimated parameters is
\begin{equation}
   \Cov \left[ \widehat{\beta}_{2SLS} \right] = \left( \widehat{X}' \Omega^{-1}
   \widehat{X} \right)^{-1}
   \label{eq:2slsCovSingleSigma}
\end{equation}
with $\Omega = \Sigma \otimes I$, 
$\sigma_{ij} = 0 \; \forall \; i \neq j$ and
$\sigma_{ii} = E \left( u_i' u_i \right)$.


%%%%%%%%%%%%%%%%%%%%%%%%%%%%%%%%%%%%%%%%%%%%%%%
\subsubsection{Weighted two-stage least squares (W2SLS)}

The Weighted Two-Stage Least Squares (W2SLS) estimator of the system 
is obtained by
\begin{equation}
   \widehat{\beta}_{W2SLS} = \left( \widehat{X}' \Omega^{-1} \widehat{X} 
   \right)^{-1} \widehat{X}' \Omega^{-1} y
\end{equation}
with $\Omega = \Sigma \otimes I$, 
$\sigma_{ij} = 0 \; \forall \; i \neq j$ and
$\sigma_{ii} = E \left( u_i' u_i \right)$.
The covariance matrix of the estimated parameters is
\begin{equation}
   \Cov \left[ \widehat{\beta}_{W2SLS} \right] = \left( \widehat{X}' \Omega^{-1}
   \widehat{X} \right)^{-1}
\end{equation}


%%%%%%%%%%%%%%%%%%%%%%%%%%%%%%%%%%%%%%%%%%%%%%%
\subsubsection{Three-stage least squares (3SLS)}

If the regressors are correlated with the disturbances 
($E \left( u | X \right) \neq 0$) and 
the disturbances are contemporaneously correlated, 
a Generalized Least Squares (GLS) version of the two-stage least squares
estimation leads to consistent and efficient estimates.
This estimation procedure is generally called �Three-stage Least
Squares� (3SLS) \citep{zellner62b}.

The standard 3SLS estimator can be obtained by:
\begin{equation}
   \widehat{\beta}_{3SLS} = \left( \widehat{X}' \Omega^{-1} \widehat{X} 
   \right)^{-1} \widehat{X}' \Omega^{-1} y
   \label{eq:3slsGls}
\end{equation}
with $\Omega = \Sigma \otimes I$ and
$\sigma_{ij} = E \left( u_i' u_j \right)$.
Its covariance matrix is:
\begin{equation}
   \Cov \left[ \widehat{\beta}_{3SLS} \right] = \left( \widehat{X}' \Omega^{-1}
   \widehat{X} \right)^{-1}
   \label{eq:cov3sls}
\end{equation}
While an unbiased 2SLS or W2SLS estimation requires only that
the instrumental variables and the disturbance terms of each single 
equation are uncorrelated
$( E \left[ u_i | H_i \right]) = 0 \; \forall \; i )$,
\cite{schmidt90} points out that this estimator is only consistent 
if all disturbance terms and all instrumental variables are uncorrelated
$( E \left[ u | H \right]) = 0 )$
with
\begin{equation}
   H =
   \left[ \begin{array}{cccc}
      H_1 & 0 & \cdots & 0\\
      0 & H_2 & \cdots & 0\\
      \vdots & \vdots & \ddots & \vdots\\
      0 & 0 & \cdots & H_G
   \end{array}\right]
\end{equation}
Since there might be occasions where this cannot be avoided,
\cite{schmidt90} analyses other approaches to obtain 3SLS estimators:

One of these approaches is based on instrumental variable estimation
(3SLS-IV):
\begin{equation}
   \widehat{\beta}_{3SLS-IV} = \left( \widehat{X}' \Omega^{-1} X
   \right)^{-1} \widehat{X}' \Omega^{-1} y
   \label{eq:3slsIv}
\end{equation}
The covariance matrix of this 3SLS-IV estimator is:
\begin{equation}
   \Cov \left[ \widehat{\beta}_{3SLS-IV} \right] = \left( \widehat{X}' \Omega^{-1}
   X \right)^{-1}
\end{equation}
Another approach is based on the Generalized Method of Moments (GMM)
estimator (3SLS-GMM):
\begin{equation}
   \widehat{\beta}_{3SLS-GMM} = \left( X' H \left( H' \Omega H \right)^{-1}
   H' X \right)^{-1} X' H \left( H' \Omega H \right)^{-1} H' y
   \label{eq:3slsGmm}
\end{equation}
The covariance matrix of the 3SLS-GMM estimator is:
\begin{equation}
   \Cov \left[ \widehat{\beta}_{3SLS-GMM} \right] =
   \left( X' H \left( H' \Omega H \right)^{-1} H' X \right)^{-1}
\end{equation}
A fourth approach developed by \cite{schmidt90} himself is:
\begin{equation}
   \widehat{\beta}_{3SLS-Schmidt} = \left( \widehat{X}' \Omega^{-1} \widehat{X}
   \right)^{-1} \widehat{X}' \Omega^{-1} 
   H \left( H' H \right)^{-1} H' y 
   \label{eq:3slsSchmidt}
\end{equation}
The covariance matrix of this estimator is:
\begin{equation}
   \Cov \left[ \widehat{\beta}_{3SLS-Schmidt} \right] =
   \left( \widehat{X}' \Omega^{-1}  \widehat{X} \right)^{-1} 
   \widehat{X}' \Omega^{-1} H \left( H' H \right)^{-1} H' \Omega 
   H \left( H' H \right)^{-1} H' \Omega^{-1} \widehat{X}
   \left( \widehat{X}' \Omega^{-1}  \widehat{X} \right)^{-1}
\end{equation}
The econometrics software EViews uses following approach:
\begin{equation}
   \widehat{\beta}_{3SLS-EViews} = \widehat{\beta}_{2SLS} + 
   \left( \widehat{X}' \Omega^{-1} \widehat{X} \right)^{-1} 
   \widehat{X}' \Omega^{-1} \left( y - X \widehat{\beta}_{2SLS} \right)
   \label{eq:3slsEViews}
\end{equation}
where $\widehat{\beta}_{2SLS}$ is the two-stage least squares estimator
as defined by (\ref{eq:beta2sls}).
EViews uses the standard 3SLS formula (\ref{eq:cov3sls}) to
calculate the covariance matrix of the 3SLS estimator.


If the same instrumental variables are used in all equations 
($H_1 = H_2 = \ldots = H_G$), 
all the above mentioned approaches lead to identical parameter estimates.
However, if this is not the case, the results depend on the 
method used \citep{schmidt90}.
The only reason to use different instruments for different equations
is a correlation of the instruments of one equation with the
disturbance terms of another equation.
Otherwise, one could simply use all instruments in every equation
\citep{schmidt90}.
In this case, only the 3SLS-GMM (\ref{eq:3slsGmm})
and the 3SLS estimator developed by \cite{schmidt90} 
(\ref{eq:3slsSchmidt}) are consistent.



%%% Local Variables: 
%%% mode: latex
%%% TeX-master: "systemfit"
%%% End: 

%%%%%%%%%%%%%%%%%%%%%%%%%%%%%%%%%%%%%%%%%%%%%%%%%%%%%%
\subsection{Imposing linear restrictions}\label{sec:Restrictions}
%%%%%%%%%%%%%%%%%%%%%%%%%%%%%%%%%%%%%%%%%%%%%%%%%%%%%%

There are two ways to impose linear parameter restrictions.
First, a matrix $T$ can be specified that
\begin{equation}
   \beta = T \cdot \beta^* \label{eq:T-restr} 
\end{equation}
where $\beta^*$ is a vector of restricted (linear independent) coefficients,
and $T$ is a matrix with the number of rows equal to the number of
unrestricted coefficients ($\beta$) and
the number of columns equal to the number of restricted coefficients
($\beta^*$).
$T$ can be used to map each unrestricted coefficient to one or more
restricted coefficients.

To impose these restrictions, the $X$ matrix is
(post-)multiplied by this $T$ matrix.
\begin{equation}
    X^* = X \cdot T
\end{equation}
Then, $X^*$ is substituted for $X$ and a standard estimation as described
in the previous section is done
(equations~\ref{eq:ols}--\ref{eq:3slsEViews}).
This results in the linear independent parameter estimates $\beta^*$ and
their covariance matrix.
The original parameters can be obtained by equation (\ref{eq:T-restr})
and the covariance matrix of the original parameters
can be obtained by:
\begin{equation}
   \Cov \left[ \widehat{\beta} \right] = T \cdot \Cov \left[ \widehat{\beta}^* \right] \cdot T'
\end{equation}

The second way to impose linear parameter restrictions
can be formulated by
\begin{equation}
   R \beta^0 = q
\end{equation}
where $\beta^0$ is the vector of the restricted coefficients, 
and $R$ and $q$ are a matrix and vector, respectively,
to impose the restrictions \citep[see][p.\ 100]{greene03}.
Each linear independent restriction is represented by one row of $R$
and the corresponding element of~$q$.

The first way is less flexible than this latter one%
\footnote{
While restrictions like $\beta_1 = 2 \beta_2$ can be imposed by
both methods,
restrictions like $\beta_1 + \beta_2 = 4$ can be imposed only
by the second method.
}, 
but the first way is preferable if equality constraints for coefficients
across many equations of the system are imposed. 
Of course, these restrictions can be also imposed using
the latter method.
However, while the latter method increases the dimension of the 
matrices to be inverted during estimation, the first reduces it. 
Thus, in some cases the latter way leads to estimation problems
(e.g.\ (near) singularity of the matrices to be inverted),
while the first doesn't.

These two methods can be combined.
In this case the restrictions imposed using the latter method are
imposed on the linear independent parameters due to the restrictions
imposed using the first method:
\begin{equation}
   R \beta^{*0} = q
\end{equation}
where $\beta^{*0}$ is the vector of the restricted $\beta^*$ coefficients.

%%%%%%%%%%%%%%%%%%%%%%%%%%%%%%%%%%%%%%%%%%%%%%%%%%
\subsubsection{Restricted OLS estimation}

The OLS estimator restricted by $R \beta^0 = q$ can be obtained by
\begin{equation}
   \left[ \begin{array}{c}
      \widehat{\beta}^0_{OLS} \\ \widehat{\lambda}
   \end{array} \right]
   =
   \left[ \begin{array}{cc}
      X' X & R' \\ 
      R & 0
   \end{array} \right]^{-1}
   \cdot
   \left[ \begin{array}{c}
      X' y \\ q 
   \end{array} \right]
\end{equation}
where $\lambda$ is a vector of the Lagrangean multipliers of the restrictions.
If the whole system is treated as one single equation,
the covariance matrix of the estimated parameters is
\begin{equation}
   \Cov 
   \left[ \begin{array}{c}
      \widehat{\beta}^0_{OLS} \\ \widehat{\lambda}
   \end{array} \right] 
   = \sigma^2 
   \left[ \begin{array}{cc}
      X' X & R' \\ 
      R & 0
   \end{array} \right]^{-1}
\end{equation}
with $\sigma^2 = E \left( u' u \right)$.
If the disturbance terms of the individual equations
are allowed to have different variances, 
the covariance matrix of the estimated parameters is
\begin{equation}
   \Cov 
   \left[ \begin{array}{c}
      \widehat{\beta}^0_{OLS} \\ \widehat{\lambda}
   \end{array} \right] 
   = 
   \left[ \begin{array}{cc}
      X' \Omega^{-1} X & R' \\ 
      R & 0
   \end{array} \right]^{-1}
\end{equation}
with $\Omega = \Sigma \otimes I$, 
$\sigma_{ij} = 0 \; \forall \; i \neq j$ and
$\sigma_{ii} = E \left( u_i' u_i \right)$.

%%%%%%%%%%%%%%%%%%%%%%%%%%%%%%%%%%%%%%%%%%%%%%%%%%
\subsubsection{Restricted WLS estimation}

The WLS estimator restricted by $R \beta^0 = q$ can be obtained by
\begin{equation}
   \left[ \begin{array}{c}
      \widehat{\beta}^0_{WLS} \\ \widehat{\lambda}
   \end{array} \right]
   =
   \left[ \begin{array}{cc}
      X' \Omega^{-1} X & R' \\ 
      R & 0
   \end{array} \right]^{-1}
   \cdot
   \left[ \begin{array}{c}
      X' \Omega^{-1} y \\ q 
   \end{array} \right]
\end{equation}
with $\Omega = \Sigma \otimes I$, 
$\sigma_{ij} = 0 \; \forall \; i \neq j$ and
$\sigma_{ii} = E \left( u_i' u_i \right)$.
The covariance matrix of the estimated parameters is
\begin{equation}
   \Cov 
   \left[ \begin{array}{c}
      \widehat{\beta}^0_{WLS} \\ \widehat{\lambda}
   \end{array} \right] 
   = 
   \left[ \begin{array}{cc}
      X' \Omega^{-1} X & R' \\ 
      R & 0
   \end{array} \right]^{-1}
\end{equation}

%%%%%%%%%%%%%%%%%%%%%%%%%%%%%%%%%%%%%%%%%%%%%%%%%%
\subsubsection{Restricted SUR estimation}

The SUR estimator restricted by $R \beta^0 = q$ can be obtained by
\begin{equation}
   \left[ \begin{array}{c}
      \widehat{\beta}^0_{SUR} \\ \widehat{\lambda}
   \end{array} \right]
   =
   \left[ \begin{array}{cc}
      X' \Omega^{-1} X & R' \\ 
      R & 0
   \end{array} \right]^{-1}
   \cdot
   \left[ \begin{array}{c}
      X' \Omega^{-1} y \\ q 
   \end{array} \right]
\end{equation}
with $\Omega = \Sigma \otimes I$ and
$\sigma_{ij} = E \left( u_i' u_j \right)$.
The covariance matrix of the estimated parameters is
\begin{equation}
   \Cov 
   \left[ \begin{array}{c}
      \widehat{\beta}^0_{SUR} \\ \widehat{\lambda}
   \end{array} \right] 
   = 
   \left[ \begin{array}{cc}
      X' \Omega^{-1} X & R' \\ 
      R & 0
   \end{array} \right]^{-1}
\end{equation}

%%%%%%%%%%%%%%%%%%%%%%%%%%%%%%%%%%%%%%%%%%%%%%%%%%
\subsubsection{Restricted 2SLS estimation}

The 2SLS estimator restricted by $R \beta^0 = q$ can be obtained by
\begin{equation}
   \left[ \begin{array}{c}
      \widehat{\beta}^0_{2SLS} \\ \widehat{\lambda}
   \end{array} \right]
   =
   \left[ \begin{array}{cc}
      \widehat{X}' \widehat{X} & R' \\ 
      R & 0
   \end{array} \right]^{-1}
   \cdot
   \left[ \begin{array}{c}
      \widehat{X}' y \\ q 
   \end{array} \right]
   \label{eq:beta2SLSr}
\end{equation}
If the whole system is treated as one single equation,
the covariance matrix of the estimated parameters is
\begin{equation}
   \Cov 
   \left[ \begin{array}{c}
      \widehat{\beta}^0_{2SLS} \\ \widehat{\lambda}
   \end{array} \right] 
   = \sigma^2 
   \left[ \begin{array}{cc}
      \widehat{X}' \widehat{X} & R' \\ 
      R & 0
   \end{array} \right]^{-1}
\end{equation}
with $\sigma^2 = E \left( u' u \right)$.
If the disturbance terms of the individual equations
are allowed to have different variances, 
the covariance matrix of the estimated parameters is
\begin{equation}
   \Cov 
   \left[ \begin{array}{c}
      \widehat{\beta}^0_{2SLS} \\ \widehat{\lambda}
   \end{array} \right] 
   = 
   \left[ \begin{array}{cc}
      \widehat{X}' \Omega^{-1} \widehat{X} & R' \\ 
      R & 0
   \end{array} \right]^{-1}
\end{equation}
with $\Omega = \Sigma \otimes I$, 
$\sigma_{ij} = 0 \; \forall \; i \neq j$ and
$\sigma_{ii} = E \left( u_i' u_i \right)$.


%%%%%%%%%%%%%%%%%%%%%%%%%%%%%%%%%%%%%%%%%%%%%%%%%%
\subsubsection{Restricted W2SLS estimation}

The W2SLS estimator restricted by $R \beta^0 = q$ can be obtained by
\begin{equation}
   \left[ \begin{array}{c}
      \widehat{\beta}^0_{W2SLS} \\ \widehat{\lambda}
   \end{array} \right]
   =
   \left[ \begin{array}{cc}
      \widehat{X}' \Omega^{-1} \widehat{X} & R' \\ 
      R & 0
   \end{array} \right]^{-1}
   \cdot
   \left[ \begin{array}{c}
      \widehat{X}' \Omega^{-1} y \\ q 
   \end{array} \right]
\end{equation}
with $\Omega = \Sigma \otimes I$, 
$\sigma_{ij} = 0 \; \forall \; i \neq j$ and
$\sigma_{ii} = E \left( u_i' u_i \right)$.
The covariance matrix of the estimated parameters is
\begin{equation}
   \Cov 
   \left[ \begin{array}{c}
      \widehat{\beta}^0_{W2SLS} \\ \widehat{\lambda}
   \end{array} \right] 
   = 
   \left[ \begin{array}{cc}
      \widehat{X}' \Omega^{-1} \widehat{X} & R' \\ 
      R & 0
   \end{array} \right]^{-1}
\end{equation}


%%%%%%%%%%%%%%%%%%%%%%%%%%%%%%%%%%%%%%%%%%%%%%%%%%
\subsubsection{Restricted 3SLS estimation}

The standard 3SLS estimator restricted by $R \beta^0 = q$ can be obtained by
\begin{equation}
   \left[ \begin{array}{c}
      \widehat{\beta}^0_{3SLS} \\ \widehat{\lambda}
   \end{array} \right]
   =
   \left[ \begin{array}{cc}
      \widehat{X}' \Omega^{-1} \widehat{X} & R' \\ 
      R & 0
   \end{array} \right]^{-1}
   \cdot
   \left[ \begin{array}{c}
      \widehat{X}' \Omega^{-1} y \\ q 
   \end{array} \right]
   \label{eq:3slsGlsR}
\end{equation}
with $\Omega = \Sigma \otimes I$ and
$\sigma_{ij} = E \left( u_i' u_j \right)$.
The covariance matrix of this estimator is
\begin{equation}
   \Cov 
   \left[ \begin{array}{c}
      \widehat{\beta}^0_{3SLS} \\ \widehat{\lambda}
   \end{array} \right] 
   = 
   \left[ \begin{array}{cc}
      \widehat{X}' \Omega^{-1} \widehat{X} & R' \\ 
      R & 0
   \end{array} \right]^{-1}
   \label{eq:cov3slsr}
\end{equation}
The 3SLS-IV estimator restricted by $R \beta^0 = q$ can be obtained by
\begin{equation}
   \left[ \begin{array}{c}
      \widehat{\beta}^0_{3SLS-IV} \\ \widehat{\lambda}
   \end{array} \right]
   =
   \left[ \begin{array}{cc}
      \widehat{X}' \Omega^{-1} X & R' \\ 
      R & 0
   \end{array} \right]^{-1}
   \cdot
   \left[ \begin{array}{c}
      \widehat{X}' \Omega^{-1} y \\ q 
   \end{array} \right]
   \label{eq:3slsIvR}
\end{equation}
with
\begin{equation}
   \Cov 
   \left[ \begin{array}{c}
      \widehat{\beta}^0_{3SLS-IV} \\ \widehat{\lambda}
   \end{array} \right] 
   = 
   \left[ \begin{array}{cc}
      \widehat{X}' \Omega^{-1} \widehat{X} & R' \\ 
      R & 0
   \end{array} \right]^{-1}
\end{equation}
The restricted 3SLS-GMM estimator can be obtained by
\begin{equation}
   \left[ \begin{array}{c}
      \widehat{\beta}^0_{3SLS-GMM} \\ \widehat{\lambda}
   \end{array} \right]
   =
   \left[ \begin{array}{cc}
      X' H \left( H' \Omega H \right)^{-1} H' X & R' \\ 
      R & 0
   \end{array} \right]^{-1}
   \cdot
   \left[ \begin{array}{c}
      X' H \left( H \Omega H \right)^{-1} H' y \\ q 
   \end{array} \right]
   \label{eq:3slsGmmR}
\end{equation}
with
\begin{equation}
   \Cov 
   \left[ \begin{array}{c}
      \widehat{\beta}^0_{3SLS-GMM} \\ \widehat{\lambda}
   \end{array} \right] 
   = 
   \left[ \begin{array}{cc}
      X' H \left( H' \Omega H \right)^{-1} H' X & R' \\ 
      R & 0
   \end{array} \right]^{-1}
\end{equation}
The restricted 3SLS estimator based on the suggestion of
\cite{schmidt90} is:
\begin{equation}
   \left[ \begin{array}{c}
      \widehat{\beta}^0_{3SLS-Schmidt} \\ \widehat{\lambda}
   \end{array} \right]
   =
   \left[ \begin{array}{cc}
      \widehat{X}' \Omega^{-1} \widehat{X} & R' \\ 
      R & 0
   \end{array} \right]^{-1}
   \cdot
   \left[ \begin{array}{c}
      \widehat{X}' \Omega^{-1} H \left( H' H \right)^{-1} H' y \\ q 
   \end{array} \right]
   \label{eq:3slsSchmidtR}
\end{equation}
with
\begin{eqnarray}
   \Cov 
   \left[ \begin{array}{c}
      \widehat{\beta}^0_{3SLS-Schmidt} \\ \widehat{\lambda}
   \end{array} \right] 
   & = & 
   \left[ \begin{array}{cc}
      \widehat{X}' \Omega^{-1} \widehat{X} & R' \\ 
      R & 0
   \end{array} \right]^{-1}
   \\
   & & \cdot
   \left[ \begin{array}{cc}
      \widehat{X}' \Omega^{-1} H \left( H' H \right)^{-1} H' \Omega
      H \left( H' H \right)^{-1} H' \Omega^{-1} \widehat{X} & 0' \\ 
      0 & 0
   \end{array} \right]^{-1}
   \nonumber \\
   & & \cdot
   \left[ \begin{array}{cc}
      \widehat{X}' \Omega^{-1} \widehat{X} & R' \\ 
      R & 0
   \end{array} \right]^{-1}
   \nonumber
\end{eqnarray}
The econometrics software EViews calculates the restricted 3SLS estimator by:
\begin{equation}
   \left[ \begin{array}{c}
      \widehat{\beta}^0_{3SLS-EViews} \\ \widehat{\lambda}
   \end{array} \right]
   =
   \left[ \begin{array}{cc}
      \widehat{X}' \Omega^{-1} \widehat{X} & R' \\ 
      R & 0
   \end{array} \right]^{-1}
   \cdot
   \left[ \begin{array}{c}
      \widehat{X}' \Omega^{-1} \left( y - X \widehat{\beta}^0_{2SLS} \right)
      \\ q 
   \end{array} \right]
   \label{eq:3slsEViewsR}
\end{equation}
where $\widehat{\beta}^0_{2SLS}$ is the restricted 2SLS estimator calculated
by equation (\ref{eq:beta2SLSr}). 
To calculate the covariance matrix
EViews uses the standard formula of the restricted 3SLS
estimator~(\ref{eq:cov3slsr}).


If the same instrumental variables are used in all equations 
($H_1 = H_2 = \ldots = H_G$), 
all the above mentioned approaches lead to identical parameter estimates
and identical covariance matrices of the estimated parameters.

%%% Local Variables: 
%%% mode: latex
%%% TeX-master: "systemfit"
%%% End: 


%       $Id$    

%%%%%%%%%%%%%%%%%%%%%%%%%%%%%%%%%%%%%%%%%%%%%%%%%%
%\section{Other issues and tools}\label{sec:Other}
%%%%%%%%%%%%%%%%%%%%%%%%%%%%%%%%%%%%%%%%%%%%%%%%%%

%%%%%%%%%%%%%%%%%%%%%%%%%%%%%%%%%%%%%%%%%%%%%%%%%%
\subsection{Residual covariance matrix}\label{sec:residcov}

Since the true residuals of the estimated equations are generally not known,
the true covariance matrix of the residuals cannot be determined.
Thus, this covariance matrix must be calculated from the
\emph{estimated} residuals. 
Generally, the estimated covariance matrix of the residuals
($\widehat{\Sigma} = \left[ \widehat{\sigma}_{ij} \right]$)
can be calculated from the residuals of a first-step OLS or 2SLS estimation.
The following formula is often applied:
\begin{equation}
   \widehat{\sigma}_{ij} = \frac{ \widehat{u}_i' \widehat{u}_j }{ T }
   \label{eq:rcov0}
\end{equation}
where $T$ is the number of observations in each equation.
However, in finite samples this estimator is biased,
because it is not corrected for degrees of freedom.
The usual single-equation procedure to correct for degrees of freedom
cannot always be applied, because the number of regressors in each equation
might differ.
Two alternative approaches to calculate the residual covariance
matrix are
\begin{equation}
   \widehat{\sigma}_{ij} = \frac{ \widehat{u}_i' \widehat{u}_j }
   { \sqrt{ \left( T - K_i \right) \cdot \left( T - K_j \right) } }
   \label{eq:rcov1}
\end{equation}
and
\begin{equation}
   \widehat{\sigma}_{ij} = \frac{ \widehat{u}_i' \widehat{u}_j }
   { T - \max \left( K_i , K_j \right) }
   \label{eq:rcov3}
\end{equation}
where $K_i$ and $K_j$ are the number of regressors in equation
$i$ and $j$, respectively.
However, these formulas yield unbiased estimators only if $K_i = K_j$
\citep[p.\ 469]{judge85}. 
% Greene (2003, p. 344) says that the second is unbiased if i = j or K_i = K_j,
% whereas the first is unbiased only if i = j. 
% However, if K_i = K_j the first and the second are equal.
% Why is the first biased if K_i = K_j ???????????


A further approach to obtain the estimated residual covariance
matrix is \citep[p.\ 309]{zellner62c}
\begin{eqnarray}
   \widehat{\sigma}_{ij} & = & 
   \frac{ \widehat{u}_i' \widehat{u}_j } 
   { T - K_i - K_j + tr \left[ X_i \left( X_i' X_i \right)^{-1}
   X_i' X_j \left( X_j' X_j \right)^{-1} X_j' \right] }
   \label{eq:rcov2} \\
   & = &
   \frac{ \widehat{u}_i' \widehat{u}_j } 
   { T - K_i - K_j + tr \left[ \left( X_i' X_i \right)^{-1}
   X_i' X_j \left( X_j' X_j \right)^{-1} X_j' X_i \right] }
\end{eqnarray} 
This yields an unbiased estimator for all elements of $\widehat{\Sigma}$,
but even if $\widehat{\Sigma}$ is an unbiased estimator of $\Sigma$, 
its inverse $\widehat{\Sigma}^{-1}$ is not an unbiased estimator 
of $\Sigma^{-1}$ \citep[p.\ 322]{theil71}.
Furthermore, the covariance matrix calculated by (\ref{eq:rcov2})
is not necessarily positive semidefinite \citep[p.\ 322]{theil71}. 
Hence, �it is doubtful whether [this formula] is really superior to 
[(\ref{eq:rcov0})]� \citep[p.\ 322]{theil71}.


The WLS, SUR, W2SLS and 3SLS parameter estimates are consistent,
if the estimated residual covariance matrix is calculated
using the residuals from a first-step OLS or 2SLS estimation.
There exists also an alternative slightly different approach.%
\footnote{
For instance, this approach is applied by
the command �TSCS� of the software LIMDEP that carries out SUR estimations
in which all coefficient vectors are constrained to be equal
\citep{greene06}.
}
This alternative approach uses the residuals of a first-step OLS or 2SLS estimation
to apply a WLS or W2SLS estimation on a second step.
Then, it calculates the residual covariance matrix
from the residuals of the second-step estimation
to estimates the model by SUR or 3SLS in a third step.
If no cross-equation restrictions are imposed,
the parameter estimates of OLS and WLS as well as 2SLS and W2SLS are identical.
Hence, in this case both approaches generate the same results.

It is also possible to iterate WLS, SUR, W2SLS and 3SLS estimations.
At each iteration the residual covariance matrix is calculated
from the residuals of the previous iteration.
If equation (\ref{eq:rcov0}) is applied to calculate the estimated
residual covariance matrix,
an iterated SUR estimation converges to maximum
likelihood \citep[p.\ 345]{greene03}.

In some uncommon cases,
for instance in pooled estimations,
where the coefficients are restricted to be equal in all equations,
the means of the residuals of each equation are not equal to zero
$( \overline{ \widehat{u} }_i \neq 0 )$.
Therefore, we subtract the means from the residuals
and substitute $\widehat{u}_i - \overline{ \widehat{u} }_i$
for $\widehat{u}_i$ in (\ref{eq:rcov0}--\ref{eq:rcov2}).


%%%%%%%%%%%%%%%%%%%%%%%%%%%%%%%%%%%%%%%%%%%%%%%%%%
\subsection{Degrees of freedom}
\label{sec:degreesOfFreedom}

To our knowledge the question about how to determine the degrees
of freedom for single-parameter t-tests is not comprehensively
discussed in the literature.
While sometimes the degrees of freedom of the entire system
(total number of observations in all equations minus
total number of estimated parameters)
are applied,
in other cases the degrees of freedom of each single equation
(number of observations in the equations minus
number of estimated parameters in the equation)
are used.
Asymptotically, this distinction doesn't make a difference.
However, in many empirical applications, the number of observations
of each equation is rather small, and
therefore, it matters.

If a system of equations is estimated by an unrestricted OLS and
the covariance matrix of the parameters is calculated
by~(\ref{eq:olsCovSingleSigma}),
the estimated parameters and their standard errors are identical
to an equation-wise OLS estimation.
In this case, it is reasonable to use the degrees of freedom of
each single equation,
because this yields the same p-values as the equation-wise
OLS estimation.

In contrast, if a system of equations is estimated with many
cross-equation restrictions and
the covariance matrix of an OLS estimation is calculated
by~(\ref{eq:olsCovSameSigma}),
the system estimation is similar to a single equation estimation.
Therefore, in this case, it seems to be reasonable to use the degrees
of freedom of whole system.



%%%%%%%%%%%%%%%%%%%%%%%%%%%%%%%%%%%%%%%%%%%%%%%%%%
\subsection{Goodness of fit}

The goodness of fit of each single equation can be measured by the
traditional $R^2$ values:
\begin{equation}
   R_i^2 = 1 - \frac{ \widehat{u}_i' \widehat{u}_i }
   { ( y_i - \overline{y_i} )' ( y_i - \overline{y_i} ) }
\end{equation}
where $R_i^2$ is the $R^2$ value of the $i$th equation
and $\overline{y_i}$ is the mean value of $y_i$.

The goodness of fit of the whole system can be measured by the
McElroy's $R^2$ value \citep{mcelroy77}: 
% also: \citep[p.\ 345]{greene03}
\begin{equation}
   R_*^2 = 1 - \frac{ \widehat{u}' \widehat{ \Omega }^{-1} \widehat{u} }
   { y' \left( \widehat{ \Sigma }^{-1} \otimes
   \left( I - \frac{i i'}{T} \right) \right) y }
\end{equation}
where $T$ is the number of observations in each equation,
$I$ is an $T \times T$ identity matrix and 
$i$ is a column vector of $T$ ones.


%%%%%%%%%%%%%%%%%%%%%%%%%%%%%%%%%%%%%%%%%%%%%%%%%%
\subsection{Testing linear restrictions}
\label{sec:testingRestrictions}

Linear restrictions can be tested by an F test, Wald test or
likelihood-ratio (LR) test.

The F-statistic for systems of equations is
\begin{equation}
F = \frac{
   ( R \hat{\beta} - q )'
   ( R ( X' ( \hat{\Sigma} \otimes I )^{-1} X )^{-1} R' )^{-1}
   ( R \hat{\beta} - q ) /
   j
}{
   \hat{u}' ( \Sigma \otimes I )^{-1} \hat{u} /
   ( M \cdot T - K )
}
\end{equation}
where $j$ is the number of restrictions,
$M$ is the number of equations,
$T$ is the number of observations per equation,
$K$ is the total number of estimated coefficients, and
$\hat{\Sigma}$ is the estimated residual covariance matrix
used in the estimation.
Under the null hypothesis, $F$ has an F-distribution
with $j$ and $M \cdot T - K$ degrees of freedom
\citep[p.\ 314]{theil71}.

The Wald-statistic for systems of equations is
\begin{equation}
W =
   ( R \hat{\beta} - q )'
   ( R \widehat{Cov} [ \hat{\beta} ] R' )^{-1}
   ( R \hat{\beta} - q )
\end{equation}
Asymptotically, $W$ has a $\chi^2$
distribution with $j$ degrees of freedom
under the null hypothesis
\citep[p.\ 347]{greene03}.

The LR-statistic for systems of equations is
\begin{equation}
LR = T \cdot \left(
   log \left| \hat{ \hat{ \Sigma } }_r \right|
   - log \left| \hat{ \hat{ \Sigma } }_u \right|
   \right)
\end{equation}
where $T$ is the number of observations per equation, and 
$\hat{\hat{\Sigma}}_r$ and $\hat{\hat{\Sigma}}_u$ are
the residual covariance matrices calculated by formula (\ref{eq:rcov0})
of the restricted and unrestricted estimation, respectively.
Asymptotically, $LR$ has a $\chi^2$
distribution with $j$ degrees of freedom
under the null hypothesis
\citep[p.\ 349]{greene03}.



%%%%%%%%%%%%%%%%%%%%%%%%%%%%%%%%%%%%%%%%%%%%%%%%%%
\subsection{Hausman test}
\label{sec:hausman}

Hausman \citep{hausman78} developed a test for misspecification.
The null hypotheses of the test is that all exogenous variables are
uncorrelated with all disturbance terms.
Under this hypothesis both the 2SLS and the 3SLS estimator are consistent
but only the 3SLS estimator is (asymptotically) efficient.
Under the alternative hypothesis the 2SLS estimator is consistent
but the 3SLS estimator is inconsistent.
The Hausman test statistic is,
\begin{equation}
  m = \left( \hat{\beta}_2 - \hat{\beta}_3 \right)^{'}
      \left( \Cov \left[ \hat{\beta}_2 \right] -
             \Cov \left[ \hat{\beta}_3 \right] \right)
      \left( \hat{\beta}_2 - \hat{\beta}_3 \right)
\label{eq:hausman}
\end{equation}
where $\hat{\beta}_2$ and $\Cov \left[ \hat{\beta}_2 \right]$ are the estimated
coefficient and covariance matrix from 2SLS estimation, and
$\hat{\beta}_3$ and $\Cov \left[ \hat{\beta}_3 \right]$ are the estimated
coefficients and covariance matrix from 3SLS estimation.
Under the null hypotheses this test statistic has a
$\chi^2$ distribution with degrees of freedom equal to the number of
estimated parameters.




%%% Local Variables: 
%%% mode: latex
%%% TeX-master: "systemfit"
%%% End: 


%%% Local Variables: 
%%% mode: latex
%%% TeX-master: "systemfit"
%%% End: 
