\documentclass[12pt]{scrartcl}
\usepackage[utf8]{inputenc}
\usepackage[T1]{fontenc}
\usepackage{textcomp}
\usepackage{lmodern}

\usepackage{geometry}
\geometry{verbose, a4paper, tmargin=25mm, bmargin=30mm, lmargin=25mm,
   rmargin=25mm, headheight=0mm, headsep=0mm, footskip=13mm}
\setlength{\emergencystretch}{2em}
\usepackage{setspace}
\onehalfspacing

\usepackage{csquotes}
\MakeOuterQuote{°}

\usepackage{natbib}
\bibliographystyle{jss}

\newcommand{\code}[1]{\texttt{#1}}
\newcommand{\pkg}[1]{\mbox{\textbf{#1}}}
\newcommand{\proglang}[1]{\mbox{\textsf{#1}}}
\newcommand{\hz}{\hspace{0pt}}

\author{Arne Henningsen and Jeff D.\ Hamann\\
   \normalsize University of Kiel and Forest Informatics, Inc.}

\title{\pkg{systemfit}: A Package for Estimating Systems
of Simultaneous Equations in \proglang{R}}

\date{March 31, 2008}

\begin{document}
\maketitle

Many theoretical models that are econometrically estimated
consist of more than one equation.
The disturbance terms of these equations are likely
to be contemporaneously correlated,
because unconsidered factors
that influence the disturbance term in one equation
probably influence the disturbance terms in other equations, too.
Ignoring this contemporaneous correlation
and estimating these equations separately
leads to inefficient estimates of the coefficients.
However, estimating all equations simultaneously
with a °generalized least squares° (GLS) estimator,
which takes the covariance structure of the residuals into account,
leads to efficient estimates.
This estimation procedure is generally called
°seemingly unrelated regression° \citep[SUR,][]{zellner62}.
Another reason to estimate a system of equations simultaneously are
cross-equation restrictions on the coefficients.
Estimating the coefficients under cross-equation restrictions
and testing these restrictions
requires a simultaneous estimation approach.

Furthermore, these models can contain variables
that appear on the left-hand side in one equation
and on the right-hand side of another equation.
Ignoring the endogeneity of these variables can lead to inconsistent
estimates.
This simultaneity bias can be corrected for by
applying a °two-stage least squares° (2SLS) estimation
to each equation.
Combining this estimation method with the SUR method results
in a simultaneous estimation of the system of equations
by the °three-stage least squares° (3SLS) method
\citep{zellner62b}.

The \pkg{systemfit} package provides the capability to estimate
systems of linear equations in \proglang{R}
\citep{r-project}.
Currently, the estimation methods
°ordinary least squares° (OLS),
°weighted least squares° (WLS),
°seemingly unrelated regression° (SUR),
°two-stage least squares° (2SLS),
°weighted two-stage least squares° (W2SLS), and
°three-stage least squares° (3SLS)
are implemented.
The WLS, SUR, W2SLS, and 3SLS estimates can be based
either on one-step (OLS or 2SLS) (co)variances
or these estimations can be iterated,
where the (co)variances are calculated from the estimates of the previous step.
Furthermore,
the \pkg{systemfit} package
provides statistical tests for restrictions on the coefficients
and for testing the consistency of the 3SLS estimation.

Although systems of linear equations can be estimated
with several other statistical and econometric software packages
(e.g., \proglang{SAS}, \proglang{EViews}, \proglang{TSP}),
\pkg{systemfit} has several advantages.
First, all estimation procedures are publicly available in the source code.
Second, the estimation algorithms can be easily modified to meet specific
requirements.
Third, the (advanced) user can control estimation details generally
not available in other software packages by overriding reasonable defaults.

A special focus of our presentation at the useR! conference
should be the changes and improvements of the \pkg{systemfit} package
that were introduced in version~1.0
(published in the very end of December 2007).

\bibliography{systemfit}

\end{document}
