
%       $Id$    


\documentclass[article]{jss}
\usepackage{csquotes}
\MakeOuterQuote{�}

\author{Arne Henningsen\\University of Kiel\And
        Jeff D. Hamann\\ Forest Informatics, Inc. }
        
\title{systemfit: A Package to Estimate\\
       Simultaneous Equation Systems in \proglang{R}}

%% for pretty printing and a nice hypersummary also set:
\Plainauthor{Arne Henningsen, Jeff D. Hamann} %% comma-separated
\Plaintitle{systemfit: A Package to Estimate Simultaneous Equation Systems 
            in R} %% without formatting
\Shorttitle{systemfit} %% a short title (if necessary)

%% an abstract and keywords
\Abstract{ Many statistical analysis are based on models that consist
  of  more than one equation.  These equations should be fitted
  simultaneously rather than equation  by equation.  The package
  \pkg{systemfit} provides the capability to estimate systems  of
  linear and non-linear equations in \proglang{R}.
} 
  
\Keywords{R, simultaneous equation system, seemingly unrelated regression}

%% at least one keyword must be supplied

%% publication information
%% NOTE: This needs to filled out ONLY IF THE PAPER WAS ACCEPTED.
%% If it was not (yet) accepted, leave them commented.
%% \Volume{13}
%% \Issue{9}
%% \Month{September}
%% \Year{2004}
%% \Submitdate{2004-09-29}
%% \Acceptdate{2004-09-29}

%% The address of (at least) one author should be given
%% in the following format:
\Address{
   Arne Henningsen\\
   Department of Agricultural Economics\\
   University of Kiel\\
   D-24098 Kiel, Germany\\
   E-mail: \email{ahenningsen@agric-econ.uni-kiel.de}\\
   URL: \url{http://www.uni-kiel.de/agrarpol/ahenningsen/}\\
   \\
   Jeff D. Hamann\\
   Forest Informatics, Inc.\\
   PO Box 1421\\
   Corvallis, Oregon 97339-1421\\
   e-mail: \email{jeff.hamann@forestinformatics.com}\\
   URL: \url{http://www.forestinformatics.com}\\
}
%% It is also possible to add a telephone and fax number
%% before the e-mail in the following format:
%% Telephone: +43/1/31336-5053
%% Fax: +43/1/31336-734

%% for those who use Sweave please include the following line (with % symbols):
%% need no \usepackage{Sweave.sty}

%% end of declarations %%%%%%%%%%%%%%%%%%%%%%%%%%%%%%%%%%%%%%%%%%%%%%%


\begin{document}


%       $Id$    


%%%%%%%%%%%%%%%%%%%%%%%%%%%%%%%%%%%%%%%%%%%%%%%%%%
\section{Introduction}
%%%%%%%%%%%%%%%%%%%%%%%%%%%%%%%%%%%%%%%%%%%%%%%%%%

Many theoretical models consist of more than one equation. These
models can contain variables that appear on the left-hand side in one
equation and on the right-hand side of another equation
(simultaneity-bias). The system can have non-zero off-diagonal
covariance elements resulting from related disturbnaces
(contemporaneous correlation). Ignoring contemporaneous correlation
leads to inefficient parameter estimates \citep{zellner62} and
ignoring simultaneity-bias can lead to inconsistent parameter
estimates. 

% If these models are econometrically estimated, it may not be
% appropriate to estimate the equations individually, but all together,
% because the disturbance terms of these equations are likely to be
% contemporaneously correlated.  Estimating all equations
% simultaneously, taking the covariance structure of the residuals into
% account, leads to efficient estimates.  Ignoring this correlation
% leads to an inefficient parameter estimation \citep{zellner62}.

Another reason to estimate an equation system simultaneously are
cross-equation parameter restrictions. These restrictions can be
tested and/or imposed only in a simultaneous estimation approach
especially the economic theory suggests many cross-equation
restrictions. (But we don't say why this is important...) Are we
imposing restrictions in the parameters, or the variance-covariance
matrix or both and for what pupose?


For all of the methods developed in the package, the disturbnaces of
the individual equations are assumed to be independent and identially
distributed (iid).  
In the future, we would like to add the ability to fit equations were
the disturbances are serially correlated (wikins 1969).

The \pkg{systemfit} package provides the capability to estimate
linear and non-linear equation systems in \proglang{R}.
Although linear and non-linear equation systems can be estimated
with several other statistical and econometric software packages
(e.g.\ \proglang{SAS}, \proglang{EViews}, \proglang{TSP}),
\pkg{systemfit} has several advantages.
First, all estimation procedures are publicly available in the source code.
Second, the estimation algorithms can be easily modified to meet specific
requirements.
Third, the (advanced) user can control estimation details generally
not available in other software packages by overriding reasonable defaults.

This paper is organized as follows: In section~\ref{sec:Estimation} we
introduce the mathematics of estimating equation systems.
Section~\ref{sec:Restrictions} shows how linear restrictions can be
imposed.
Section~\ref{sec:Usage} demonstrates how to run
\pkg{systemfit}, especially how the features presented in the previous
sections can be used.
All other relevant issues are discussed in section~\ref{sec:Other}.
Finally, a summary and outlook are presented in
section~\ref{sec:Summmary}.


%%% Local Variables: 
%%% mode: latex
%%% TeX-master: "systemfit"
%%% End: 

%%%%%%%%%%%%%%%%%%%%%%%%%%%%%%%%%%%%%%%%%%%%%%%%%%
\subsection{Estimation}\label{sec:Estimation}
%%%%%%%%%%%%%%%%%%%%%%%%%%%%%%%%%%%%%%%%%%%%%%%%%%


%%%%%%%%%%%%%%%%%%%%%%%%%%%%%%%%%%%%%%%%%
\subsubsection{Ordinary least squares (OLS)}

The Ordinary Least Squares (OLS) estimator of the system 
is obtained by
\begin{equation}
   \widehat{\beta}_{OLS} = \left( X'X \right)^{-1} X'y
   \label{eq:ols}
\end{equation}
These estimates are efficient only if the disturbance terms are not
contemporaneously correlated, which means
$\sigma_{ij} = 0 \; \forall \; i \neq j$.
If the whole system is treated as one single equation,
the covariance matrix of the estimated parameters is
\begin{equation}
   \Cov \left[ \widehat{\beta}_{OLS} \right] = \sigma^2 \left( X'X \right)^{-1}
   \label{eq:olsCovSameSigma}
\end{equation}
with $\sigma^2 = E \left( u' u \right)$.
This assumes that the disturbances of all equations have the
same variance.

If the disturbance terms of the individual equations
are allowed to have different variances,
the covariance matrix of the estimated parameters is
\begin{equation}
   \Cov \left[ \widehat{\beta}_{OLS} \right] = \left( X' \Omega^{-1} X \right)^{-1}
   \label{eq:olsCovSingleSigma}
\end{equation}
with $\Omega = \Sigma \otimes I$,
$\sigma_{ij} = 0 \; \forall \; i \neq j$ and
$\sigma_{ii} = E \left( u_i' u_i \right)$.

If no cross-equation parameter restrictions are imposed, the simultaneous 
OLS estimation of the system leads to the same parameter estimates 
as an equation-wise OLS estimation.
The covariance matrix of the parameters from an equation-wise
OLS estimation is equal to the covariance matrix obtained
by equation (\ref{eq:olsCovSingleSigma}).


%%%%%%%%%%%%%%%%%%%%%%%%%%%%%%%%%%%%%%%%%%%%%%%
\subsubsection{Weighted least squares (WLS)}

The Weighted Least Squares (WLS) estimator of the system 
is obtained by
\begin{equation}
   \widehat{\beta}_{WLS} = \left( X' \Omega^{-1} X \right)^{-1} X' \Omega^{-1} y
\end{equation}
with $\Omega = \Sigma \otimes I$, 
$\sigma_{ij} = 0 \; \forall \; i \neq j$ and
$\sigma_{ii} = E \left( u_i' u_i \right)$.
Like the OLS estimates these estimates are only efficient
if the disturbance terms are not contemporaneously correlated.
The covariance matrix of the estimated parameters is
\begin{equation}
   \Cov \left[ \widehat{\beta}_{WLS} \right] = \left( X' \Omega^{-1} X \right)^{-1}
\end{equation}
If no cross-equation parameter restrictions are imposed,
the parameter estimates are equal to the OLS estimates.

%%%%%%%%%%%%%%%%%%%%%%%%%%%%%%%%%%%%%%%%%%%%%%%
\subsubsection{Seemingly unrelated regression (SUR)}

When the disturbances are contemporaneously correlated, a Generalized 
Least Squares (GLS) estimation leads to efficient parameter estimates.
In this case the GLS is generally called �Seemingly Unrelated Regression�
(SUR) (see \citealp{zellner62}). 
Note, while an unbiased OLS or WLS estimation requires only that 
the regressors and the disturbance terms of each single 
equation are uncorrelated
$( E \left[ u_i | X_i \right] = 0 \; \forall \; i )$,
a consistent SUR estimation requires that all disturbance terms and all 
regressors are uncorrelated
$( E \left[ u | X \right] = 0 )$.

The SUR estimator can be obtained by:%
\footnote{
To calculate $\Omega^{-1}$ it is not necessary to invert 
the (huge) $\Omega$ matrix, 
because it is sufficient to invert the (small) $\Sigma$ matrix and
calculate it by $\Omega^{-1} = \Sigma^{-1} \otimes I$
\citep[p.\ 342]{greene03}.
}
\begin{equation}
   \widehat{\beta}_{SUR} = \left( X' \Omega^{-1} X \right)^{-1} X' \Omega^{-1} y
\end{equation}
with $\Omega = \Sigma \otimes I$ and
$\sigma_{ij} = E \left( u_i' u_j \right)$.
And the covariance matrix of the estimated parameters is
\begin{equation}
   \Cov \left[ \widehat{\beta}_{SUR} \right] = \left( X' \Omega^{-1} X \right)^{-1}
\end{equation}


%%%%%%%%%%%%%%%%%%%%%%%%%%%%%%%%%%%%%%%%%%%%%%%
\subsubsection{Two-Stage least squares (2SLS)}

If the regressors of one or more equations are correlated 
with the disturbances ($E \left( u_i | X_i \right) \neq 0$), 
the estimated coefficients are biased.
This can be circumvented by an instrumental variable (IV)
two-stage least squares (2SLS) estimation.
The instrumental variables for each equation $H_i$ 
can be either different or identical for all equations.
The instrumental variables of each equation may not be correlated with 
the disturbance terms of the corresponding equation 
($E \left( u_i | H_i \right) = 0$).

At the first stage new ('fitted') regressors are obtained by
\begin{equation}
   \widehat{X_i} = H_i \left( H_i' H_i \right)^{-1} H_i' X
\end{equation}
At the second stage the unbiased two-stage least squares estimates
of $\beta$ are obtained by:
\begin{equation}
   \widehat{\beta}_{2SLS} = \left( \widehat{X}' \widehat{X} \right)^{-1} 
   \widehat{X}' y 
   \label{eq:beta2sls}
\end{equation}
If the whole system is treated as one single equation,
the covariance matrix of the estimated parameters is
\begin{equation}
   \Cov \left[ \widehat{\beta}_{2SLS} \right] = \sigma^2 \left( \widehat{X}'
   \widehat{X} \right)^{-1}
   \label{eq:2slsCovSameSigma}
\end{equation}
with $\sigma^2 = E \left( u' u \right)$.
If the disturbance terms of the individual equations
are allowed to have different variances, 
the covariance matrix of the estimated parameters is
\begin{equation}
   \Cov \left[ \widehat{\beta}_{2SLS} \right] = \left( \widehat{X}' \Omega^{-1}
   \widehat{X} \right)^{-1}
   \label{eq:2slsCovSingleSigma}
\end{equation}
with $\Omega = \Sigma \otimes I$, 
$\sigma_{ij} = 0 \; \forall \; i \neq j$ and
$\sigma_{ii} = E \left( u_i' u_i \right)$.


%%%%%%%%%%%%%%%%%%%%%%%%%%%%%%%%%%%%%%%%%%%%%%%
\subsubsection{Weighted two-stage least squares (W2SLS)}

The Weighted Two-Stage Least Squares (W2SLS) estimator of the system 
is obtained by
\begin{equation}
   \widehat{\beta}_{W2SLS} = \left( \widehat{X}' \Omega^{-1} \widehat{X} 
   \right)^{-1} \widehat{X}' \Omega^{-1} y
\end{equation}
with $\Omega = \Sigma \otimes I$, 
$\sigma_{ij} = 0 \; \forall \; i \neq j$ and
$\sigma_{ii} = E \left( u_i' u_i \right)$.
The covariance matrix of the estimated parameters is
\begin{equation}
   \Cov \left[ \widehat{\beta}_{W2SLS} \right] = \left( \widehat{X}' \Omega^{-1}
   \widehat{X} \right)^{-1}
\end{equation}


%%%%%%%%%%%%%%%%%%%%%%%%%%%%%%%%%%%%%%%%%%%%%%%
\subsubsection{Three-Stage least squares (3SLS)}

If the regressors are correlated with the disturbances 
($E \left( u | X \right) \neq 0$) and 
the disturbances are contemporaneously correlated, 
a Generalized Least Squares (GLS) version of the two-stage least squares
estimation leads to consistent and efficient estimates.
This estimation procedure is generally called �Three-stage Least
Squares� (3SLS).

The standard 3SLS estimator can be obtained by:
\begin{equation}
   \widehat{\beta}_{3SLS} = \left( \widehat{X}' \Omega^{-1} \widehat{X} 
   \right)^{-1} \widehat{X}' \Omega^{-1} y
   \label{eq:3slsGls}
\end{equation}
with $\Omega = \Sigma \otimes I$ and
$\sigma_{ij} = E \left( u_i' u_j \right)$.
Its covariance matrix is:
\begin{equation}
   \Cov \left[ \widehat{\beta}_{3SLS} \right] = \left( \widehat{X}' \Omega^{-1}
   \widehat{X} \right)^{-1}
   \label{eq:cov3sls}
\end{equation}
While an unbiased 2SLS or W2SLS estimation requires only that
the instrumental variables and the disturbance terms of each single 
equation are uncorrelated
$( E \left[ u_i | H_i \right]) = 0 \; \forall \; i )$,
\cite{schmidt90} points out that this estimator is only consistent 
if all disturbance terms and all instrumental variables are uncorrelated
$( E \left[ u | H \right]) = 0 )$
with
\begin{equation}
   H =
   \left[ \begin{array}{cccc}
      H_1 & 0 & \cdots & 0\\
      0 & H_2 & \cdots & 0\\
      \vdots & \vdots & \ddots & \vdots\\
      0 & 0 & \cdots & H_G
   \end{array}\right]
\end{equation}
Since there might be occasions where this cannot be avoided,
\cite{schmidt90} analyses other approaches to obtain 3SLS estimators:

One of these approaches is based on instrumental variable estimation
(3SLS-IV):
\begin{equation}
   \widehat{\beta}_{3SLS-IV} = \left( \widehat{X}' \Omega^{-1} X
   \right)^{-1} \widehat{X}' \Omega^{-1} y
   \label{eq:3slsIv}
\end{equation}
The covariance matrix of this 3SLS-IV estimator is:
\begin{equation}
   \Cov \left[ \widehat{\beta}_{3SLS-IV} \right] = \left( \widehat{X}' \Omega^{-1}
   X \right)^{-1}
\end{equation}
Another approach is based on the Generalized Method of Moments (GMM)
estimator (3SLS-GMM):
\begin{equation}
   \widehat{\beta}_{3SLS-GMM} = \left( X' H \left( H' \Omega H \right)^{-1}
   H' X \right)^{-1} X' H \left( H' \Omega H \right)^{-1} H' y
   \label{eq:3slsGmm}
\end{equation}
The covariance matrix of the 3SLS-GMM estimator is:
\begin{equation}
   \Cov \left[ \widehat{\beta}_{3SLS-GMM} \right] =
   \left( X' H \left( H' \Omega H \right)^{-1} H' X \right)^{-1}
\end{equation}
A fourth approach developed by \cite{schmidt90} himself is:
\begin{equation}
   \widehat{\beta}_{3SLS-Schmidt} = \left( \widehat{X}' \Omega^{-1} \widehat{X}
   \right)^{-1} \widehat{X}' \Omega^{-1} 
   H \left( H' H \right)^{-1} H' y 
   \label{eq:3slsSchmidt}
\end{equation}
The covariance matrix of this estimator is:
\begin{equation}
   \Cov \left[ \widehat{\beta}_{3SLS-Schmidt} \right] =
   \left( \widehat{X}' \Omega^{-1}  \widehat{X} \right)^{-1} 
   \widehat{X}' \Omega^{-1} H \left( H' H \right)^{-1} H' \Omega 
   H \left( H' H \right)^{-1} H' \Omega^{-1} \widehat{X}
   \left( \widehat{X}' \Omega^{-1}  \widehat{X} \right)^{-1}
\end{equation}
The econometrics software EViews uses following approach:
\begin{equation}
   \widehat{\beta}_{3SLS-EViews} = \widehat{\beta}_{2SLS} + 
   \left( \widehat{X}' \Omega^{-1} \widehat{X} \right)^{-1} 
   \widehat{X}' \Omega^{-1} \left( y - X \widehat{\beta}_{2SLS} \right)
   \label{eq:3slsEViews}
\end{equation}
where $\widehat{\beta}_{2SLS}$ is the two-stage least squares estimator
as defined by (\ref{eq:beta2sls}).
EViews uses the standard 3SLS formula (\ref{eq:cov3sls}) to
calculate the covariance matrix of the 3SLS estimator.


If the same instrumental variables are used in all equations 
($H_1 = H_2 = \ldots = H_G$), 
all the above mentioned approaches lead to identical parameter estimates.
However, if this is not the case, the results depend on the 
method used \citep{schmidt90}.
The only reason to use different instruments for different equations
is a correlation of the instruments of one equation with the
disturbance terms of another equation.
Otherwise, one could simply use all instruments in every equation
\citep{schmidt90}.
In this case, only the 3SLS-GMM (\ref{eq:3slsGmm})
and the 3SLS estimator developed by \cite{schmidt90} 
(\ref{eq:3slsSchmidt}) are consistent.



%%% Local Variables: 
%%% mode: latex
%%% TeX-master: "systemfit"
%%% End: 

%%%%%%%%%%%%%%%%%%%%%%%%%%%%%%%%%%%%%%%%%%%%%%%%%%%%%%
\subsection{Imposing linear restrictions}\label{sec:Restrictions}
%%%%%%%%%%%%%%%%%%%%%%%%%%%%%%%%%%%%%%%%%%%%%%%%%%%%%%

It is common to perform hypothesis tests by imposing restrictions on
the parameter estimates.
There are two ways to impose linear parameter restrictions.
First, a matrix $M$ can be specified that
\begin{equation}
   \beta = M \cdot \beta^* \label{eq:T-restr} ,
\end{equation}
where $\beta^*$ is a vector of restricted (linear independent) coefficients,
and $M$ is a matrix with the number of rows equal to the number of
unrestricted coefficients ($\beta$) and
the number of columns equal to the number of restricted coefficients
($\beta^*$).
$M$ can be used to map each unrestricted coefficient to one or more
restricted coefficients.

To impose these restrictions, the $X$ matrix is
(post-)\hspace{0pt}multiplied by this $M$ matrix, so that
\begin{equation}
    X^* = X \cdot M .
\end{equation}
Then, $X^*$ is substituted for $X$ and a standard estimation as described
in the previous section is done
(equations~\ref{eq:ols}--\ref{eq:3slsEViews}).
This results in the linear independent parameter estimates $\widehat{\beta}^*$ and
their covariance matrix.
The original parameters can be obtained by equation (\ref{eq:T-restr})
and the estimated covariance matrix of the original parameters
can be obtained by
\begin{equation}
   \COVHat \left[ \widehat{\beta} \right]
   = M \cdot \COVHat \left[ \widehat{\beta}^* \right] \cdot M' .
\end{equation}

The second way to impose linear parameter restrictions
can be formulated by
\begin{equation}
   R \beta^0 = q ,
\end{equation}
where $\beta^0$ is the vector of the restricted coefficients, 
and $R$ and $q$ are a matrix and vector, respectively,
to impose the restrictions \citep[see][p.\ 100]{greene03}.
Each linear independent restriction is represented by one row of $R$
and the corresponding element of~$q$.

The first way is less flexible than this latter one%
\footnote{
While restrictions like $\beta_1 = 2 \beta_2$ can be imposed by
both methods,
restrictions like $\beta_1 + \beta_2 = 4$ can be imposed only
by the second method.
}, 
but the first way is preferable if equality constraints for coefficients
across many equations of the system are imposed. 
Of course, these restrictions can be also imposed using
the latter method.
However, while the latter method increases the dimension of the 
matrices to be inverted during estimation, the first reduces it. 
Thus, in some cases the latter way leads to estimation problems
(e.g.\ (near) singularity of the matrices to be inverted),
while the first does not.

These two methods can be combined.
In this case the restrictions imposed using the latter method are
imposed on the linear independent parameters due to the restrictions
imposed using the first method, so that
\begin{equation}
   R \beta^{*0} = q ,
\end{equation}
where $\beta^{*0}$ is the vector of the restricted $\beta^*$ coefficients.

%%%%%%%%%%%%%%%%%%%%%%%%%%%%%%%%%%%%%%%%%%%%%%%%%%
\subsubsection{Restricted OLS estimation}

The OLS estimator restricted by $R \beta^0 = q$ can be obtained by
\begin{equation}
   \left[ \begin{array}{c}
      \widehat{\beta}^0_{OLS} \\ \widehat{\lambda}
   \end{array} \right]
   =
   \left[ \begin{array}{cc}
      X' X & R' \\ 
      R & 0
   \end{array} \right]^{-1}
   \cdot
   \left[ \begin{array}{c}
      X' y \\ q 
   \end{array} \right] ,
\end{equation}
where $\lambda$ is a vector of the Lagrangean multipliers of the restrictions.
If the whole system is treated as one single equation,
an estimator of the covariance matrix of the estimated parameters is
\begin{equation}
   \COVHat
   \left[ \begin{array}{c}
      \widehat{\beta}^0_{OLS} \\ \widehat{\lambda}
   \end{array} \right] 
   = \sHat^2 
   \left[ \begin{array}{cc}
      X' X & R' \\ 
      R & 0
   \end{array} \right]^{-1} ,
\end{equation}
where $\sHat$ is the estimated variance of the disturbance terms.
If the disturbance terms of the individual equations
are allowed to have different variances, 
an estimator of the covariance matrix of the estimated parameters is
\begin{equation}
   \COVHat
   \left[ \begin{array}{c}
      \widehat{\beta}^0_{OLS} \\ \widehat{\lambda}
   \end{array} \right] 
   = 
   \left[ \begin{array}{cc}
      X' \OHat^{-1} X & R' \\
      R & 0
   \end{array} \right]^{-1} ,
\end{equation}
where $\OHat = \SHat \otimes I$,
$\sHat_{ij} = 0 \; \forall \; i \neq j$ and
$\sHat_{ii} = \sHat_i^2$ is the estimated variance
of the disturbance term in the $i$th equation.

%%%%%%%%%%%%%%%%%%%%%%%%%%%%%%%%%%%%%%%%%%%%%%%%%%
\subsubsection{Restricted WLS estimation}

The WLS estimator restricted by $R \beta^0 = q$ can be obtained by
\begin{equation}
   \left[ \begin{array}{c}
      \widehat{\beta}^0_{WLS} \\ \widehat{\lambda}
   \end{array} \right]
   =
   \left[ \begin{array}{cc}
      X' \OHat^{-1} X & R' \\
      R & 0
   \end{array} \right]^{-1}
   \cdot
   \left[ \begin{array}{c}
      X' \OHat^{-1} y \\ q
   \end{array} \right] ,
\end{equation}
where $\OHat = \SHat \otimes I$,
$\sHat_{ij} = 0 \; \forall \; i \neq j$ and
$\sHat_{ii} = \sHat_i^2$ is the estimated variance
of the disturbance term in the $i$th equation.
An estimator of the covariance matrix of the estimated parameters is
\begin{equation}
   \COVHat
   \left[ \begin{array}{c}
      \widehat{\beta}^0_{WLS} \\ \widehat{\lambda}
   \end{array} \right] 
   = 
   \left[ \begin{array}{cc}
      X' \OHat^{-1} X & R' \\
      R & 0
   \end{array} \right]^{-1} .
\end{equation}

%%%%%%%%%%%%%%%%%%%%%%%%%%%%%%%%%%%%%%%%%%%%%%%%%%
\subsubsection{Restricted SUR estimation}

The SUR estimator restricted by $R \beta^0 = q$ can be obtained by
\begin{equation}
   \left[ \begin{array}{c}
      \widehat{\beta}^0_{SUR} \\ \widehat{\lambda}
   \end{array} \right]
   =
   \left[ \begin{array}{cc}
      X' \OHat^{-1} X & R' \\
      R & 0
   \end{array} \right]^{-1}
   \cdot
   \left[ \begin{array}{c}
      X' \OHat^{-1} y \\ q
   \end{array} \right] ,
\end{equation}
where $\OHat = \SHat \otimes I$ and
$\SHat$ is the estimated covariance matrix of the disturbance terms.
An estimator of the covariance matrix of the estimated parameters is
\begin{equation}
   \COVHat
   \left[ \begin{array}{c}
      \widehat{\beta}^0_{SUR} \\ \widehat{\lambda}
   \end{array} \right] 
   = 
   \left[ \begin{array}{cc}
      X' \OHat^{-1} X & R' \\
      R & 0
   \end{array} \right]^{-1} .
\end{equation}

%%%%%%%%%%%%%%%%%%%%%%%%%%%%%%%%%%%%%%%%%%%%%%%%%%
\subsubsection{Restricted 2SLS estimation}

The 2SLS estimator restricted by $R \beta^0 = q$ can be obtained by
\begin{equation}
   \left[ \begin{array}{c}
      \widehat{\beta}^0_{2SLS} \\ \widehat{\lambda}
   \end{array} \right]
   =
   \left[ \begin{array}{cc}
      \widehat{X}' \widehat{X} & R' \\ 
      R & 0
   \end{array} \right]^{-1}
   \cdot
   \left[ \begin{array}{c}
      \widehat{X}' y \\ q 
   \end{array} \right] .
   \label{eq:beta2SLSr}
\end{equation}
If the whole system is treated as one single equation,
an estimator of the covariance matrix of the estimated parameters is
\begin{equation}
   \COVHat
   \left[ \begin{array}{c}
      \widehat{\beta}^0_{2SLS} \\ \widehat{\lambda}
   \end{array} \right] 
   = \sHat^2 
   \left[ \begin{array}{cc}
      \widehat{X}' \widehat{X} & R' \\ 
      R & 0
   \end{array} \right]^{-1} ,
\end{equation}
where $\sHat^2$ is the estimated variance of the disturbance terms.
If the disturbance terms of the individual equations
are allowed to have different variances, 
an estimator of the covariance matrix of the estimated parameters is
\begin{equation}
   \COVHat
   \left[ \begin{array}{c}
      \widehat{\beta}^0_{2SLS} \\ \widehat{\lambda}
   \end{array} \right] 
   = 
   \left[ \begin{array}{cc}
      \widehat{X}' \OHat^{-1} \widehat{X} & R' \\
      R & 0
   \end{array} \right]^{-1} ,
\end{equation}
where $\OHat = \SHat \otimes I$,
$\sHat_{ij} = 0 \; \forall \; i \neq j$ and
$\sHat_{ii} = \sHat_i^2$ is the estimated variance
of the disturbance term in the $i$th equation.


%%%%%%%%%%%%%%%%%%%%%%%%%%%%%%%%%%%%%%%%%%%%%%%%%%
\subsubsection{Restricted W2SLS estimation}

The W2SLS estimator restricted by $R \beta^0 = q$ can be obtained by
\begin{equation}
   \left[ \begin{array}{c}
      \widehat{\beta}^0_{W2SLS} \\ \widehat{\lambda}
   \end{array} \right]
   =
   \left[ \begin{array}{cc}
      \widehat{X}' \OHat^{-1} \widehat{X} & R' \\
      R & 0
   \end{array} \right]^{-1}
   \cdot
   \left[ \begin{array}{c}
      \widehat{X}' \OHat^{-1} y \\ q
   \end{array} \right] ,
\end{equation}
where $\OHat = \SHat \otimes I$,
$\sHat_{ij} = 0 \; \forall \; i \neq j$ and
$\sHat_{ii} = \sHat_i^2$ is the estimated variance
of the disturbance term in the $i$th equation.
An estimator of the covariance matrix of the estimated parameters is
\begin{equation}
   \COVHat
   \left[ \begin{array}{c}
      \widehat{\beta}^0_{W2SLS} \\ \widehat{\lambda}
   \end{array} \right] 
   = 
   \left[ \begin{array}{cc}
      \widehat{X}' \OHat^{-1} \widehat{X} & R' \\
      R & 0
   \end{array} \right]^{-1} .
\end{equation}


%%%%%%%%%%%%%%%%%%%%%%%%%%%%%%%%%%%%%%%%%%%%%%%%%%
\subsubsection{Restricted 3SLS estimation}

The standard 3SLS estimator restricted by $R \beta^0 = q$ can be obtained by
\begin{equation}
   \left[ \begin{array}{c}
      \widehat{\beta}^0_{3SLS} \\ \widehat{\lambda}
   \end{array} \right]
   =
   \left[ \begin{array}{cc}
      \widehat{X}' \OHat^{-1} \widehat{X} & R' \\
      R & 0
   \end{array} \right]^{-1}
   \cdot
   \left[ \begin{array}{c}
      \widehat{X}' \OHat^{-1} y \\ q
   \end{array} \right] ,
   \label{eq:3slsGlsR}
\end{equation}
where $\OHat = \SHat \otimes I$ and
$\SHat$ is the estimated covariance matrix of the disturbance terms.
An estimator of the covariance matrix of the estimated parameters is
\begin{equation}
   \COVHat
   \left[ \begin{array}{c}
      \widehat{\beta}^0_{3SLS} \\ \widehat{\lambda}
   \end{array} \right] 
   = 
   \left[ \begin{array}{cc}
      \widehat{X}' \OHat^{-1} \widehat{X} & R' \\
      R & 0
   \end{array} \right]^{-1} .
   \label{eq:cov3slsr}
\end{equation}
The 3SLS-IV estimator restricted by $R \beta^0 = q$ can be obtained by
\begin{equation}
   \left[ \begin{array}{c}
      \widehat{\beta}^0_{3SLS-IV} \\ \widehat{\lambda}
   \end{array} \right]
   =
   \left[ \begin{array}{cc}
      \widehat{X}' \OHat^{-1} X & R' \\
      R & 0
   \end{array} \right]^{-1}
   \cdot
   \left[ \begin{array}{c}
      \widehat{X}' \OHat^{-1} y \\ q
   \end{array} \right] ,
   \label{eq:3slsIvR}
\end{equation}
where
\begin{equation}
   \COVHat
   \left[ \begin{array}{c}
      \widehat{\beta}^0_{3SLS-IV} \\ \widehat{\lambda}
   \end{array} \right] 
   = 
   \left[ \begin{array}{cc}
      \widehat{X}' \OHat^{-1} \widehat{X} & R' \\
      R & 0
   \end{array} \right]^{-1} .
\end{equation}
The restricted 3SLS-GMM estimator can be obtained by
\begin{equation}
   \left[ \begin{array}{c}
      \widehat{\beta}^0_{3SLS-GMM} \\ \widehat{\lambda}
   \end{array} \right]
   =
   \left[ \begin{array}{cc}
      X' H \left( H' \OHat H \right)^{-1} H' X & R' \\
      R & 0
   \end{array} \right]^{-1}
   \cdot
   \left[ \begin{array}{c}
      X' H \left( H \OHat H \right)^{-1} H' y \\ q
   \end{array} \right] ,
   \label{eq:3slsGmmR}
\end{equation}
where
\begin{equation}
   \COVHat
   \left[ \begin{array}{c}
      \widehat{\beta}^0_{3SLS-GMM} \\ \widehat{\lambda}
   \end{array} \right] 
   = 
   \left[ \begin{array}{cc}
      X' H \left( H' \OHat H \right)^{-1} H' X & R' \\
      R & 0
   \end{array} \right]^{-1} .
\end{equation}
The restricted 3SLS estimator based on the suggestion of
\cite{schmidt90} is
\begin{equation}
   \left[ \begin{array}{c}
      \widehat{\beta}^0_{3SLS-Schmidt} \\ \widehat{\lambda}
   \end{array} \right]
   =
   \left[ \begin{array}{cc}
      \widehat{X}' \OHat^{-1} \widehat{X} & R' \\
      R & 0
   \end{array} \right]^{-1}
   \cdot
   \left[ \begin{array}{c}
      \widehat{X}' \OHat^{-1} H \left( H' H \right)^{-1} H' y \\ q
   \end{array} \right] ,
   \label{eq:3slsSchmidtR}
\end{equation}
where
\begin{eqnarray}
   \COVHat
   \left[ \begin{array}{c}
      \widehat{\beta}^0_{3SLS-Schmidt} \\ \widehat{\lambda}
   \end{array} \right] 
   & = & 
   \left[ \begin{array}{cc}
      \widehat{X}' \OHat^{-1} \widehat{X} & R' \\
      R & 0
   \end{array} \right]^{-1}
   \\
   & & \cdot
   \left[ \begin{array}{cc}
      \widehat{X}' \OHat^{-1} H \left( H' H \right)^{-1} H' \OHat
      H \left( H' H \right)^{-1} H' \OHat^{-1} \widehat{X} & 0' \\
      0 & 0
   \end{array} \right]^{-1}
   \nonumber \\
   & & \cdot
   \left[ \begin{array}{cc}
      \widehat{X}' \OHat^{-1} \widehat{X} & R' \\
      R & 0
   \end{array} \right]^{-1} .
   \nonumber
\end{eqnarray}
The econometrics software \proglang{EViews} calculates the restricted 3SLS estimator by
\begin{equation}
   \left[ \begin{array}{c}
      \widehat{\beta}^0_{3SLS-EViews} \\ \widehat{\lambda}
   \end{array} \right]
   =
   \left[ \begin{array}{cc}
      \widehat{X}' \OHat^{-1} \widehat{X} & R' \\
      R & 0
   \end{array} \right]^{-1}
   \cdot
   \left[ \begin{array}{c}
      \widehat{X}' \OHat^{-1} \left( y - X \widehat{\beta}^0_{2SLS} \right)
      \\ q 
   \end{array} \right] ,
   \label{eq:3slsEViewsR}
\end{equation}
where $\widehat{\beta}^0_{2SLS}$ is the restricted 2SLS estimator calculated
by equation (\ref{eq:beta2SLSr}). 
\proglang{EViews} uses the standard formula of the restricted 3SLS
estimator~(\ref{eq:cov3slsr}) to calculate an estimator
for the covariance matrix of the estimated parameters.


If the same instrumental variables are used in all equations 
($H_1 = H_2 = \ldots = H_G$), 
all the above mentioned approaches lead to identical parameter estimates
and identical covariance matrices of the estimated parameters.

%%% Local Variables: 
%%% mode: latex
%%% TeX-master: "systemfit"
%%% End: 


%       $Id$    

%%%%%%%%%%%%%%%%%%%%%%%%%%%%%%%%%%%%%%%%%%%%%%%%%%
%\section{Other issues and tools}\label{sec:Other}
%%%%%%%%%%%%%%%%%%%%%%%%%%%%%%%%%%%%%%%%%%%%%%%%%%

%%%%%%%%%%%%%%%%%%%%%%%%%%%%%%%%%%%%%%%%%%%%%%%%%%
\subsection{Residual covariance matrix}\label{sec:residcov}

Since the true residuals of the estimated equations are generally not known,
the true covariance matrix of the residuals cannot be determined.
Thus, this covariance matrix must be calculated from the
\emph{estimated} residuals. 
Generally, the estimated covariance matrix of the residuals
($\widehat{\Sigma} = \left[ \widehat{\sigma}_{ij} \right]$)
can be calculated from the residuals of a first-step OLS or 2SLS estimation.
The following formula is often applied:
\begin{equation}
   \widehat{\sigma}_{ij} = \frac{ \widehat{u}_i' \widehat{u}_j }{ T }
   \label{eq:rcov0}
\end{equation}
where $T$ is the number of observations in each equation.
However, in finite samples this estimator is biased,
because it is not corrected for degrees of freedom.
The usual single-equation procedure to correct for degrees of freedom
cannot always be applied, because the number of regressors in each equation
might differ.
Two alternative approaches to calculate the residual covariance
matrix are
\begin{equation}
   \widehat{\sigma}_{ij} = \frac{ \widehat{u}_i' \widehat{u}_j }
   { \sqrt{ \left( T - K_i \right) \cdot \left( T - K_j \right) } }
   \label{eq:rcov1}
\end{equation}
and
\begin{equation}
   \widehat{\sigma}_{ij} = \frac{ \widehat{u}_i' \widehat{u}_j }
   { T - \max \left( K_i , K_j \right) }
   \label{eq:rcov3}
\end{equation}
where $K_i$ and $K_j$ are the number of regressors in equation
$i$ and $j$, respectively.
However, these formulas yield unbiased estimators only if $K_i = K_j$
\citep[p.\ 469]{judge85}. 
% Greene (2003, p. 344) says that the second is unbiased if i = j or K_i = K_j,
% whereas the first is unbiased only if i = j. 
% However, if K_i = K_j the first and the second are equal.
% Why is the first biased if K_i = K_j ???????????


A further approach to obtain the estimated residual covariance
matrix is \citep[p.\ 309]{zellner62c}
\begin{eqnarray}
   \widehat{\sigma}_{ij} & = & 
   \frac{ \widehat{u}_i' \widehat{u}_j } 
   { T - K_i - K_j + tr \left[ X_i \left( X_i' X_i \right)^{-1}
   X_i' X_j \left( X_j' X_j \right)^{-1} X_j' \right] }
   \label{eq:rcov2} \\
   & = &
   \frac{ \widehat{u}_i' \widehat{u}_j } 
   { T - K_i - K_j + tr \left[ \left( X_i' X_i \right)^{-1}
   X_i' X_j \left( X_j' X_j \right)^{-1} X_j' X_i \right] }
\end{eqnarray} 
This yields an unbiased estimator for all elements of $\Sigma$,
but even if $\widehat{\Sigma}$ is an unbiased estimator of $\Sigma$, 
its inverse $\widehat{\Sigma}^{-1}$ is not an unbiased estimator 
of $\Sigma^{-1}$ \citep[p.\ 322]{theil71}.
Furthermore, the covariance matrix calculated by (\ref{eq:rcov2})
is not necessarily positive semidefinite \citep[p.\ 322]{theil71}. 
Hence, �it is doubtful whether [this formula] is really superior to 
[(\ref{eq:rcov0})]� \citep[p.\ 322]{theil71}.


The WLS, SUR, W2SLS and 3SLS parameter estimates are consistent
if the estimated residual covariance matrix is calculated
using the residuals from a first-step OLS or 2SLS estimation.
There exists also an alternative slightly different approach.%
\footnote{
For instance, this approach is applied by
the command �TSCS� of the software LIMDEP that carries out SUR estimations
in which all coefficient vectors are constrained to be equal
\citep{greene06}.
}
This alternative approach uses the residuals of a first-step OLS or 2SLS estimation
to apply a WLS or W2SLS estimation on a second step.
Then, it calculates the residual covariance matrix
from the residuals of the second-step estimation
to estimates the model by SUR or 3SLS in a third step.
If no cross-equation restrictions are imposed,
the parameter estimates of OLS and WLS as well as 2SLS and W2SLS are identical.
Hence, in this case both approaches generate the same results.

It is also possible to iterate WLS, SUR, W2SLS and 3SLS estimations.
At each iteration the residual covariance matrix is calculated
from the residuals of the previous iteration.
If equation (\ref{eq:rcov0}) is applied to calculate the estimated
residual covariance matrix,
an iterated SUR estimation converges to maximum
likelihood \citep[p.\ 345]{greene03}.

In some uncommon cases,
for instance in pooled estimations,
where the coefficients are restricted to be equal in all equations,
the means of the residuals of each equation are not equal to zero
$( \overline{ \widehat{u} }_i \neq 0 )$.
Therefore, it might be argued
that the residual covariance matrix should be calculated
by subtracting the means from the residuals
and substituting $\widehat{u}_i - \overline{ \widehat{u} }_i$
for $\widehat{u}_i$ in (\ref{eq:rcov0}--\ref{eq:rcov2}).


%%%%%%%%%%%%%%%%%%%%%%%%%%%%%%%%%%%%%%%%%%%%%%%%%%
\subsection{Degrees of freedom}
\label{sec:degreesOfFreedom}

To our knowledge the question about how to determine the degrees
of freedom for single-parameter $t$ tests is not comprehensively
discussed in the literature.
While sometimes the degrees of freedom of the entire system
(total number of observations in all equations minus
total number of estimated parameters)
are applied,
in other cases the degrees of freedom of each single equation
(number of observations in the equations minus
number of estimated parameters in the equation)
are used.
Asymptotically, this distinction does not make a difference.
However, in many empirical applications, the number of observations
of each equation is rather small, and
therefore, it matters.

If a system of equations is estimated by an unrestricted OLS and
the covariance matrix of the parameters is calculated
by~(\ref{eq:olsCovSingleSigma}),
the estimated parameters and their standard errors are identical
to an equation-wise OLS estimation.
In this case, it is reasonable to use the degrees of freedom of
each single equation,
because this yields the same $p$ values as the equation-wise
OLS estimation.

In contrast, if a system of equations is estimated with many
cross-equation restrictions and
the covariance matrix of an OLS estimation is calculated
by~(\ref{eq:olsCovSameSigma}),
the system estimation is similar to a single equation estimation.
Therefore, in this case, it seems to be reasonable to use the degrees
of freedom of the entire system.



%%%%%%%%%%%%%%%%%%%%%%%%%%%%%%%%%%%%%%%%%%%%%%%%%%
\subsection{Goodness of fit}

The goodness of fit of each single equation can be measured by the
traditional $R^2$ values:
\begin{equation}
   R_i^2 = 1 - \frac{ \widehat{u}_i' \widehat{u}_i }
   { ( y_i - \overline{y_i} )' ( y_i - \overline{y_i} ) }
\end{equation}
where $R_i^2$ is the $R^2$ value of the $i$th equation
and $\overline{y_i}$ is the mean value of $y_i$.

The goodness of fit of the whole system can be measured by the
McElroy's $R^2$ value \citep{mcelroy77}: 
% also: \citep[p.\ 345]{greene03}
\begin{equation}
   R_*^2 = 1 - \frac{ \widehat{u}' \widehat{ \Omega }^{-1} \widehat{u} }
   { y' \left( \widehat{ \Sigma }^{-1} \otimes
   \left( I - \frac{i i'}{T} \right) \right) y }
\end{equation}
where $T$ is the number of observations in each equation,
$I$ is an $T \times T$ identity matrix and 
$i$ is a column vector of $T$ ones.


%%%%%%%%%%%%%%%%%%%%%%%%%%%%%%%%%%%%%%%%%%%%%%%%%%
\subsection{Testing linear restrictions}
\label{sec:testingRestrictions}

Linear restrictions can be tested by an $F$ test, Wald test or
likelihood ratio (LR) test.

The $F$ statistic for systems of equations is
\begin{equation}
F = \frac{
   ( R \widehat{\beta} - q )'
   ( R ( X' ( \widehat{\Sigma} \otimes I )^{-1} X )^{-1} R' )^{-1}
   ( R \widehat{\beta} - q ) /
   j
}{
   \widehat{u}' ( \widehat{\Sigma} \otimes I )^{-1} \widehat{u} /
   ( M \cdot T - K )
}
\end{equation}
where $j$ is the number of restrictions,
$M$ is the number of equations,
$T$ is the number of observations per equation,
$K$ is the total number of estimated coefficients, and
$\widehat{\Sigma}$ is the estimated residual covariance matrix
used in the estimation.
Under the null hypothesis, $F$ has an $F$ distribution
with $j$ and $M \cdot T - K$ degrees of freedom
\citep[p.\ 314]{theil71}.

The Wald statistic for systems of equations is
\begin{equation}
W =
   ( R \widehat{\beta} - q )'
   ( R \widehat{Cov} [ \widehat{\beta} ] R' )^{-1}
   ( R \widehat{\beta} - q )
\end{equation}
Asymptotically, $W$ has a $\chi^2$
distribution with $j$ degrees of freedom
under the null hypothesis
\citep[p.\ 347]{greene03}.

The LR statistic for systems of equations is
\begin{equation}
LR = T \cdot \left(
   log \left| \widehat{ \Sigma }_r \right|
   - log \left| \widehat{ \Sigma }_u \right|
   \right)
\end{equation}
where $T$ is the number of observations per equation, and 
$\widehat{\Sigma}_r$ and $\widehat{\Sigma}_u$ are
the residual covariance matrices calculated by formula (\ref{eq:rcov0})
of the restricted and unrestricted estimation, respectively.
Asymptotically, $LR$ has a $\chi^2$
distribution with $j$ degrees of freedom
under the null hypothesis
\citep[p.\ 349]{greene03}.



%%%%%%%%%%%%%%%%%%%%%%%%%%%%%%%%%%%%%%%%%%%%%%%%%%
\subsection{Hausman test}
\label{sec:hausman}

\citet{hausman78} developed a test for misspecification.
The null hypothesis of the test is that all exogenous variables are
uncorrelated with all disturbance terms.
Under this hypothesis both the 2SLS and the 3SLS estimator are consistent
but only the 3SLS estimator is (asymptotically) efficient.
Under the alternative hypothesis the 2SLS estimator is consistent
but the 3SLS estimator is inconsistent.
The Hausman test statistic is,
\begin{equation}
  m = \left( \widehat{\beta}_{2SLS} - \widehat{\beta}_{3SLS} \right)^{'}
      \left( \Cov \left[ \widehat{\beta}_{2SLS} \right] -
             \Cov \left[ \widehat{\beta}_{3SLS} \right] \right)
      \left( \widehat{\beta}_{2SLS} - \widehat{\beta}_{3SLS} \right)
\label{eq:hausman}
\end{equation}
where $\widehat{\beta}_{2SLS}$ and $\Cov \left[ \widehat{\beta}_{2SLS} \right]$ are the estimated
coefficient and covariance matrix from 2SLS estimation, and
$\widehat{\beta}_{3SLS}$ and $\Cov \left[ \widehat{\beta}_{3SLS} \right]$ are the estimated
coefficients and covariance matrix from 3SLS estimation.
Under the null hypothesis this test statistic has a
$\chi^2$ distribution with degrees of freedom equal to the number of
estimated parameters.




%%% Local Variables: 
%%% mode: latex
%%% TeX-master: "systemfit"
%%% End: 

%%%%%%%%%%%%%%%%%%%%%%%%%%%%%%%%%%%%%%%%%%%%%%%%%
\section{Using systemfit}\label{sec:Usage}
%%%%%%%%%%%%%%%%%%%%%%%%%%%%%%%%%%%%%%%%%%%%%%%%%%


%%%%%%%%%%%%%%%%%%%%%%%%%%%%%%%%%%%%%%%%%%%%%%%%%%
\subsection{Standard usage}

\pkg{systemfit} is generally called by

\code{
R> systemfit( method, eqns )
}

There are two mandatory arguments: \code{method} and \code{eqns}.

The argument \code{method} is a string determining the estimation method.
It must be one of "OLS", "WLS", "SUR", "2SLS", "W2SLS" or "3SLS".

The other mandatory argument \code{eqns} is a list of the equations 
to estimate. 
Each equation is a standard formula in \proglang{R}.
It starts with a dependent variable on the left hand side.
After a tilde ($\sim$) the regressors are listed%
\footnote{For Details see the \proglang{R} help files to \code{formula}}.

This is now demonstrated using an example: \\
\code{
R> library( systemfit ) \\
R> data( kmenta ) \\
R> attach( kmenta ) \\
R> fitsur <- systemfit( "SUR", list( q $\sim$ p + d, q $\sim$ p + f + a ) ) \\
}

The first line loads the \pkg{systemfit} package. 
The second line loads example data that are included in this package.
These data come from \cite{kmenta86}.
They are attached to the \proglang{R} search path in line three.
In the last line a seemingly unrelated regression is done.
The first equation represents the demand side of the food market.
The dependant variable is \code{q} (food consumption per capita). 
The regressors are \code{p} (ratio of food prices to general consumer prices)
and \code{d} (disposable income) as well as a constant%
\footnote{a regression constant is always implied if not explicitly omitted.}.
The second equation represents the supply side.
Variable \code{q} (food consumption per capita) is also the dependant 
variable of this equation. 
The regressors are again \code{p} (ratio of food prices to general 
consumer prices) and a constant as well as 
\code{f} (ratio of preceding year's prices received by farmers) and 
\code{a} (a time trend in years).
The regression result is assigned to the variable \code{fitsur}.

Summary results can be printed by\\
\code{
R>~summary(~fitsur~)~\\
~\\
systemfit~results~\\
method:~SUR~\\
\\
\mbox{}~~~N~DF~~~~~~SSR~~~~~MSE~~~~RMSE~~~~~~~R2~~~Adj~R2 \\
1~20~17~~65.6829~3.86370~1.96563~0.755019~0.726198~\\
2~20~16~104.0584~6.50365~2.55023~0.611888~0.539117~\\
~\\
The~covariance~matrix~of~the~residuals~used~for~estimation\\
\mbox{}~~~~~~~~1~~~~~~~2~\\
1~3.72539~4.13696~\\
2~4.13696~5.78444~\\
~\\
The~covariance~matrix~of~the~residuals\\
\mbox{}~~~~~~~~1~~~~~~~2~\\
1~3.86370~4.92431~\\
2~4.92431~6.50365~\\
~\\
The~correlations~of~the~residuals\\
\mbox{}~~~~~~~~~1~~~~~~~~2~\\
1~1.000000~0.982348~\\
2~0.982348~1.000000~\\
~\\
The~determinant~of~the~residual~covariance~matrix:~0.879285~\\
OLS~R-squared~value~of~the~system:~0.683453~\\
McElroy's~R-squared~value~for~the~system:~0.788722~\\
~\\
SUR~estimates~for~1~~(equation~1~)~\\
Model~Formula:~q~~~p~+~d\\
\\
\mbox{}~~~~~~~~~~~~~Estimate~Std.~Error~~~t~value~Pr(>|t|)~\\
(Intercept)~99.332894~~~7.514452~13.218913~~~~~~~~0~***~\\
p~~~~~~~~~~~-0.275486~~~0.088509~-3.112513~0.006332~~**~\\
d~~~~~~~~~~~~~0.29855~~~0.041945~~7.117605~~~~2e-06~***~\\
---~\\
Signif.~codes:~~0~`***'~0.001~`**'~0.01~`*'~0.05~`.'~0.1~`~'~1~\\
~\\
Residual~standard~error:~1.96563~on~17~degrees~of~freedom~\\
Number~of~observations:~20~Degrees~of~Freedom:~17~\\
SSR:~65.682902~MSE:~3.8637~Root~MSE:~1.96563~\\
Multiple~R-Squared:~0.755019~Adjusted~R-Squared:~0.726198~\\
~\\
~\\
SUR~estimates~for~2~~(equation~2~)~\\
Model~Formula:~q~~~p~+~f~+~a\\
\\
\mbox{}~~~~~~~~~~~~~Estimate~Std.~Error~~t~value~Pr(>|t|)~\\
(Intercept)~61.966166~~~11.08079~5.592215~~~~4e-05~***~\\
p~~~~~~~~~~~~0.146884~~~0.094435~1.555397~0.139408~\\
f~~~~~~~~~~~~0.214004~~~0.039868~5.367761~~6.3e-05~***~\\
a~~~~~~~~~~~~0.339304~~~0.067911~4.996283~0.000132~***~\\
---~\\
Signif.~codes:~~0~`***'~0.001~`**'~0.01~`*'~0.05~`.'~0.1~`~'~1~\\
~\\
Residual~standard~error:~2.550226~on~16~degrees~of~freedom~\\
Number~of~observations:~20~Degrees~of~Freedom:~16~\\
SSR:~104.05843~MSE:~6.503652~Root~MSE:~2.550226~\\
Multiple~R-Squared:~0.611888~Adjusted~R-Squared:~0.539117~\\
}



%%%%%%%%%%%%%%%%%%%%%%%%%%%%%%%%%%%%%%%%%%%%%%%%%%
\subsection{User options}

Following additional options can be set by the user:

%%%%%%%%%%%%%%%%%%%%%%%%%%%%%%%%%%%%%%%%%%%%%%%%%%
% \subsubsection{Equation labels}
\paragraph{Equation labels}
The optional argument \code{eqnlabels} allows the user to label the equations.
It has to be a vector of strings naming the equations.\\
\code{
R>~fitsur~<-~systemfit(~"SUR",~list(~q~~~p~+~d,~q~~~p~+~f~+~a~),\\~
R+~~~~eqnlabels~=~c(~"demand",~"supply"~)~)\\
R>~summary(~fitsur~)\\
systemfit~results\\
method:~SUR\\
\\
\mbox{}~~~~~~~~N~DF~~~~~~SSR~~~~~MSE~~~~RMSE~~~~~~~R2~~~Adj~R2\\
demand~20~17~~65.6829~3.86370~1.96563~0.755019~0.726198\\
supply~20~16~104.0584~6.50365~2.55023~0.611888~0.539117\\
\ldots\\
}
If no equation labels are provided, the equations are numbered.
 
%%%%%%%%%%%%%%%%%%%%%%%%%%%%%%%%%%%%%%%%%%%%%%%%%%
%\subsubsection{Instrumental variables}   
\paragraph{Instrumental variables}   
\code{inst}one-sided model formula specifying instrumental variables
   or a list of one-sided model formulas if different instruments should
   be used for the different equations (only needed for 2SLS, W2SLS and
   3SLS estimations).

%%%%%%%%%%%%%%%%%%%%%%%%%%%%%%%%%%%%%%%%%%%%%%%%%%
%\subsubsection{Data}   
\paragraph{Data}   
\code{data} an optional data frame containing the variables in the model.
   By default the variables are taken from the environment from which
   systemfit is called.

%%%%%%%%%%%%%%%%%%%%%%%%%%%%%%%%%%%%%%%%%%%%%%%%%%
%\subsubsection{Restrictions}   
\paragraph{Restrictions}   
\code{R.restr} an optional j x k matrix to impose linear
   restrictions on the parameters by \code{R.restr} * $\beta$ = \code{q.restr}
   (j = number of restrictions, k = number of all parameters,
   $\beta$ = vector of all parameters).

\code{q.restr} an optional j x 1 matrix to impose linear
   restrictions (see \code{R.restr}); default is a j x 1 matrix
   that contains only zeros.

\code{TX} an optional matrix to transform the regressor matrix and,
   hence, also the coefficient vector (see details).

%%%%%%%%%%%%%%%%%%%%%%%%%%%%%%%%%%%%%%%%%%%%%%%%%%
%\subsubsection{Iteration control}   
\paragraph{Iteration control}
\code{maxiter} maximum number of iterations for WLS, SUR, W2SLS and
   3SLS estimations.

\code{tol} tolerance level indicating when to stop the iteration (only
   WLS, SUR, W2SLS and 3SLS estimations).

%%%%%%%%%%%%%%%%%%%%%%%%%%%%%%%%%%%%%%%%%%%%%%%%%%
%\subsubsection{Residual covariance matrix}   
\paragraph{Residual covariance matrix}   
\code{rcovformula} formula to calculate the estimated residual covariance
   matrix (see details).

%%%%%%%%%%%%%%%%%%%%%%%%%%%%%%%%%%%%%%%%%%%%%%%%%%
%\subsubsection{3SLS formula}   
\paragraph{3SLS formula}   
\code{formula3sls} formula for calculating the 3SLS estimator,
   one of "GLS", "IV", "GMM", "Schmidt" or "EViews" (see details).

%%%%%%%%%%%%%%%%%%%%%%%%%%%%%%%%%%%%%%%%%%%%%%%%%%
%\subsubsection{Degrees of freedom for t-tests}   
\paragraph{Degrees of freedom for t-tests}   
\code{probdfsys} use the degrees of freedom of the whole system
   (in place of the degrees of freedom of the single equation)
   to calculate prob values for the t-test of individual parameters.

%%%%%%%%%%%%%%%%%%%%%%%%%%%%%%%%%%%%%%%%%%%%%%%%%%
%\subsubsection{Sigma squared}   
\paragraph{Sigma squared}   
\code{single.eq.sigma} use different $\sigma^2$s for each
   single equation to calculate the covariance matrix and the
   standard errors of the coefficients (only OLS and 2SLS).

%%%%%%%%%%%%%%%%%%%%%%%%%%%%%%%%%%%%%%%%%%%%%%%%%%
%\subsubsection{System options}   
\paragraph{System options}   

\code{solvetol} tolerance level for detecting linear dependencies
   when inverting a matrix or calculating a determinant (see
   \code{solve} and \code{det}).

\code{saveMemory} save memory by omitting some calculation that
   are not crucial for the basic estimation (e.g McElroy's
   $R^2$).



%%%%%%%%%%%%%%%%%%%%%%%%%%%%%%%%%%%%%%%%%%%%%%%%%%
\section{Summary and outlook}\label{sec:Summmary}
%%%%%%%%%%%%%%%%%%%%%%%%%%%%%%%%%%%%%%%%%%%%%%%%%%

What should we place in this section?


%%%%%%%%%%%%%%%%%%%%%%%%%%%%%%%%%%%%%%%%%%%%%%%%%%
\subsection{Unbalanced Observations}\label{sec:summary_unbalanced_data}
%%%%%%%%%%%%%%%%%%%%%%%%%%%%%%%%%%%%%%%%%%%%%%%%%%

The \pkg{systemfit} package was originally developed to fit
simultaneous equations for forestry datasets. Forestry datasets
typically contain observations for many inexpensive observations
(diameter at breast height) and very few expensive observations such
as total tree height. While the package currently does not allow for
unbalanced datasets, data where the equations contain different
numbers of observations, future releases of the package will have that
feature implemented. We are still trying to figure out how develop the
code for instrumental variable methods (2SLS,3SLS) for the correct
estimation methods if possible.

%%%%%%%%%%%%%%%%%%%%%%%%%%%%%%%%%%%%%%%%%%%%%%%%%%
\subsection{Additional Methods}\label{sec:summary_additional_methods}
%%%%%%%%%%%%%%%%%%%%%%%%%%%%%%%%%%%%%%%%%%%%%%%%%%


\subsubsection{Full information maximum likelihood (FIML)}

Describe, in a paragraph the process and give some reasons why you
might want this in the package. 


\subsubsection{Generalized method of moments (GMM)}

Describe, in a paragraph the process and give some reasons why you
might want this in the package. 



%%%%%%%%%%%%%%%%%%%%%%%%%%%%%%%%%%%%%%%%%%%%%%%%%%
\subsubsection{Standard usage}

\pkg{systemfit} is generally called by

\code{
R> systemfit( method, eqns )
}

There are two mandatory arguments: \code{method} and \code{eqns}.

The argument \code{method} is a string determining the estimation method.
It must be one of "OLS", "WLS", "SUR", "2SLS", "W2SLS" or "3SLS".

The other mandatory argument \code{eqns} is a list of the equations 
to estimate. 
Each equation is a standard formula in \proglang{R}.
It starts with a dependent variable on the left hand side.
After a tilde ($\sim$) the regressors are listed%
\footnote{For Details see the \proglang{R} help files to \code{formula}}.

This is now demonstrated using an example: \\
\code{
R> library( systemfit ) \\
R> data( kmenta ) \\
R> attach( kmenta ) \\
R> fitsur <- systemfit( "SUR", list( q $\sim$ p + d, q $\sim$ p + f + a ) ) \\
}

The first line loads the \pkg{systemfit} package. 
The second line loads example data that are included in this package.
These data come from \cite{kmenta86}.
They are attached to the \proglang{R} search path in line three.
In the last line a seemingly unrelated regression is done.
The first equation represents the demand side of the food market.
The dependant variable is \code{q} (food consumption per capita). 
The regressors are \code{p} (ratio of food prices to general consumer prices)
and \code{d} (disposable income) as well as a constant%
\footnote{a regression constant is always implied if not explicitly omitted.}.
The second equation represents the supply side.
Variable \code{q} (food consumption per capita) is also the dependant 
variable of this equation. 
The regressors are again \code{p} (ratio of food prices to general 
consumer prices) and a constant as well as 
\code{f} (ratio of preceding year's prices received by farmers) and 
\code{a} (a time trend in years).
The regression result is assigned to the variable \code{fitsur}.

Summary results can be printed by\\
\code{
R>~summary(~fitsur~)~\\
~\\
systemfit~results~\\
method:~SUR~\\
\\
\mbox{}~~~N~DF~~~~~~SSR~~~~~MSE~~~~RMSE~~~~~~~R2~~~Adj~R2 \\
1~20~17~~65.6829~3.86370~1.96563~0.755019~0.726198~\\
2~20~16~104.0584~6.50365~2.55023~0.611888~0.539117~\\
~\\
The~covariance~matrix~of~the~residuals~used~for~estimation\\
\mbox{}~~~~~~~~1~~~~~~~2~\\
1~3.72539~4.13696~\\
2~4.13696~5.78444~\\
~\\
The~covariance~matrix~of~the~residuals\\
\mbox{}~~~~~~~~1~~~~~~~2~\\
1~3.86370~4.92431~\\
2~4.92431~6.50365~\\
~\\
The~correlations~of~the~residuals\\
\mbox{}~~~~~~~~~1~~~~~~~~2~\\
1~1.000000~0.982348~\\
2~0.982348~1.000000~\\
~\\
The~determinant~of~the~residual~covariance~matrix:~0.879285~\\
OLS~R-squared~value~of~the~system:~0.683453~\\
McElroy's~R-squared~value~for~the~system:~0.788722~\\
~\\
SUR~estimates~for~1~~(equation~1~)~\\
Model~Formula:~q~~~p~+~d\\
\\
\mbox{}~~~~~~~~~~~~~Estimate~Std.~Error~~~t~value~Pr(>|t|)~\\
(Intercept)~99.332894~~~7.514452~13.218913~~~~~~~~0~***~\\
p~~~~~~~~~~~-0.275486~~~0.088509~-3.112513~0.006332~~**~\\
d~~~~~~~~~~~~~0.29855~~~0.041945~~7.117605~~~~2e-06~***~\\
---~\\
Signif.~codes:~~0~`***'~0.001~`**'~0.01~`*'~0.05~`.'~0.1~`~'~1~\\
~\\
Residual~standard~error:~1.96563~on~17~degrees~of~freedom~\\
Number~of~observations:~20~Degrees~of~Freedom:~17~\\
SSR:~65.682902~MSE:~3.8637~Root~MSE:~1.96563~\\
Multiple~R-Squared:~0.755019~Adjusted~R-Squared:~0.726198~\\
~\\
~\\
SUR~estimates~for~2~~(equation~2~)~\\
Model~Formula:~q~~~p~+~f~+~a\\
\\
\mbox{}~~~~~~~~~~~~~Estimate~Std.~Error~~t~value~Pr(>|t|)~\\
(Intercept)~61.966166~~~11.08079~5.592215~~~~4e-05~***~\\
p~~~~~~~~~~~~0.146884~~~0.094435~1.555397~0.139408~\\
f~~~~~~~~~~~~0.214004~~~0.039868~5.367761~~6.3e-05~***~\\
a~~~~~~~~~~~~0.339304~~~0.067911~4.996283~0.000132~***~\\
---~\\
Signif.~codes:~~0~`***'~0.001~`**'~0.01~`*'~0.05~`.'~0.1~`~'~1~\\
~\\
Residual~standard~error:~2.550226~on~16~degrees~of~freedom~\\
Number~of~observations:~20~Degrees~of~Freedom:~16~\\
SSR:~104.05843~MSE:~6.503652~Root~MSE:~2.550226~\\
Multiple~R-Squared:~0.611888~Adjusted~R-Squared:~0.539117~\\
}



%%%%%%%%%%%%%%%%%%%%%%%%%%%%%%%%%%%%%%%%%%%%%%%%%%
\subsubsection{User options}

Following additional options can be set by the user:

%%%%%%%%%%%%%%%%%%%%%%%%%%%%%%%%%%%%%%%%%%%%%%%%%%
% \subsubsection{Equation labels}
\paragraph{Equation labels}
The optional argument \code{eqnlabels} allows the user to label the equations.
It has to be a vector of strings naming the equations.\\
\code{
R>~fitsur~<-~systemfit(~"SUR",~list(~q~~~p~+~d,~q~~~p~+~f~+~a~),\\~
R+~~~~eqnlabels~=~c(~"demand",~"supply"~)~)\\
R>~summary(~fitsur~)\\
systemfit~results\\
method:~SUR\\
\\
\mbox{}~~~~~~~~N~DF~~~~~~SSR~~~~~MSE~~~~RMSE~~~~~~~R2~~~Adj~R2\\
demand~20~17~~65.6829~3.86370~1.96563~0.755019~0.726198\\
supply~20~16~104.0584~6.50365~2.55023~0.611888~0.539117\\
\ldots\\
}
If no equation labels are provided, the equations are numbered.
 
%%%%%%%%%%%%%%%%%%%%%%%%%%%%%%%%%%%%%%%%%%%%%%%%%%
%\subsubsection{Instrumental variables}   
\paragraph{Instrumental variables}   
\code{inst}one-sided model formula specifying instrumental variables
   or a list of one-sided model formulas if different instruments should
   be used for the different equations (only needed for 2SLS, W2SLS and
   3SLS estimations).

%%%%%%%%%%%%%%%%%%%%%%%%%%%%%%%%%%%%%%%%%%%%%%%%%%
%\subsubsection{Data}   
\paragraph{Data}   
\code{data} an optional data frame containing the variables in the model.
   By default the variables are taken from the environment from which
   systemfit is called.

%%%%%%%%%%%%%%%%%%%%%%%%%%%%%%%%%%%%%%%%%%%%%%%%%%
%\subsubsection{Restrictions}   
\paragraph{Restrictions}   
\code{R.restr} an optional j x k matrix to impose linear
   restrictions on the parameters by \code{R.restr} * $\beta$ = \code{q.restr}
   (j = number of restrictions, k = number of all parameters,
   $\beta$ = vector of all parameters).

\code{q.restr} an optional j x 1 matrix to impose linear
   restrictions (see \code{R.restr}); default is a j x 1 matrix
   that contains only zeros.

\code{TX} an optional matrix to transform the regressor matrix and,
   hence, also the coefficient vector (see details).

%%%%%%%%%%%%%%%%%%%%%%%%%%%%%%%%%%%%%%%%%%%%%%%%%%
%\subsubsection{Iteration control}   
\paragraph{Iteration control}
\code{maxiter} maximum number of iterations for WLS, SUR, W2SLS and
   3SLS estimations.

\code{tol} tolerance level indicating when to stop the iteration (only
   WLS, SUR, W2SLS and 3SLS estimations).

%%%%%%%%%%%%%%%%%%%%%%%%%%%%%%%%%%%%%%%%%%%%%%%%%%
%\subsubsection{Residual covariance matrix}   
\paragraph{Residual covariance matrix}   
\code{rcovformula} formula to calculate the estimated residual covariance
   matrix (see details).

%%%%%%%%%%%%%%%%%%%%%%%%%%%%%%%%%%%%%%%%%%%%%%%%%%
%\subsubsection{3SLS formula}   
\paragraph{3SLS formula}   
\code{formula3sls} formula for calculating the 3SLS estimator,
   one of "GLS", "IV", "GMM", "Schmidt" or "EViews" (see details).

%%%%%%%%%%%%%%%%%%%%%%%%%%%%%%%%%%%%%%%%%%%%%%%%%%
%\subsubsection{Degrees of freedom for t-tests}   
\paragraph{Degrees of freedom for t-tests}   
\code{probdfsys} use the degrees of freedom of the whole system
   (in place of the degrees of freedom of the single equation)
   to calculate prob values for the t-test of individual parameters.

%%%%%%%%%%%%%%%%%%%%%%%%%%%%%%%%%%%%%%%%%%%%%%%%%%
%\subsubsection{Sigma squared}   
\paragraph{Sigma squared}   
\code{single.eq.sigma} use different $\sigma^2$s for each
   single equation to calculate the covariance matrix and the
   standard errors of the coefficients (only OLS and 2SLS).

%%%%%%%%%%%%%%%%%%%%%%%%%%%%%%%%%%%%%%%%%%%%%%%%%%
%\subsubsection{System options}   
\paragraph{System options}   

\code{solvetol} tolerance level for detecting linear dependencies
   when inverting a matrix or calculating a determinant (see
   \code{solve} and \code{det}).

\code{saveMemory} save memory by omitting some calculation that
   are not crucial for the basic estimation (e.g McElroy's
   $R^2$).



%%%%%%%%%%%%%%%%%%%%%%%%%%%%%%%%%%%%%%%%%%%%%%%
\section{Estimating nonlinear equation systems}
%%%%%%%%%%%%%%%%%%%%%%%%%%%%%%%%%%%%%%%%%%%%%%%

The \pkg{systemfit} package also contains a method for fitting
nonlinear systems of equations.




%%%%%%%%%%%%%%%%%%%%%%%%%%%%%%%%%%%%%%%%%%%%%%%
\subsection{Nonlinear parameter estimation}
\label{sec:nonlinear_estimation}
%%%%%%%%%%%%%%%%%%%%%%%%%%%%%%%%%%%%%%%%%%%%%%%

A system of nonlinear equations can be written as:
  
\begin{equation}
  \label{eq:non_linear_eq_1}
  \epsilon_{t} = q( y_t, x_t, \beta )
\end{equation}
%%% It is not clear what these variable are. 
%%% Adding an equation representing only a single equation
%%% would make things clearer.

\noindent and

\begin{equation}
  \label{eq:non_linear_eq_2}
  z_{t} = Z( x_t )
\end{equation}

where $\epsilon_{t}$ are the residuals from the y observations and
$z_{t}$ are the function evaluated at the parameter estimates.

\textbf{you really need to check this!!!}

The objective functions for the methods are:
  
\begin{center}
\begin{tabular}{|l|c|c|c|} \hline
  Method & Instruments & Objective Function & Covariance of $\theta$ \\ \hline  
  OLS & No & $r'r$ & $(X(diag(S)^{-1}\otimes I)X)^{-1}$ \\ \hline
  SUR & No & $r'(diag(S)_{OLS}^{-1}\otimes I)r$ & $(X(S^{-1}\otimes I)X)^{-1}$ \\ \hline
  2SLS & Yes & $r'(I \otimes W)r$ & $(X(diag(S)^{-1}\otimes I)X)^{-1}$ \\ \hline 
  3SLS & Yes & $r'(S_{2SLS}^{-1} \otimes W)r$ & $(X(diag(S)^{-1}\otimes W)X)^{-1}$ \\ \hline
\end{tabular}
\end{center}

where, $r$ is a column vector for the residuals for each equation,
%% what is the difference to \epsilon?
$X$ is matrix of the partial derivatives of the dependent variable 
with respect to the parameters $\left( \frac{ \partial y }{ \theta} \right)$,
%% is this correct?
%% in the linear section X is used for the regressors, 
%% thus we should use a different variable name here.
$W$ is a matrix of the instrument variables, $Z(Z'Z)^{-1}Z$, $Z$ is a
matrix of the instrument variables, and $I$ is an $n \times n $
identity matrix and $S$ is the estimated variance-covariance matrix between the
equations

\begin{equation}
  \label{eq:non-linear_varcov}
  \hat{\sigma}_{ij} = (\hat{e}_i' \hat{e}_j) / \sqrt{(T - k_i)*(T- k_j)} 
\end{equation}

\textbf{The residual variance-covariance matrix can be calculated in 
different ways (section~\ref{sec:residcov}). 
It should be relatively easy to implement this also in nlsystemfit(), 
e.g. the function 'calcRCov' in systemfit() could be moved outside
systemfit() that it can be used also by nlsystemfit().}

\textbf{You need to clear this up.}

%%%%%%%%%%%%%%%%%%%%%%%%%%%%%%%%%%%%%%%%%%%%%%%%%%
\subsection{Using nlsystemfit}\label{sec:UsingnlSystemfit}
%%%%%%%%%%%%%%%%%%%%%%%%%%%%%%%%%%%%%%%%%%%%%%%%%%

\code {nlsystemfit} fits a set of structural nonlinear equations using
Ordinary Least Squares (OLS), Seemingly Unrelated Regression (SUR),
Two-Stage Least Squares (2SLS), Three-Stage Least Squares (3SLS) using
the objective functions described in section
\ref{sec:nonlinear_estimation}.

%%%%%%%%%%%%%%%%%%%%%%%%%%%%%%%%%%%%%%%%%%%%%%%%%%
\subsubsection{Standard usage}

Similar to calling the \code{systemfit} function, \code{nlsystemfit}
is called with a minimum of three arguments,

\code{
R> nlsystemfit( method, eqns, start )
}

where \code{method} is one of the following estimation methods: "OLS",
"SUR", "2SLS", or "3SLS", \code{eqns} is a list of equations similar
to those described for \code{systemfit}, and \code{start} is a list of
starting values of the parameter estimates.

% The mandatory argument \code{eqns} is a list of the equations  to
% estimate.  Each equation is a standard formula in \proglang{R}.  It
% starts with a dependent variable on the left hand side.  After a tilde
% ($~$) the regressors are listed  \footnote{For Details see the
%   \proglang{R} help files to \code{formula}}.

This is now demonstrated using an example: \\

% \code{
% R> library( systemfit ) \\
% R> data( ppine ) \\
% R> hg.formula <- hg $\sim$ exp( h0 + h1*log(tht) + h2*tht$\hat$2 + h3*elev + h4*cr) \\
% R> dg.formula <- dg $\sim$ exp( d0 + d1*log(dbh) + d2*hg + d3*cr + d4*ba  ) \\
% R> labels <- list( "height.growth", "diameter.growth" )\\
% R> inst <- $\sim$ tht + dbh + elev + cr + ba\\
% R> start.values <- c(h0=-0.5, h1=0.5, h2=-0.001, h3=0.0001, h4=0.08,\\
% R+   d0=-0.5, d1=0.009, d2=0.25, d3=0.005, d4=-0.02 )\\
% R> model <- list( hg.formula, dg.formula )\\
% R> model.3sls <- nlsystemfit( "3SLS", model, start.values, data=ppine,\\
% R+   eqnlabels=labels, inst=inst ) \\
% R> summary( model.3sls ) \\
% }


The nlsystemfit function relies on \code{nlm} to perform the
minimization of the objective functions and the \code{qr} set of
functions. If the user does not have a set of estimates for the
initial parameters, it is suggested using linearized forms in one of
the linear methods, or simply using \code{nls} to obtain estimates for
the equations. 

The outputs are similar to those for a systemfit object. 

\textbf{Describe the code and outputs as Arne did in the previous
  section.}

The first line loads the \pkg{systemfit} package. 
The second line loads example data that are included in this package.
These data come from \cite{kmenta86}.
They are attached to the \proglang{R} search path in line three.
In the last line a seemingly unrelated regression is done.
The first equation represents the demand side of the food market.
The dependant variable is \code{q} (food consumption per capita). 
The regressors are \code{p} (ratio of food prices to general consumer prices)
and \code{disposable income} as well as a constant%
\footnote{a regression constant is always implied if not explicitly omitted.}.
The second equation represents the supply side.
Variable \code{q} (food consumption per capita) is also the dependant 
variable of this equation. 
The regressors are again \code{p} (ratio of food prices to general 
consumer prices) and a constant as well as 
\code{f} (ratio of preceding year's prices received by farmers) and 
\code{a} (a time trend in years).
The regression result is assigned to the variable \code{fitsur}.

%% Arne -- Did you put the ~'s in by hand?
%% Yes, I did. I used find and replace to substitute the "~" for " ". 
%% This is necessary to retain the format of the output.

Summary results can be printed by\\
\code{
R>~summary(~fitsur~)~\\
~\\
systemfit~results~\\
method:~SUR~\\
\\
\mbox{}~~~N~DF~~~~~~SSR~~~~~MSE~~~~RMSE~~~~~~~R2~~~Adj~R2 \\
1~20~17~~65.6829~3.86370~1.96563~0.755019~0.726198~\\
2~20~16~104.0584~6.50365~2.55023~0.611888~0.539117~\\
~\\
The~covariance~matrix~of~the~residuals~used~for~estimation\\
\mbox{}~~~~~~~~1~~~~~~~2~\\
1~3.72539~4.13696~\\
2~4.13696~5.78444~\\
~\\
The~covariance~matrix~of~the~residuals\\
\mbox{}~~~~~~~~1~~~~~~~2~\\
1~3.86370~4.92431~\\
2~4.92431~6.50365~\\
~\\
The~correlations~of~the~residuals\\
\mbox{}~~~~~~~~~1~~~~~~~~2~\\
1~1.000000~0.982348~\\
2~0.982348~1.000000~\\
~\\
The~determinant~of~the~residual~covariance~matrix:~0.879285~\\
OLS~R-squared~value~of~the~system:~0.683453~\\
McElroy's~R-squared~value~for~the~system:~0.788722~\\
~\\
SUR~estimates~for~1~~(equation~1~)~\\
Model~Formula:~q~~~p~+~d\\
\\
\mbox{}~~~~~~~~~~~~~Estimate~Std.~Error~~~t~value~Pr(>|t|)~\\
(Intercept)~99.332894~~~7.514452~13.218913~~~~~~~~0~***~\\
p~~~~~~~~~~~-0.275486~~~0.088509~-3.112513~0.006332~~**~\\
d~~~~~~~~~~~~~0.29855~~~0.041945~~7.117605~~~~2e-06~***~\\
---~\\
Signif.~codes:~~0~`***'~0.001~`**'~0.01~`*'~0.05~`.'~0.1~`~'~1~\\
~\\
Residual~standard~error:~1.96563~on~17~degrees~of~freedom~\\
Number~of~observations:~20~Degrees~of~Freedom:~17~\\
SSR:~65.682902~MSE:~3.8637~Root~MSE:~1.96563~\\
Multiple~R-Squared:~0.755019~Adjusted~R-Squared:~0.726198~\\
~\\
~\\
SUR~estimates~for~2~~(equation~2~)~\\
Model~Formula:~q~~~p~+~f~+~a\\
\\
\mbox{}~~~~~~~~~~~~~Estimate~Std.~Error~~t~value~Pr(>|t|)~\\
(Intercept)~61.966166~~~11.08079~5.592215~~~~4e-05~***~\\
p~~~~~~~~~~~~0.146884~~~0.094435~1.555397~0.139408~\\
f~~~~~~~~~~~~0.214004~~~0.039868~5.367761~~6.3e-05~***~\\
a~~~~~~~~~~~~0.339304~~~0.067911~4.996283~0.000132~***~\\
---~\\
Signif.~codes:~~0~`***'~0.001~`**'~0.01~`*'~0.05~`.'~0.1~`~'~1~\\
~\\
Residual~standard~error:~2.550226~on~16~degrees~of~freedom~\\
Number~of~observations:~20~Degrees~of~Freedom:~16~\\
SSR:~104.05843~MSE:~6.503652~Root~MSE:~2.550226~\\
Multiple~R-Squared:~0.611888~Adjusted~R-Squared:~0.539117~\\
}



%%%%%%%%%%%%%%%%%%%%%%%%%%%%%%%%%%%%%%%%%%%%%%%%%%
\subsection{Other issues}


\subsubsection{nlsystemfit issue 1}
% The user should be aware that the function is \bold{VERY} sensative to
% the starting values and the nlm function may not converge. 

% did you try optim()? Especially if you provide (analytical) derivatives
% it is very good in finding the optimum.

% The nlm
% function will be called with the \code{typsize} argument set the
% absolute values of the starting values for the OLS and 2SLS
% methods. For the SUR and 3SLS methods, the \code{typsize} argument is
% set to the absolute values of the resulting OLS and 2SLS parameter
% estimates from the nlm result structre. In addition, the starting
% values for the SUR and 3SLS methods are obtained from the OLS and 2SLS
% parameter estimates to shorten the number of iterations. The number of
% iterations reported in the summary are only those used in the last
% call to nlm, thus the number of iterations in the OLS portion of the
% SUR fit and the 2SLS portion of the 3SLS fit are not included.  }




%%%%%%%%%%%%%%%%%%%%%%%%%%%%%%%%%%%%%%%%%%%%%%%%%%
\section{Other tools}


  
%%%%%%%%%%%%%%%%%%%%%%%%%%%%%%%%%%%%%%%%%%%%%%%%%%
\subsection{Likelihood ratio test}

%%%%%%%%%%%%%%%%%%%%%%%%%%%%%%%%%%%%%%%%%%%%%%%%%%
\subsection{Hausman test}

\textbf{Man, I really have to clean this up.}

Hausman \citep{hausman1978} developed a test have power against
alternatives for which $\hat{\theta}$ and $\tilde{\theta}$ diverge
under misspecification.

The Hausman result if the difference between the covariance of an
efficient and the covariance of the inefficient estimator is zero, the
model is correctly specified.  

The test is based on the idea that under the hypothesis of no
correlation, both the limited information estimates and the
full-information results are consistent, but only the full-information
estimates are efficient and under the alternative, the limited
information estimates are consistent, but the full information
estimates are not. 

Since the models that are fit using the \pkg{systemfit} package are
typically random effects models, the effects may be inconsistent
between the introduced variables and the regressors.


%%%%%%%%%%%%%%%%%%%%%%%%%%%%%%%%%%%%%%%%%%%%%%%%%%
\section{Summary and outlook}\label{sec:Summmary}
%%%%%%%%%%%%%%%%%%%%%%%%%%%%%%%%%%%%%%%%%%%%%%%%%%

What should we place in this section?


%%%%%%%%%%%%%%%%%%%%%%%%%%%%%%%%%%%%%%%%%%%%%%%%%%
\subsection{Unbalanced Observations}\label{sec:summary_unbalanced_data}
%%%%%%%%%%%%%%%%%%%%%%%%%%%%%%%%%%%%%%%%%%%%%%%%%%

The \pkg{systemfit} package was originally developed to fit
simultaneous equations for forestry datasets. Forestry datasets
typically contain observations for many inexpensive observations
(diameter at breast height) and very few expensive observations such
as total tree height. While the package currently does not allow for
unbalanced datasets, data where the equations contain different
numbers of observations, future releases of the package will have that
feature implemented. We are still trying to figure out how develop the
code for instrumental variable methods (2SLS,3SLS) for the correct
estimation methods if possible.

%%%%%%%%%%%%%%%%%%%%%%%%%%%%%%%%%%%%%%%%%%%%%%%%%%
\subsection{Additional Methods}\label{sec:summary_additional_methods}
%%%%%%%%%%%%%%%%%%%%%%%%%%%%%%%%%%%%%%%%%%%%%%%%%%


\subsubsection{Full Information Maximum Likelihood}

Describe, in a paragraph the process and give some reasons why you
might want this in the package. 


\subsubsection{Generalized Method of Moments}

Describe, in a paragraph the process and give some reasons why you
might want this in the package. 




\section{CVS Revision Log}
\begin{verbatim}
$Log$
Revision 1.14  2005/10/26 21:16:53  hamannj
checked out and updated local copy before edits

Revision 1.13  2005/09/26 20:34:12  henningsena
using now package csquotes to have uniform and correct quotation marks (BibTeX-Style jss.bst has to be modified)

Revision 1.12  2005/09/26 11:24:05  henningsena
some minor changes

Revision 1.11  2005/09/01 12:27:55  henningsena
revised structure of the document (sections, subsections, subsubsections) due to removal of nlsystemfit; moved each section into a separate file.

Revision 1.10  2005/09/01 11:49:02  henningsena
moved section about non-linear estimation from systemfit.tex into new file nlsystemfit.tex, because this section should not be included in (this version of) the article

Revision 1.9  2004/11/16 10:59:09  henningsena
some minor changes and some comments on nlsystemfit

Revision 1.8  2004/11/16 08:55:53  henningsena
merged file and resolved conflict due to some minor changes I did while Jeff was adding nlsystemfit.

Revision 1.7  2004/11/16 05:34:47  hamannj
question about nlsystemfit code section (example)

Revision 1.6  2004/11/16 05:32:46  hamannj
made small changes because my code section wouldn't compile. argh.

Revision 1.5  2004/11/16 05:31:19  hamannj
added cvs header and log section. to be removed when submitted.

\end{verbatim}




%%%%%%%%%%%%%%%%%%%%%%%%%%%%%%%%%%%%%%%%%%%%%%%%%%
\bibliography{systemfit} % a subset of my big bibtex file
%\bibliography{/home/suapm095/Documents/Literatur/arne}
%%%%%%%%%%%%%%%%%%%%%%%%%%%%%%%%%%%%%%%%%%%%%%%%%%





\end{document}


