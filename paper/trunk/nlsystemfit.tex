%%%%%%%%%%%%%%%%%%%%%%%%%%%%%%%%%%%%%%%%%%%%%%%
\section{Estimating nonlinear equation systems}
%%%%%%%%%%%%%%%%%%%%%%%%%%%%%%%%%%%%%%%%%%%%%%%

The \pkg{systemfit} package also contains a method for fitting
nonlinear systems of equations.




%%%%%%%%%%%%%%%%%%%%%%%%%%%%%%%%%%%%%%%%%%%%%%%
\subsection{Nonlinear parameter estimation}
\label{sec:nonlinear_estimation}
%%%%%%%%%%%%%%%%%%%%%%%%%%%%%%%%%%%%%%%%%%%%%%%

A system of nonlinear equations can be written as:
  
\begin{equation}
  \label{eq:non_linear_eq_1}
  \epsilon_{t} = q( y_t, x_t, \theta )
\end{equation}
%%% It is not clear what these variable are. 
%%% Adding an equation representing only a single equation
%%% would make things clearer.

\noindent and

\begin{equation}
  \label{eq:non_linear_eq_2}
  z_{t} = Z( x_t )
\end{equation}

where $\epsilon_{t}$ are the residuals from the y observations and
$z_{t}$ are the function evaluated at the parameter estimates.

\textbf{you really need to check this!!!}



The objective functions for the methods are:
  
\begin{center}
\begin{tabular}{|l|c|c|c|} \hline
  Method & Instruments & Objective Function & Covariance of $\theta$ \\ \hline  
  OLS & No & $r'r$ & $(X(diag(S)^{-1}\otimes I)X)^{-1}$ \\ \hline
  SUR & No & $r'(diag(S)_{OLS}^{-1}\otimes I)r$ & $(X(S^{-1}\otimes I)X)^{-1}$ \\ \hline
  2SLS & Yes & $r'(I \otimes W)r$ & $(X(diag(S)^{-1}\otimes I)X)^{-1}$ \\ \hline 
  3SLS & Yes & $r'(S_{2SLS}^{-1} \otimes W)r$ & $(X(diag(S)^{-1}\otimes W)X)^{-1}$ \\ \hline
\end{tabular}
\end{center}

where, $r$ is a column vector for the residuals for each equation,
%% what is the difference to \epsilon?
$X$ is matrix of the partial derivatives of the dependent variable 
with respect to the parameters $\left( \frac{ \partial y }{ \theta} \right)$,
%% is this correct?
%% in the linear section X is used for the regressors, 
%% thus we should use a different variable name here.
$W$ is a matrix of the instrument variables, $Z(Z'Z)^{-1}Z$, $Z$ is a
matrix of the instrument variables, and $I$ is an $n \times n $
identity matrix and $S$ is the estimated covariance matrix between the
equations

\begin{equation}
  \label{eq:non-linear_varcov}
  \hat{\sigma}_{ij} = (\hat{e}_i' \hat{e}_j) / \sqrt{(T - k_i)*(T- k_j)} 
\end{equation}

\textbf{The residual covariance matrix can be calculated in
different ways (section~\ref{sec:residcov}). 
It should be relatively easy to implement this also in nlsystemfit(), 
e.g. the function 'calcRCov' in systemfit() could be moved outside
systemfit() that it can be used also by nlsystemfit().}

\textbf{You need to clear this up.}

%%%%%%%%%%%%%%%%%%%%%%%%%%%%%%%%%%%%%%%%%%%%%%%%%%
\subsection{Using nlsystemfit}\label{sec:UsingnlSystemfit}
%%%%%%%%%%%%%%%%%%%%%%%%%%%%%%%%%%%%%%%%%%%%%%%%%%

\code {nlsystemfit} fits a set of structural nonlinear equations using
Ordinary Least Squares (OLS), Seemingly Unrelated Regression (SUR),
Two-Stage Least Squares (2SLS), Three-Stage Least Squares (3SLS) using
the objective functions described in section
\ref{sec:nonlinear_estimation}.

%%%%%%%%%%%%%%%%%%%%%%%%%%%%%%%%%%%%%%%%%%%%%%%%%%
\subsubsection{Standard usage}

Similar to calling the \code{systemfit} function, \code{nlsystemfit}
is called with a minimum of three arguments,

\code{
R> nlsystemfit( method, eqns, start )
}

where \code{method} is one of the following estimation methods: "OLS",
"SUR", "2SLS", or "3SLS", \code{eqns} is a list of equations similar
to those described for \code{systemfit}, and \code{start} is a list of
starting values of the parameter estimates.

% The mandatory argument \code{eqns} is a list of the equations  to
% estimate.  Each equation is a standard formula in \proglang{R}.  It
% starts with a dependent variable on the left hand side.  After a tilde
% ($~$) the regressors are listed  \footnote{For Details see the
%   \proglang{R} help files to \code{formula}}.

This is now demonstrated using an example: \\

% \code{
% R> library( systemfit ) \\
% R> data( ppine ) \\
% R> hg.formula <- hg $\sim$ exp( h0 + h1*log(tht) + h2*tht$\hat$2 + h3*elev + h4*cr) \\
% R> dg.formula <- dg $\sim$ exp( d0 + d1*log(dbh) + d2*hg + d3*cr + d4*ba  ) \\
% R> labels <- list( "height.growth", "diameter.growth" )\\
% R> inst <- $\sim$ tht + dbh + elev + cr + ba\\
% R> start.values <- c(h0=-0.5, h1=0.5, h2=-0.001, h3=0.0001, h4=0.08,\\
% R+   d0=-0.5, d1=0.009, d2=0.25, d3=0.005, d4=-0.02 )\\
% R> model <- list( hg.formula, dg.formula )\\
% R> model.3sls <- nlsystemfit( "3SLS", model, start.values, data=ppine,\\
% R+   eqnlabels=labels, inst=inst ) \\
% R> summary( model.3sls ) \\
% }


The nlsystemfit function relies on \code{nlm} to perform the
minimization of the objective functions and the \code{qr} set of
functions. If the user does not have a set of estimates for the
initial parameters, it is suggested using linearized forms in one of
the linear methods, or simply using \code{nls} to obtain estimates for
the equations. 

The outputs are similar to those for a systemfit object. 

\textbf{Describe the code and outputs as Arne did in the previous
  section.}

The first line loads the \pkg{systemfit} package. 
The second line loads example data that are included in this package.
These data come from \cite{kmenta86}.
They are attached to the \proglang{R} search path in line three.
In the last line a seemingly unrelated regression is done.
The first equation represents the demand side of the food market.
The dependant variable is \code{q} (food consumption per capita). 
The regressors are \code{p} (ratio of food prices to general consumer prices)
and \code{disposable income} as well as a constant%
\footnote{a regression constant is always implied if not explicitly omitted.}.
The second equation represents the supply side.
Variable \code{q} (food consumption per capita) is also the dependant 
variable of this equation. 
The regressors are again \code{p} (ratio of food prices to general 
consumer prices) and a constant as well as 
\code{f} (ratio of preceding year's prices received by farmers) and 
\code{a} (a time trend in years).
The regression result is assigned to the variable \code{fitsur}.

%% Arne -- Did you put the ~'s in by hand?
%% Yes, I did. I used find and replace to substitute the "~" for " ". 
%% This is necessary to retain the format of the output.

Summary results can be printed by\\
\code{
R>~summary(~fitsur~)~\\
~\\
systemfit~results~\\
method:~SUR~\\
\\
\mbox{}~~~N~DF~~~~~~SSR~~~~~MSE~~~~RMSE~~~~~~~R2~~~Adj~R2 \\
1~20~17~~65.6829~3.86370~1.96563~0.755019~0.726198~\\
2~20~16~104.0584~6.50365~2.55023~0.611888~0.539117~\\
~\\
The~covariance~matrix~of~the~residuals~used~for~estimation\\
\mbox{}~~~~~~~~1~~~~~~~2~\\
1~3.72539~4.13696~\\
2~4.13696~5.78444~\\
~\\
The~covariance~matrix~of~the~residuals\\
\mbox{}~~~~~~~~1~~~~~~~2~\\
1~3.86370~4.92431~\\
2~4.92431~6.50365~\\
~\\
The~correlations~of~the~residuals\\
\mbox{}~~~~~~~~~1~~~~~~~~2~\\
1~1.000000~0.982348~\\
2~0.982348~1.000000~\\
~\\
The~determinant~of~the~residual~covariance~matrix:~0.879285~\\
OLS~R-squared~value~of~the~system:~0.683453~\\
McElroy's~R-squared~value~for~the~system:~0.788722~\\
~\\
SUR~estimates~for~1~~(equation~1~)~\\
Model~Formula:~q~~~p~+~d\\
\\
\mbox{}~~~~~~~~~~~~~Estimate~Std.~Error~~~t~value~Pr(>|t|)~\\
(Intercept)~99.332894~~~7.514452~13.218913~~~~~~~~0~***~\\
p~~~~~~~~~~~-0.275486~~~0.088509~-3.112513~0.006332~~**~\\
d~~~~~~~~~~~~~0.29855~~~0.041945~~7.117605~~~~2e-06~***~\\
---~\\
Signif.~codes:~~0~`***'~0.001~`**'~0.01~`*'~0.05~`.'~0.1~`~'~1~\\
~\\
Residual~standard~error:~1.96563~on~17~degrees~of~freedom~\\
Number~of~observations:~20~Degrees~of~Freedom:~17~\\
SSR:~65.682902~MSE:~3.8637~Root~MSE:~1.96563~\\
Multiple~R-Squared:~0.755019~Adjusted~R-Squared:~0.726198~\\
~\\
~\\
SUR~estimates~for~2~~(equation~2~)~\\
Model~Formula:~q~~~p~+~f~+~a\\
\\
\mbox{}~~~~~~~~~~~~~Estimate~Std.~Error~~t~value~Pr(>|t|)~\\
(Intercept)~61.966166~~~11.08079~5.592215~~~~4e-05~***~\\
p~~~~~~~~~~~~0.146884~~~0.094435~1.555397~0.139408~\\
f~~~~~~~~~~~~0.214004~~~0.039868~5.367761~~6.3e-05~***~\\
a~~~~~~~~~~~~0.339304~~~0.067911~4.996283~0.000132~***~\\
---~\\
Signif.~codes:~~0~`***'~0.001~`**'~0.01~`*'~0.05~`.'~0.1~`~'~1~\\
~\\
Residual~standard~error:~2.550226~on~16~degrees~of~freedom~\\
Number~of~observations:~20~Degrees~of~Freedom:~16~\\
SSR:~104.05843~MSE:~6.503652~Root~MSE:~2.550226~\\
Multiple~R-Squared:~0.611888~Adjusted~R-Squared:~0.539117~\\
}



%%%%%%%%%%%%%%%%%%%%%%%%%%%%%%%%%%%%%%%%%%%%%%%%%%
\subsection{Other issues}


\subsubsection{nlsystemfit issue 1}
% The user should be aware that the function is \bold{VERY} sensative to
% the starting values and the nlm function may not converge. 

% did you try optim()? Especially if you provide (analytical) derivatives
% it is very good in finding the optimum.

% The nlm
% function will be called with the \code{typsize} argument set the
% absolute values of the starting values for the OLS and 2SLS
% methods. For the SUR and 3SLS methods, the \code{typsize} argument is
% set to the absolute values of the resulting OLS and 2SLS parameter
% estimates from the nlm result structre. In addition, the starting
% values for the SUR and 3SLS methods are obtained from the OLS and 2SLS
% parameter estimates to shorten the number of iterations. The number of
% iterations reported in the summary are only those used in the last
% call to nlm, thus the number of iterations in the OLS portion of the
% SUR fit and the 2SLS portion of the 3SLS fit are not included.  }


