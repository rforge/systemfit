%%%%%%%%%%%%%%%%%%%%%%%%%%%%%%%%%%%%%%%%%%%%%%%%%%
\section{Summary and outlook}\label{sec:Summmary}
%%%%%%%%%%%%%%%%%%%%%%%%%%%%%%%%%%%%%%%%%%%%%%%%%%

What should we place in this section?


%%%%%%%%%%%%%%%%%%%%%%%%%%%%%%%%%%%%%%%%%%%%%%%%%%
\subsection{Unbalanced Observations}\label{sec:summary_unbalanced_data}
%%%%%%%%%%%%%%%%%%%%%%%%%%%%%%%%%%%%%%%%%%%%%%%%%%

The \pkg{systemfit} package was originally developed to fit
simultaneous equations for forestry datasets. Forestry datasets
typically contain observations for many inexpensive observations
(diameter at breast height) and very few expensive observations such
as total tree height. While the package currently does not allow for
unbalanced datasets, data where the equations contain different
numbers of observations, future releases of the package will have that
feature implemented. We are still trying to figure out how develop the
code for instrumental variable methods (2SLS,3SLS) for the correct
estimation methods if possible.

%%%%%%%%%%%%%%%%%%%%%%%%%%%%%%%%%%%%%%%%%%%%%%%%%%
\subsection{Additional Methods}\label{sec:summary_additional_methods}
%%%%%%%%%%%%%%%%%%%%%%%%%%%%%%%%%%%%%%%%%%%%%%%%%%


\subsubsection{Full information maximum likelihood (FIML)}

Describe, in a paragraph the process and give some reasons why you
might want this in the package. 


\subsubsection{Generalized method of moments (GMM)}

Describe, in a paragraph the process and give some reasons why you
might want this in the package. 
