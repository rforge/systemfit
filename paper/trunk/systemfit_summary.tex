%%%%%%%%%%%%%%%%%%%%%%%%%%%%%%%%%%%%%%%%%%%%%%%%%%
\section{Summary and outlook}\label{sec:Summmary}
%%%%%%%%%%%%%%%%%%%%%%%%%%%%%%%%%%%%%%%%%%%%%%%%%%

The \pkg{systemfit} package was originally developed to fit
simultaneous equations for forestry datasets. Forestry datasets
typically contain observations of many inexpensive observations (stem
diameter) and few expensive observations such as stem height and being
able the utilize as many observations as possible is advantageous. As
most software packages simply drop data points that do not contain
both observations, the \pkg{systemfit} package currently does not
allow for unbalanced datasets, but it is our intention to include this
capability for certain estimation methods in future releases.

In this article, we have described some of the basic features of the
\pkg{systemfit} package for estimation of linear systems of
equations. It has been tested on a variety of datasets and has
produced satisfactory for a few years. While more sophisticated tools
exist, the \pkg{systemfit} package will perform the basic fitting
methods. We wish to include more sophisticated estimation methods such
as full-information maximum likelihood and generalized methods of
moments.


%%% Local Variables: 
%%% mode: latex
%%% TeX-master: "systemfit"
%%% End: 
