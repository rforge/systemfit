
%       $Id$    

%%%%%%%%%%%%%%%%%%%%%%%%%%%%%%%%%%%%%%%%%%%%%%%%%%
\section{Summary and outlook}\label{sec:Summmary}
%%%%%%%%%%%%%%%%%%%%%%%%%%%%%%%%%%%%%%%%%%%%%%%%%%
\nopagebreak
In this article, we have described some of the basic features of the
\pkg{systemfit} package for estimating of systems of linear
equations.
Many details of the estimation can be controlled by the user.
Furthermore, the package provides some statistical tests
for parameter restrictions and consistency of 3SLS estimation.
It has been tested on a variety of datasets and has produced satisfactory
for a few years.
While the \pkg{systemfit} package performs the basic fitting methods,
more sophisticated tools exist.
We hope to implement missing functionalities
in the near future.
% Some of these are discussed in the following.

\subsubsection*{Unbalanced datasets}
Currently, the \pkg{systemfit} package requires
that all equations have the same number of observations.
However, many data sets have unbalanced observations.%
\footnote{
For instance,
forestry datasets typically contain many observations of inexpensive
variables (stem diameter, tree count) and few expensive variables such
as stem height or volume.
}
Simply dropping data points that do not contain observations for all
equations may reduce the number of observations considerably, and
thus, the information utilized in the estimation.
Hence, it is our intention to include the capability for estimations
with unbalanced data sets as described in \citet{schmidt77} in future
releases of \pkg{systemfit}.

\subsubsection*{Serial correlation and heteroscedasticity}
For all of the methods developed in the package, the disturbances of
the individual equations are assumed to be independent and identically
distributed (iid).
The package could be enhanced by the inclusion of methods to fit
equations with serially correlated and heteroscedastic disturbances
\citep{parks67}. 

\subsubsection*{Estimation methods}
In the future, we wish to include more sophisticated estimation
methods such as limited information maximum likelihood (LIML),
full information maximum likelihood (FIML), generalized methods of
moments (GMM) and spatial econometric methods
\citep{paelinck79,anselin88}.

\subsubsection*{Non-linear estimation}
Finally, the \pkg{systemfit} package provides a function to estimate
systems of non-linear equations.
However, the function \code{nlsystemfit} is currently under
development and the results are not yet always reliable due to
convergence difficulties.


%%% Local Variables: 
%%% mode: latex
%%% TeX-master: "systemfit"
%%% End: 


