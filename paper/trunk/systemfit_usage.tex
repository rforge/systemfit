%%%%%%%%%%%%%%%%%%%%%%%%%%%%%%%%%%%%%%%%%%%%%%%%%
\section{Using systemfit}\label{sec:Usage}
%%%%%%%%%%%%%%%%%%%%%%%%%%%%%%%%%%%%%%%%%%%%%%%%%%

In this section we demonstrate how to use the \pkg{systemfit} package.
First, we show the standard usage of \code{systemfit} by a simple example.
Second, several options that can be specified by the user are presented.
Then, the wrapper function \code{systemfitClassic} is described.
Finally, we demonstrate how to apply some statistical tests.


%%%%%%%%%%%%%%%%%%%%%%%%%%%%%%%%%%%%%%%%%%%%%%%%%%
\subsection[Standard usage of systemfit]{Standard usage of \code{systemfit}}
\label{sec:standard-usage}

As described in the previous section,
equation systems can be econometrically estimated
with the function \code{systemfit}.
It is generally called by
\begin{Schunk}
\begin{Sinput}
> systemfit(method, eqns)
\end{Sinput}
\end{Schunk}

There are two mandatory arguments: \code{method} and \code{eqns}.
The argument \code{method} is a string determining the estimation method.
It must be either �OLS�, �WLS�, �SUR�, �WSUR�, �2SLS�, �W2SLS�, �3SLS�,
or �W3SLS�.
While six of these methods correspond to the estimation methods
described in sections~\ref{sec:Estimation} and~\ref{sec:Restrictions},
the methods �WSUR� and �W3SLS� are �SUR� and �3SLS� estimations
using the residual covariance matrices from
�WLS� and �W2SLS� estimations, respectively
(see section~\ref{sec:residcov}).
The other mandatory argument, \code{eqns}, is a list of the equations
to be estimated.
Each equation is a standard formula in \proglang{R} and
starts with a dependent variable on the left hand side.
After a tilde ($\sim$) the regressors are listed,
separated by plus signs%
\footnote{
For Details see the \proglang{R} help files to \code{formula}.
}.

The following demonstration uses an example taken from
\citet[p.\ 685]{kmenta86}.
We want to estimate a small model of US the food market:
\begin{align}
\texttt{consump} & = \beta_1 + \beta_2 * \texttt{price}
   + \beta_3 * \texttt{income} \\
\texttt{consump} & = \beta_4 + \beta_5 * \texttt{price}
   + \beta_6 * \texttt{farmPrice} + \beta_7 * \texttt{trend}
\end{align}
The first equation represents the demand side of the food market.
Variable \code{consump} (food consumption per capita) is the dependant variable.
The regressors are \code{price} (ratio of food prices to general consumer prices)
and \code{income} (disposable income) as well as a constant.
The second equation specifies the supply side of the food market.
Variable \code{consump} is the dependant variable of this equation as well.
The regressors are again \code{price} (ratio of food prices to general 
consumer prices) and a constant as well as 
\code{farmPrice} (ratio of preceding year's prices received by farmers) and 
\code{trend} (a time trend in years).
These equations can be estimated as SUR in \proglang{R} by
\begin{Schunk}
\begin{Sinput}
> library(systemfit)
> data(Kmenta)
> attach(Kmenta)
> eqDemand <- consump ~ price + income
> eqSupply <- consump ~ price + farmPrice + trend
> fitsur <- systemfit("SUR", list(demand = eqDemand, supply = eqSupply))
\end{Sinput}
\end{Schunk}

The first line loads the \pkg{systemfit} package.
The second line loads example data that are included with the package.
They are attached to the \proglang{R} search path in line three.
In the fourth and fifth line, the demand and supply equations are specified,
respectively%
\footnote{
A regression constant is always implied if not explicitly omitted.
}.
Finally, in the last line, a seemingly unrelated regression is performed
and the regression result is assigned to the variable \code{fitsur}.

Summary results can be printed by
\begin{Schunk}
\begin{Sinput}
> summary(fitsur)
\end{Sinput}
\begin{Soutput}
systemfit results 
method: SUR 

        N DF      SSR     MSE    RMSE       R2   Adj R2
demand 20 17  65.6829 3.86370 1.96563 0.755019 0.726198
supply 20 16 104.0584 6.50365 2.55023 0.611888 0.539117

The covariance matrix of the residuals used for estimation
        demand  supply
demand 3.72539 4.13696
supply 4.13696 5.78444

The covariance matrix of the residuals
        demand  supply
demand 3.86370 4.92431
supply 4.92431 6.50365

The correlations of the residuals
         demand   supply
demand 1.000000 0.982348
supply 0.982348 1.000000

The determinant of the residual covariance matrix: 0.879285
OLS R-squared value of the system: 0.683453
McElroy's R-squared value for the system: 0.788722

SUR estimates for 'demand' (equation 1)
Model Formula: consump ~ price + income

             Estimate Std. Error   t value Pr(>|t|)    
(Intercept) 99.332894   7.514452 13.218913        0 ***
price       -0.275486   0.088509 -3.112513 0.006332  **
income        0.29855   0.041945  7.117605    2e-06 ***
---
Signif. codes:  0 '***' 0.001 '**' 0.01 '*' 0.05 '.' 0.1 ' ' 1 

Residual standard error: 1.96563 on 17 degrees of freedom
Number of observations: 20 Degrees of Freedom: 17 
SSR: 65.682902 MSE: 3.8637 Root MSE: 1.96563 
Multiple R-Squared: 0.755019 Adjusted R-Squared: 0.726198 


SUR estimates for 'supply' (equation 2)
Model Formula: consump ~ price + farmPrice + trend

             Estimate Std. Error  t value Pr(>|t|)    
(Intercept) 61.966166   11.08079 5.592215    4e-05 ***
price        0.146884   0.094435 1.555397 0.139408    
farmPrice    0.214004   0.039868 5.367761  6.3e-05 ***
trend        0.339304   0.067911 4.996283 0.000132 ***
---
Signif. codes:  0 '***' 0.001 '**' 0.01 '*' 0.05 '.' 0.1 ' ' 1 

Residual standard error: 2.550226 on 16 degrees of freedom
Number of observations: 20 Degrees of Freedom: 16 
SSR: 104.05843 MSE: 6.503652 Root MSE: 2.550226 
Multiple R-Squared: 0.611888 Adjusted R-Squared: 0.539117 
\end{Soutput}
\end{Schunk}

First, the estimation method is reported
and a few summary statistics for each equation are given.
Then, some results regarding the whole equation system are printed:
covariance matrix and correlations of the residuals,
log of the determinant of the residual covariance matrix,
$R^2$ value of the whole system, and
McElroy's $R^2$ value.
If the model is estimated by (W)SUR or (W)3SLS,
the covariance matrix used for estimation is printed additionally.
Finally, the estimation results of each equation are reported:
formula of the estimated equation,
estimated parameters, their standard errors, t-values, p-values
and codes indicating their statistical significance,
and some other statistics like
standard error of the residuals or
$R^2$ value of the equation.


%%%%%%%%%%%%%%%%%%%%%%%%%%%%%%%%%%%%%%%%%%%%%%%%%%
\subsection[User options of systemfit]{User options of \code{systemfit}}
\label{sec:user-options}

The user can modify the default estimation method by providing
additional optional arguments,
e.g.\ to specify instrumental variables
or to impose parameter restrictions.
All optional arguments are described in the following:

%%%%%%%%%%%%%%%%%%%%%%%%%%%%%%%%%%%%%%%%%%%%%%%%%%
\subsubsection{Equation labels}
The optional argument \code{eqnlabels} allows the user to label the equations.
It has to be a vector of strings.
If this argument is not provided, the labels are taken from the names of
the equations in argument \code{eqns}.
And if the equations have no names, they are numbered consecutively.
Hence, the following command has the same effect as the command above.
\begin{Schunk}
\begin{Sinput}
> fitsur <- systemfit("SUR", list(eqDemand, eqSupply), eqnlabels = c("demand", 
+     "supply"))
\end{Sinput}
\end{Schunk}
 
%%%%%%%%%%%%%%%%%%%%%%%%%%%%%%%%%%%%%%%%%%%%%%%%%%
\subsubsection{Instrumental variables}
The instruments for a 2SLS, W2SLS or 3SLS estimation can be specified
by the argument \code{inst}.
If the same instruments should be used for all equations,
\code{inst} must be a one-sided formula.
If different instruments should be used for the equations,
\code{inst} must be a list that contains a one-sided formula
for each equation.
The first example uses the same instruments for all equations,
and the second uses different instruments:
\begin{Schunk}
\begin{Sinput}
> fit3sls <- systemfit("3SLS", list(demand = eqDemand, supply = eqSupply), 
+     inst = ~income + farmPrice + trend)
> fit3sls2 <- systemfit("3SLS", list(demand = eqDemand, supply = eqSupply), 
+     inst = list(~farmPrice + trend, ~income + farmPrice + trend))
\end{Sinput}
\end{Schunk}

%%%%%%%%%%%%%%%%%%%%%%%%%%%%%%%%%%%%%%%%%%%%%%%%%%
\subsubsection{Data}
Having all data in the global environment or attached to the search path
is often inconvenient.
Therefore, a data frame \code{data} can be provided
that contains the variables of the model.
In the following example we do not need to attach the data frame
\code{Kmenta} before calling systemfit:
\begin{Schunk}
\begin{Sinput}
> fitsur <- systemfit("SUR", list(eqDemand, eqSupply), data = Kmenta)
\end{Sinput}
\end{Schunk}


%%%%%%%%%%%%%%%%%%%%%%%%%%%%%%%%%%%%%%%%%%%%%%%%%%
\subsubsection{Parameter restrictions}
As outlined in section~\ref{sec:Restrictions}, parameter restrictions
can be imposed in two ways.
The first way is to use the transformation matrix $T$
that can be specified by the argument \code{TX}.
If we want to impose the restriction, say $\beta_2 = - \beta_6$,
we can do this as follows
\begin{Schunk}
\begin{Sinput}
> tx <- matrix(0, nrow = 7, ncol = 6)
> tx[1, 1] <- 1
> tx[2, 2] <- 1
> tx[3, 3] <- 1
> tx[4, 4] <- 1
> tx[5, 5] <- 1
> tx[6, 2] <- -1
> tx[7, 6] <- 1
> fitsurTx <- systemfit("SUR", list(eqDemand, eqSupply), TX = tx)
\end{Sinput}
\end{Schunk}
The first line creates a $7 \times 6$ matrix of zeros,
where 7 is the number of unrestricted coefficients and
6 is the number of restricted coefficients.
The following seven lines specify this matrix in a way
that the unrestricted coefficients $( \beta )$ are assigned
to the restricted coefficients $( \beta^* )$ with
$\beta_1 = \beta^*_1$,
$\beta_2 = \beta^*_2$,
$\beta_3 = \beta^*_3$,
$\beta_4 = \beta^*_4$,
$\beta_5 = \beta^*_5$,
$\beta_6 = -\beta^*_2$, and
$\beta_7 = \beta^*_6$.
Finally the model is estimated with restriction
$\beta^*_2 = \beta_2 = - \beta_6$ imposed.

The second way to impose parameter restrictions is to use
the matrix $R$ and the vector $q$
(see section~\ref{sec:Restrictions}).
Matrix $R$ can be specified with the argument \code{R.restr}
and vector $q$ with argument \code{q.restr}.
We convert the restriction specified above to $\beta_2 + \beta_6 = 0$
and impose it in the second way:
\begin{Schunk}
\begin{Sinput}
> Rmat <- matrix(0, nrow = 1, ncol = 7)
> Rmat[1, 2] <- 1
> Rmat[1, 6] <- 1
> qvec <- c(0)
> fitsurRmat <- systemfit("SUR", list(eqDemand, eqSupply), R.restr = Rmat, 
+     q.restr = qvec)
\end{Sinput}
\end{Schunk}
\label{code:Rmat}
The first line creates a $1 \times 7$ matrix of zeros,
where 1 is the number of restrictions and
7 is the number of unrestricted coefficients.
The following two lines specify this matrix in a way
that the multiplication with the parameter vector
results in $ \beta_2 + \beta_6 $.
The fourth line creates a vector with a single element
that contains the left hand side of the restriction, i.e.\ zero.
Finally the model is estimated with restriction
$\beta_2 + \beta_6 = 0$ imposed.


%%%%%%%%%%%%%%%%%%%%%%%%%%%%%%%%%%%%%%%%%%%%%%%%%%
\subsubsection{Iteration control}
The estimation methods WLS, SUR, W2SLS and 3SLS need a
covariance matrix of the residuals
that can be calculated from a first-step OLS or 2SLS estimation
(see section~\ref{sec:residcov}).
If the argument \code{maxiter} is set to a number large than one,
this procedure is iterated and at each iteration the covariance
matrix is calculated from the previous step estimation.
This iteration is repeated until the maximum number of iterations
is reached or the parameter estimates have converged.
The maximum number of iterations is specified by the argument \code{maxiter}.
Its default value is one, which means no iteration.
The convergence criterion is
\begin{equation}
   \sqrt{ \frac{ \sum_i (b_{i,g} - b_{i,g-1})^2 }{ \sum_i b_{i,g-1}^2 }}
      < \texttt{tol}
\end{equation}
where $b_{i,g}$ is the $i$th coefficient of the $g$th iteration.
The default value of \code{tol} is $10^{-5}$.


%%%%%%%%%%%%%%%%%%%%%%%%%%%%%%%%%%%%%%%%%%%%%%%%%%
\subsubsection{Residual covariance matrix}
It was explained in section~\ref{sec:residcov} that several different
formulas have been proposed to calculate the residual
covariance matrix.
The user can specify, which formula \pkg{systemfit} should use.
Possible values of the argument \code{rcovformula} are presented in
table~\ref{tab:rcovformula}.
By default, \pkg{systemfit} uses equation (\ref{eq:rcov1}).

\begin{table}[H]
\caption{Possible values of argument \code{rcovformula}}
\label{tab:rcovformula}
\centering
\begin{tabular}{|c|c|}
\hline
argument & equation \\
\code{rcovformula} & \\
\hline
0              & \ref{eq:rcov0} \\
\hline
1 or 'geomean' & \ref{eq:rcov1} \\
\hline
2 or 'Theil'   & \ref{eq:rcov2} \\
\hline
3 or 'max'     & \ref{eq:rcov3} \\
\hline
\end{tabular}
\end{table}


%%%%%%%%%%%%%%%%%%%%%%%%%%%%%%%%%%%%%%%%%%%%%%%%%%
\subsubsection{3SLS formula}
As discussed in sections~\ref{sec:Estimation} and~\ref{sec:Restrictions},
there exist several different
formulas to perform a 3SLS estimation.
The user can specify the applied formula by the argument \code{formula3sls}.
Possible values are presented in table~\ref{tab:formula3sls}.
The default value is 'GLS'.

\begin{table}[H]
\caption{Possible values of argument \code{formula3sls}}
\label{tab:formula3sls}
\centering
\begin{tabular}{|c|c|c|}
\hline
argument           & equation       & equation \\
\code{formula3sls} & (unrestricted) & (restricted) \\
\hline
'GLS'     & \ref{eq:3slsGls}     & \ref{eq:3slsGlsR} \\
\hline
'IV'      & \ref{eq:3slsIv}      & \ref{eq:3slsIvR} \\
\hline
'GMM'     & \ref{eq:3slsGmm}     & \ref{eq:3slsGmmR} \\
\hline
'Schmidt' & \ref{eq:3slsSchmidt} & \ref{eq:3slsSchmidtR} \\
\hline
'EViews'  & \ref{eq:3slsEViews}  & \ref{eq:3slsEViewsR} \\
\hline
\end{tabular}
\end{table}


%%%%%%%%%%%%%%%%%%%%%%%%%%%%%%%%%%%%%%%%%%%%%%%%%%
\subsubsection{Degrees of freedom for t-tests}
There exist two different approaches to determine the degrees of freedom
to perform t-tests on the estimated parameters
(section~\ref{sec:degreesOfFreedom}).
This can be specified with argument \code{probdfsys}.
If it is \code{TRUE},
the degrees of freedom of the whole system are taken.
In contrast, if \code{probdfsys} is \code{FALSE},
the degrees of freedom of the single equation are taken.
By default, \code{probdfsys} is \code{TRUE}, if any restrictions
are specified using either the argument \code{R.restr} or the
argument \code{TX}, and it is \code{FALSE} otherwise.


%%%%%%%%%%%%%%%%%%%%%%%%%%%%%%%%%%%%%%%%%%%%%%%%%%
\subsubsection{Sigma squared}
In case of OLS or 2SLS estimations, argument \code{single.eq.sigma}
can be used to specify,
whether different $\sigma^2$s for each single equation
or the same $\sigma^2$ for all equations should be used.
If argument \code{single.eq.sigma} is \code{TRUE},
equations (\ref{eq:olsCovSingleSigma})
and (\ref{eq:2slsCovSingleSigma}) are applied.
In contrast, if argument \code{single.eq.sigma} is \code{FALSE},
equations (\ref{eq:olsCovSameSigma})
and (\ref{eq:2slsCovSameSigma}) are applied.
By default, \code{single.eq.sigma} is \code{FALSE}, if any restrictions
are specified using either the argument \code{R.restr} or the
argument \code{TX}, and it is \code{TRUE} otherwise.



%%%%%%%%%%%%%%%%%%%%%%%%%%%%%%%%%%%%%%%%%%%%%%%%%%
\subsubsection{System options}
Finally, two options regarding some internal calculations are available.
First, argument \code{solvetol} specifies the tolerance level
for detecting linear dependencies when inverting a matrix or
calculating a determinant (using functions \code{solve} and \code{det}).
The default value depends on the used computer system and is equal to the
default tolerance level of \code{solve} and \code{det}.
Second, argument \code{saveMemory} can be used in case of large data sets
to accelerate the estimation by omitting some calculation
that are not crucial for the basic estimation.
Currently, only the calculation of McElroy's $R^2$ is omitted.
The default value of argument \code{saveMemory} is \code{TRUE},
if the argument \code{data} is specified and the number of observations
times the number of equations is larger than 1000,
and it is \code{FALSE} otherwise.


%%%%%%%%%%%%%%%%%%%%%%%%%%%%%%%%%%%%%%%%%%%%%%%%%%
\subsection[Wrapper function systemfitClassic]{The wrapper function
   \code{systemfitClassic}}

The wrapper function \code{systemfitClassic}
can be applied to (classical) panel-like data in long format
if the regressors are the same for all equations.
This function is called by
\begin{Schunk}
\begin{Sinput}
> systemfitClassic(method, formula, eqnVar, timeVar, data)
\end{Sinput}
\end{Schunk}

The mandatory arguments are \code{method}, \code{formula}, \code{eqnVar},
and \code{timeVar}.
Argument \code{method} is the same as in \code{systemfit}
(see section~\ref{sec:standard-usage}).
The second argument \code{formula} is a standard formula in \proglang{R}
that will be applied to all equations.
Argument \code{eqnVar} specifies the variable name indicating
the equation to which the observation belongs,
and argument \code{timeVar} specifies the variable name indicating
the time.
Finally, \code{data} is a \code{data.frame} that contains all required data.

We demonstrate the usage of \code{systemfitClassic}
using an example taken from \citet[pp.\ 295, 300]{theil71}
that is based on \citet{grunfeld58}.
We want to estimate a model for gross investment of 2 US firms
in the years 1935--1954:
\begin{equation}
\texttt{invest}_{it} = \beta_1 + \beta_2 * \texttt{value}_{it} +
   \beta_3 * \texttt{capital}_{it}
\end{equation}
where \code{invest} is the gross investment of firm $i$ in year $t$,
\code{value} is the market value of the firm at the end of the
previous year, and
\code{capital} is the capital stock of the firm at the end of the
previous year.

This model can be estimated by
\begin{Schunk}
\begin{Sinput}
> data("GrunfeldTheil")
> theilSur <- systemfitClassic("SUR", invest ~ value + capital, 
+     "firm", "year", data = GrunfeldTheil)
\end{Sinput}
\end{Schunk}
The first line loads example data that are included with the package.
And then, a seemingly unrelated regression is performed,
where the variable �firm� indicates the firm
and the variable �year� indicates the time.

The function \code{systemfitClassic} has also an optional argument
\code{pooled} that is a logical variable indicating
whether the coefficients should be restricted to be equal in all equations.
By default, this argument is set to �\code{FALSE}�.
In addition all optional arguments of \code{systemfit}
(see section~\ref{sec:user-options})
except for \code{eqnLabels} and \code{TX}
can be used with \code{systemfitClassic}, too.


%%%%%%%%%%%%%%%%%%%%%%%%%%%%%%%%%%%%%%%%%%%%%%%%%%
\subsection{Testing linear restrictions}

As described in section~\ref{sec:testingRestrictions},
linear restrictions can be tested by an F test, Wald test or
LR test.
The corresponding functions in package \pkg{systemfit} are
\code{ftest.systemfit}, \code{waldtest.systemfit}, and
\code{lrtest.systemfit}, respectively.

We will now test the restriction $\beta_2 = -\beta_6$
that was specified by the matrix \code{Rmat} and the vector \code{qvec}
in the example above (p.~\pageref{code:Rmat}).
\begin{Schunk}
\begin{Sinput}
> ftest.systemfit(fitsur, Rmat, qvec)
\end{Sinput}
\begin{Soutput}
F-test for linear parameter restrictions in equation systems
F-statistic: 0.9322 
degrees of freedom of the numerator: 1 
degrees of freedom of the denominator: 33 
p-value: 0.3413 
\end{Soutput}
\begin{Sinput}
> waldtest.systemfit(fitsur, Rmat, qvec)
\end{Sinput}
\begin{Soutput}
Wald-test for linear parameter restrictions in equation systems
Wald-statistic: 0.6092 
degrees of freedom: 1 
p-value: 0.4351 
\end{Soutput}
\begin{Sinput}
> lrtest.systemfit(fitsurRmat, fitsur)
\end{Sinput}
\begin{Soutput}
Likelihood-Ratio-test for parameter restrictions in equation systems
LR-statistic: 1.004 
degrees of freedom: 1 
p-value: 0.3163 
\end{Soutput}
\end{Schunk}
The linear restrictions are tested by an F test first,
then by a Wald test, and finally by an LR test.
The functions \code{ftest.systemfit} and \code{waldtest.systemfit}
have three arguments.
The first argument must be an unrestricted regression
returned by \code{systemfit}.
The second and third argument are the restriction matrix $R$ and
the vector $q$ as described in section~\ref{sec:Restrictions}.
In contrast, the function \code{lrtest.systemfit} needs two arguments.
The first argument must be a restricted and the second an unrestricted
regression returned by \code{systemfit}.

All tests print a short explanation first.
Then the empirical test statistic and the degree(s) of freedom are reported.
Finally the p-value is printed.
While there is some variation of the p-values across
the three different tests,
all tests suggest the same decision:
The null hypothesis $\beta_2 = -\beta_6$ cannot be rejected at any
reasonable level of significance.


%%%%%%%%%%%%%%%%%%%%%%%%%%%%%%%%%%%%%%%%%%%%%%%%%%
\subsection{Hausman test}
A Hausman test, which is described in section~\ref{sec:hausman},
can be carried out with following commands:
\begin{Schunk}
\begin{Sinput}
> fit2sls <- systemfit("2SLS", list(demand = eqDemand, supply = eqSupply), 
+     inst = ~income + farmPrice + trend, data = Kmenta)
> fit3sls <- systemfit("3SLS", list(demand = eqDemand, supply = eqSupply), 
+     inst = ~income + farmPrice + trend, data = Kmenta)
> hausman.systemfit(fit2sls, fit3sls)
\end{Sinput}
\begin{Soutput}
	Hausman specification test for consistency of the 3SLS estimation

data:  Kmenta 
Hausman = 2.5357, df = 7, p-value = 0.9244
\end{Soutput}
\end{Schunk}
First of all, the model is estimated by 2SLS and then by 3SLS.
Finally, in the last line
the test is carried out by the command \code{hausman.systemfit}.
This function requires two arguments:
the result of a 2SLS estimation and
the result of a 3SLS estimation.
The Hausman test statistic is
2.536,
which has a $\chi^2$ distribution
with 7 degrees of freedom
under the null hypothesis.
The corresponding p-value is
0.924.
This shows that the null hypothesis is not rejected
at any reasonable level of significance.
Hence, we can assume that the 3SLS estimator is consistent.
