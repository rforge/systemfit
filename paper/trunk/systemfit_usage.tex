%%%%%%%%%%%%%%%%%%%%%%%%%%%%%%%%%%%%%%%%%%%%%%%%%
\section{Using systemfit}\label{sec:Usage}
%%%%%%%%%%%%%%%%%%%%%%%%%%%%%%%%%%%%%%%%%%%%%%%%%%


%%%%%%%%%%%%%%%%%%%%%%%%%%%%%%%%%%%%%%%%%%%%%%%%%%
\subsection{Standard usage}

\pkg{systemfit} is generally called by

\code{
R> systemfit( method, eqns )
}

There are two mandatory arguments: \code{method} and \code{eqns}.

The argument \code{method} is a string determining the estimation method.
It must be one of "OLS", "WLS", "SUR", "2SLS", "W2SLS" or "3SLS".

The other mandatory argument \code{eqns} is a list of the equations 
to estimate. 
Each equation is a standard formula in \proglang{R}.
It starts with a dependent variable on the left hand side.
After a tilde ($\sim$) the regressors are listed%
\footnote{For Details see the \proglang{R} help files to \code{formula}}.

This is now demonstrated using an example: \\
\code{
R> library( systemfit ) \\
R> data( kmenta ) \\
R> attach( kmenta ) \\
R> fitsur <- systemfit( "SUR", list( q $\sim$ p + d, q $\sim$ p + f + a ) ) \\
}

The first line loads the \pkg{systemfit} package. 
The second line loads example data that are included in this package.
These data come from \cite{kmenta86}.
They are attached to the \proglang{R} search path in line three.
In the last line a seemingly unrelated regression is done.
The first equation represents the demand side of the food market.
The dependant variable is \code{q} (food consumption per capita). 
The regressors are \code{p} (ratio of food prices to general consumer prices)
and \code{d} (disposable income) as well as a constant%
\footnote{a regression constant is always implied if not explicitly omitted.}.
The second equation represents the supply side.
Variable \code{q} (food consumption per capita) is also the dependant 
variable of this equation. 
The regressors are again \code{p} (ratio of food prices to general 
consumer prices) and a constant as well as 
\code{f} (ratio of preceding year's prices received by farmers) and 
\code{a} (a time trend in years).
The regression result is assigned to the variable \code{fitsur}.

Summary results can be printed by\\
\code{
R>~summary(~fitsur~)~\\
~\\
systemfit~results~\\
method:~SUR~\\
\\
\mbox{}~~~N~DF~~~~~~SSR~~~~~MSE~~~~RMSE~~~~~~~R2~~~Adj~R2 \\
1~20~17~~65.6829~3.86370~1.96563~0.755019~0.726198~\\
2~20~16~104.0584~6.50365~2.55023~0.611888~0.539117~\\
~\\
The~covariance~matrix~of~the~residuals~used~for~estimation\\
\mbox{}~~~~~~~~1~~~~~~~2~\\
1~3.72539~4.13696~\\
2~4.13696~5.78444~\\
~\\
The~covariance~matrix~of~the~residuals\\
\mbox{}~~~~~~~~1~~~~~~~2~\\
1~3.86370~4.92431~\\
2~4.92431~6.50365~\\
~\\
The~correlations~of~the~residuals\\
\mbox{}~~~~~~~~~1~~~~~~~~2~\\
1~1.000000~0.982348~\\
2~0.982348~1.000000~\\
~\\
The~determinant~of~the~residual~covariance~matrix:~0.879285~\\
OLS~R-squared~value~of~the~system:~0.683453~\\
McElroy's~R-squared~value~for~the~system:~0.788722~\\
~\\
SUR~estimates~for~1~~(equation~1~)~\\
Model~Formula:~q~~~p~+~d\\
\\
\mbox{}~~~~~~~~~~~~~Estimate~Std.~Error~~~t~value~Pr(>|t|)~\\
(Intercept)~99.332894~~~7.514452~13.218913~~~~~~~~0~***~\\
p~~~~~~~~~~~-0.275486~~~0.088509~-3.112513~0.006332~~**~\\
d~~~~~~~~~~~~~0.29855~~~0.041945~~7.117605~~~~2e-06~***~\\
---~\\
Signif.~codes:~~0~`***'~0.001~`**'~0.01~`*'~0.05~`.'~0.1~`~'~1~\\
~\\
Residual~standard~error:~1.96563~on~17~degrees~of~freedom~\\
Number~of~observations:~20~Degrees~of~Freedom:~17~\\
SSR:~65.682902~MSE:~3.8637~Root~MSE:~1.96563~\\
Multiple~R-Squared:~0.755019~Adjusted~R-Squared:~0.726198~\\
~\\
~\\
SUR~estimates~for~2~~(equation~2~)~\\
Model~Formula:~q~~~p~+~f~+~a\\
\\
\mbox{}~~~~~~~~~~~~~Estimate~Std.~Error~~t~value~Pr(>|t|)~\\
(Intercept)~61.966166~~~11.08079~5.592215~~~~4e-05~***~\\
p~~~~~~~~~~~~0.146884~~~0.094435~1.555397~0.139408~\\
f~~~~~~~~~~~~0.214004~~~0.039868~5.367761~~6.3e-05~***~\\
a~~~~~~~~~~~~0.339304~~~0.067911~4.996283~0.000132~***~\\
---~\\
Signif.~codes:~~0~`***'~0.001~`**'~0.01~`*'~0.05~`.'~0.1~`~'~1~\\
~\\
Residual~standard~error:~2.550226~on~16~degrees~of~freedom~\\
Number~of~observations:~20~Degrees~of~Freedom:~16~\\
SSR:~104.05843~MSE:~6.503652~Root~MSE:~2.550226~\\
Multiple~R-Squared:~0.611888~Adjusted~R-Squared:~0.539117~\\
}



%%%%%%%%%%%%%%%%%%%%%%%%%%%%%%%%%%%%%%%%%%%%%%%%%%
\subsection{User options}

Following additional options can be set by the user:

%%%%%%%%%%%%%%%%%%%%%%%%%%%%%%%%%%%%%%%%%%%%%%%%%%
% \subsubsection{Equation labels}
\paragraph{Equation labels}
The optional argument \code{eqnlabels} allows the user to label the equations.
It has to be a vector of strings naming the equations.\\
\code{
R>~fitsur~<-~systemfit(~"SUR",~list(~q~~~p~+~d,~q~~~p~+~f~+~a~),\\~
R+~~~~eqnlabels~=~c(~"demand",~"supply"~)~)\\
R>~summary(~fitsur~)\\
systemfit~results\\
method:~SUR\\
\\
\mbox{}~~~~~~~~N~DF~~~~~~SSR~~~~~MSE~~~~RMSE~~~~~~~R2~~~Adj~R2\\
demand~20~17~~65.6829~3.86370~1.96563~0.755019~0.726198\\
supply~20~16~104.0584~6.50365~2.55023~0.611888~0.539117\\
\ldots\\
}
If no equation labels are provided, the equations are numbered.
 
%%%%%%%%%%%%%%%%%%%%%%%%%%%%%%%%%%%%%%%%%%%%%%%%%%
%\subsubsection{Instrumental variables}   
\paragraph{Instrumental variables}   
\code{inst}one-sided model formula specifying instrumental variables
   or a list of one-sided model formulas if different instruments should
   be used for the different equations (only needed for 2SLS, W2SLS and
   3SLS estimations).

%%%%%%%%%%%%%%%%%%%%%%%%%%%%%%%%%%%%%%%%%%%%%%%%%%
%\subsubsection{Data}   
\paragraph{Data}   
\code{data} an optional data frame containing the variables in the model.
   By default the variables are taken from the environment from which
   systemfit is called.

%%%%%%%%%%%%%%%%%%%%%%%%%%%%%%%%%%%%%%%%%%%%%%%%%%
%\subsubsection{Restrictions}   
\paragraph{Restrictions}   
\code{R.restr} an optional j x k matrix to impose linear
   restrictions on the parameters by \code{R.restr} * $\beta$ = \code{q.restr}
   (j = number of restrictions, k = number of all parameters,
   $\beta$ = vector of all parameters).

\code{q.restr} an optional j x 1 matrix to impose linear
   restrictions (see \code{R.restr}); default is a j x 1 matrix
   that contains only zeros.

\code{TX} an optional matrix to transform the regressor matrix and,
   hence, also the coefficient vector (see details).

%%%%%%%%%%%%%%%%%%%%%%%%%%%%%%%%%%%%%%%%%%%%%%%%%%
%\subsubsection{Iteration control}   
\paragraph{Iteration control}
\code{maxiter} maximum number of iterations for WLS, SUR, W2SLS and
   3SLS estimations.

\code{tol} tolerance level indicating when to stop the iteration (only
   WLS, SUR, W2SLS and 3SLS estimations).

%%%%%%%%%%%%%%%%%%%%%%%%%%%%%%%%%%%%%%%%%%%%%%%%%%
%\subsubsection{Residual covariance matrix}   
\paragraph{Residual covariance matrix}   
\code{rcovformula} formula to calculate the estimated residual covariance
   matrix (see details).

%%%%%%%%%%%%%%%%%%%%%%%%%%%%%%%%%%%%%%%%%%%%%%%%%%
%\subsubsection{3SLS formula}   
\paragraph{3SLS formula}   
\code{formula3sls} formula for calculating the 3SLS estimator,
   one of "GLS", "IV", "GMM", "Schmidt" or "EViews" (see details).

%%%%%%%%%%%%%%%%%%%%%%%%%%%%%%%%%%%%%%%%%%%%%%%%%%
%\subsubsection{Degrees of freedom for t-tests}   
\paragraph{Degrees of freedom for t-tests}   
\code{probdfsys} use the degrees of freedom of the whole system
   (in place of the degrees of freedom of the single equation)
   to calculate prob values for the t-test of individual parameters.

%%%%%%%%%%%%%%%%%%%%%%%%%%%%%%%%%%%%%%%%%%%%%%%%%%
%\subsubsection{Sigma squared}   
\paragraph{Sigma squared}   
\code{single.eq.sigma} use different $\sigma^2$s for each
   single equation to calculate the covariance matrix and the
   standard errors of the coefficients (only OLS and 2SLS).

%%%%%%%%%%%%%%%%%%%%%%%%%%%%%%%%%%%%%%%%%%%%%%%%%%
%\subsubsection{System options}   
\paragraph{System options}   

\code{solvetol} tolerance level for detecting linear dependencies
   when inverting a matrix or calculating a determinant (see
   \code{solve} and \code{det}).

\code{saveMemory} save memory by omitting some calculation that
   are not crucial for the basic estimation (e.g McElroy's
   $R^2$).


