%%%%%%%%%%%%%%%%%%%%%%%%%%%%%%%%%%%%%%%%%%%%%%%%%%%%%%
\subsection{Estimation under linear restrictions on the coefficients}
\label{sec:Restrictions}
%%%%%%%%%%%%%%%%%%%%%%%%%%%%%%%%%%%%%%%%%%%%%%%%%%%%%%

In many empirical applications,
it is desirable to estimate the coefficients under linear restrictions.
For instance, in econometric demand and production analysis,
it is common to estimate the coefficients
under homogeneity and symmetry restrictions
that are derived from the underlying theoretical model.

There are two different methods to estimate the coefficients
under linear restrictions.
First, a matrix $M$ can be specified that
\begin{equation}
   \beta = M \cdot \beta^M \label{eq:T-restr} ,
\end{equation}
where $\beta^M$ is a vector of restricted (linear independent) coefficients,
and $M$ is a matrix with the number of rows equal to the number of
unrestricted coefficients ($\beta$) and
the number of columns equal to the number of restricted coefficients
($\beta^M$).
$M$ can be used to map each unrestricted coefficient to one or more
restricted coefficients.

The second method to estimate the coefficients under linear restrictions
constrains the coefficients by
\begin{equation}
   R \beta^R = q ,
   \label{eq:restr-R}
\end{equation}
where $\beta^R$ is the vector of the restricted coefficients,
and $R$ and $q$ are a matrix and vector, respectively,
that specify the restrictions \citep[see][p.~100]{greene03}.
Each linear independent restriction is represented by one row of $R$
and the corresponding element of~$q$.

The first method is less flexible than the second%
\footnote{
While restrictions like $\beta_1 = 2 \beta_2$ can be specified by
both methods,
restrictions like $\beta_1 + \beta_2 = 4$ can be specified only
by the second method.
}, 
but is preferable if the coefficients are estimated
under many equality constraints across different equations of the system.
Of course, these restrictions can be also specified using
the latter method.
However, while the latter method increases the dimension of the 
matrices to be inverted during estimation, the first reduces it. 
Thus, in some cases the latter way leads to estimation problems
(e.g.\ (near) singularity of the matrices to be inverted),
while the first does not.

These two methods can be combined.
In this case, the restrictions specified using the latter method are
imposed on the linear independent coefficients
that are restricted by the first method, so that
\begin{equation}
   R \beta^{MR} = q ,
\end{equation}
where $\beta^{MR}$ is the vector of the restricted $\beta^M$ coefficients.


\subsubsection{Calculation of restricted estimators}

If the first method (Equation~\ref{eq:T-restr}) is chosen
to estimate the coefficients under these restrictions,
the matrix of regressors $X$ is (post-)\hspace{0pt}multiplied
by the $M$ matrix,
so that
\begin{equation}
    X^M = X \cdot M .
\end{equation}
Then, $X^M$ is substituted for $X$ and a standard estimation as described
in the previous section is done
(Equations~\ref{eq:ols-wls-sur}--\ref{eq:3slsEViews}).
This results in the linear independent coefficient estimates $\bHat^M$ and
their covariance matrix.
The original coefficients can be obtained by Equation~\ref{eq:T-restr}
and the estimated covariance matrix of the original coefficients
can be obtained by
\begin{equation}
   \COVHat \left[ \bHat \right]
   = M \cdot \COVHat \left[ \bHat^M \right] \cdot M^\top .
\end{equation}

The implementation of the second method to estimate the coefficients
under linear restrictions (Equation~\ref{eq:restr-R})
is described for each estimation method in the following sections.


%%%%%%%%%%%%%%%%%%%%%%%%%%%%%%%%%%%%%%%%%%%%%%%%%%
\subsubsection{Restricted OLS, WLS, and SUR estimation}

The OLS, WLS, and SUR estimators restricted by $R \beta^R = q$ can be obtained by
\begin{equation}
   \left[ \begin{array}{c}
      \bHat^R \\[0.2em] \lHat
   \end{array} \right]
   =
   \left[ \begin{array}{cc}
      X^\top \OHat^{-1} X & R^\top \\
      R & 0
   \end{array} \right]^{-1}
   \cdot
   \left[ \begin{array}{c}
      X^\top \OHat^{-1} y \\ q
   \end{array} \right] ,
   \label{eq:ols-wls-sur-r}
\end{equation}
where $\lambda$ is a vector of the Lagrangean multipliers of the restrictions
and $\OHat$ is defined as in section~\ref{sec:Estimation-ols-wls-sur}.
An estimator of the covariance matrix of the estimated coefficients is
\begin{equation}
   \COVHat
   \left[ \begin{array}{c}
      \bHat^R \\[0.2em] \lHat
   \end{array} \right] 
   = 
   \left[ \begin{array}{cc}
      X^\top \OHat^{-1} X & R^\top \\
      R & 0
   \end{array} \right]^{-1} .
   \label{eq:cov-ols-wls-sur-r}
\end{equation}

%%%%%%%%%%%%%%%%%%%%%%%%%%%%%%%%%%%%%%%%%%%%%%%%%%
\subsubsection{Restricted 2SLS, W2SLS, and 3SLS estimation}

The 2SLS, W2SLS, and standard 3SLS estimators
restricted by $R \beta^R = q$ can be obtained by
\begin{equation}
   \left[ \begin{array}{c}
      \bHat^R \\[0.2em] \lHat
   \end{array} \right]
   =
   \left[ \begin{array}{cc}
      \XHat^\top \OHat^{-1} \XHat & R^\top \\
      R & 0
   \end{array} \right]^{-1}
   \cdot
   \left[ \begin{array}{c}
      \XHat^\top \OHat^{-1} y \\ q
   \end{array} \right] ,
   \label{eq:2sls-w2sls-3sls-r}
\end{equation}
where $\OHat$ is defined as in section~\ref{sec:Estimation-2sls-w2sls-3sls}.
An estimator of the covariance matrix of the estimated coefficients is
\begin{equation}
   \COVHat
   \left[ \begin{array}{c}
      \bHat^R \\[0.2em] \lHat
   \end{array} \right] 
   = 
   \left[ \begin{array}{cc}
      \XHat^\top \OHat^{-1} \XHat & R^\top \\
      R & 0
   \end{array} \right]^{-1} .
   \label{eq:cov-2sls-w2sls-3sls-r}
\end{equation}

The 3SLS-IV estimator restricted by $R \beta^R = q$ can be obtained by
\begin{equation}
   \left[ \begin{array}{c}
      \bHat^R_{3SLS-IV} \\[0.2em] \lHat
   \end{array} \right]
   =
   \left[ \begin{array}{cc}
      \XHat^\top \OHat^{-1} X & R^\top \\
      R & 0
   \end{array} \right]^{-1}
   \cdot
   \left[ \begin{array}{c}
      \XHat^\top \OHat^{-1} y \\ q
   \end{array} \right] ,
   \label{eq:3slsIvR}
\end{equation}
where
\begin{equation}
   \COVHat
   \left[ \begin{array}{c}
      \bHat^R_{3SLS-IV} \\[0.2em] \lHat
   \end{array} \right] 
   = 
   \left[ \begin{array}{cc}
      \XHat^\top \OHat^{-1} X & R^\top \\
      R & 0
   \end{array} \right]^{-1} .
\end{equation}
The restricted 3SLS-GMM estimator can be obtained by
\begin{equation}
   \left[ \begin{array}{c}
      \bHat^R_{3SLS-GMM} \\[0.2em] \lHat
   \end{array} \right]
   =
   \left[ \begin{array}{cc}
      X^\top Z \left( Z^\top \OHat Z \right)^{-1} Z^\top X & R^\top \\
      R & 0
   \end{array} \right]^{-1}
   \cdot
   \left[ \begin{array}{c}
      X^\top Z \left( Z \OHat Z \right)^{-1} Z^\top y \\ q
   \end{array} \right] ,
   \label{eq:3slsGmmR}
\end{equation}
where
\begin{equation}
   \COVHat
   \left[ \begin{array}{c}
      \bHat^R_{3SLS-GMM} \\[0.2em] \lHat
   \end{array} \right] 
   = 
   \left[ \begin{array}{cc}
      X^\top Z \left( Z^\top \OHat Z \right)^{-1} Z^\top X & R^\top \\
      R & 0
   \end{array} \right]^{-1} .
\end{equation}
The restricted 3SLS estimator based on the suggestion of
\cite{schmidt90} is
\begin{equation}
   \left[ \begin{array}{c}
      \bHat^R_{3SLS-Schmidt} \\[0.2em] \lHat
   \end{array} \right]
   =
   \left[ \begin{array}{cc}
      \XHat^\top \OHat^{-1} \XHat & R^\top \\
      R & 0
   \end{array} \right]^{-1}
   \cdot
   \left[ \begin{array}{c}
      \XHat^\top \OHat^{-1} Z \left( Z^\top Z \right)^{-1} Z^\top y \\ q
   \end{array} \right] ,
   \label{eq:3slsSchmidtR}
\end{equation}
where
\begin{eqnarray}
   \COVHat
   \left[ \begin{array}{c}
      \bHat^R_{3SLS-Schmidt} \\[0.2em] \lHat
   \end{array} \right] 
   & = & 
   \left[ \begin{array}{cc}
      \XHat^\top \OHat^{-1} \XHat & R^\top \\
      R & 0
   \end{array} \right]^{-1}
   \\
   & & \cdot
   \left[ \begin{array}{cc}
      \XHat^\top \OHat^{-1} Z \left( Z^\top Z \right)^{-1} Z^\top \OHat
      Z \left( Z^\top Z \right)^{-1} Z^\top \OHat^{-1} \XHat & 0^\top \\
      0 & 0
   \end{array} \right]^{-1}
   \nonumber \\
   & & \cdot
   \left[ \begin{array}{cc}
      \XHat^\top \OHat^{-1} \XHat & R^\top \\
      R & 0
   \end{array} \right]^{-1} .
   \nonumber
\end{eqnarray}
The econometrics software \proglang{EViews} calculates the restricted 3SLS estimator by
\begin{equation}
   \left[ \begin{array}{c}
      \bHat^R_{3SLS-EViews} \\[0.2em] \lHat
   \end{array} \right]
   =
   \left[ \begin{array}{cc}
      \XHat^\top \OHat^{-1} \XHat & R^\top \\
      R & 0
   \end{array} \right]^{-1}
   \cdot
   \left[ \begin{array}{c}
      \XHat^\top \OHat^{-1} \left( y - X \bHat^R_{2SLS} \right)
      \\ q 
   \end{array} \right] ,
   \label{eq:3slsEViewsR}
\end{equation}
where $\bHat^R_{2SLS}$ is the restricted 2SLS estimator calculated
by Equation~\ref{eq:2sls-w2sls-3sls-r}.
\proglang{EViews} uses the standard formula of the restricted 3SLS
estimator (Equation~\ref{eq:cov-2sls-w2sls-3sls-r}) to calculate an estimator
for the covariance matrix of the estimated coefficients.


If the same instrumental variables are used in all equations 
($Z_1 = Z_2 = \ldots = Z_G$),
all the above mentioned approaches lead to identical coefficient estimates
and identical covariance matrices of the estimated coefficients.

%%% Local Variables: 
%%% mode: latex
%%% TeX-master: "systemfit"
%%% End: 
