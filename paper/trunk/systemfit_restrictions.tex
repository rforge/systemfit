%%%%%%%%%%%%%%%%%%%%%%%%%%%%%%%%%%%%%%%%%%%%%%%%%%%%%%
\subsection{Imposing linear restrictions}\label{sec:Restrictions}
%%%%%%%%%%%%%%%%%%%%%%%%%%%%%%%%%%%%%%%%%%%%%%%%%%%%%%

It is common to perform hypothosis tests by imposing restrictions on
the parameter estimates.
There are two ways to impose linear parameter restrictions.
First, a matrix $T$ can be specified that
\begin{equation}
   \beta = T \cdot \beta^* \label{eq:T-restr} 
\end{equation}
where $\beta^*$ is a vector of restricted (linear independent) coefficients,
and $T$ is a matrix with the number of rows equal to the number of
unrestricted coefficients ($\beta$) and
the number of columns equal to the number of restricted coefficients
($\beta^*$).
$T$ can be used to map each unrestricted coefficient to one or more
restricted coefficients.

To impose these restrictions, the $X$ matrix is
(post-)multiplied by this $T$ matrix.
\begin{equation}
    X^* = X \cdot T
\end{equation}
Then, $X^*$ is substituted for $X$ and a standard estimation as described
in the previous section is done
(equations~\ref{eq:ols}--\ref{eq:3slsEViews}).
This results in the linear independent parameter estimates $\beta^*$ and
their covariance matrix.
The original parameters can be obtained by equation (\ref{eq:T-restr})
and the covariance matrix of the original parameters
can be obtained by:
\begin{equation}
   \Cov \left[ \widehat{\beta} \right] = T \cdot \Cov \left[ \widehat{\beta}^* \right] \cdot T'
\end{equation}

The second way to impose linear parameter restrictions
can be formulated by
\begin{equation}
   R \beta^0 = q
\end{equation}
where $\beta^0$ is the vector of the restricted coefficients, 
and $R$ and $q$ are a matrix and vector, respectively,
to impose the restrictions \citep[see][p.\ 100]{greene03}.
Each linear independent restriction is represented by one row of $R$
and the corresponding element of~$q$.

The first way is less flexible than this latter one%
\footnote{
While restrictions like $\beta_1 = 2 \beta_2$ can be imposed by
both methods,
restrictions like $\beta_1 + \beta_2 = 4$ can be imposed only
by the second method.
}, 
but the first way is preferable if equality constraints for coefficients
across many equations of the system are imposed. 
Of course, these restrictions can be also imposed using
the latter method.
However, while the latter method increases the dimension of the 
matrices to be inverted during estimation, the first reduces it. 
Thus, in some cases the latter way leads to estimation problems
(e.g.\ (near) singularity of the matrices to be inverted),
while the first doesn't.

These two methods can be combined.
In this case the restrictions imposed using the latter method are
imposed on the linear independent parameters due to the restrictions
imposed using the first method:
\begin{equation}
   R \beta^{*0} = q
\end{equation}
where $\beta^{*0}$ is the vector of the restricted $\beta^*$ coefficients.

%%%%%%%%%%%%%%%%%%%%%%%%%%%%%%%%%%%%%%%%%%%%%%%%%%
\subsubsection{Restricted OLS estimation}

The OLS estimator restricted by $R \beta^0 = q$ can be obtained by
\begin{equation}
   \left[ \begin{array}{c}
      \widehat{\beta}^0_{OLS} \\ \widehat{\lambda}
   \end{array} \right]
   =
   \left[ \begin{array}{cc}
      X' X & R' \\ 
      R & 0
   \end{array} \right]^{-1}
   \cdot
   \left[ \begin{array}{c}
      X' y \\ q 
   \end{array} \right]
\end{equation}
where $\lambda$ is a vector of the Lagrangean multipliers of the restrictions.
If the whole system is treated as one single equation,
the covariance matrix of the estimated parameters is
\begin{equation}
   \Cov 
   \left[ \begin{array}{c}
      \widehat{\beta}^0_{OLS} \\ \widehat{\lambda}
   \end{array} \right] 
   = \sigma^2 
   \left[ \begin{array}{cc}
      X' X & R' \\ 
      R & 0
   \end{array} \right]^{-1}
\end{equation}
with $\sigma^2 = E \left( u' u \right)$.
If the disturbance terms of the individual equations
are allowed to have different variances, 
the covariance matrix of the estimated parameters is
\begin{equation}
   \Cov 
   \left[ \begin{array}{c}
      \widehat{\beta}^0_{OLS} \\ \widehat{\lambda}
   \end{array} \right] 
   = 
   \left[ \begin{array}{cc}
      X' \Omega^{-1} X & R' \\ 
      R & 0
   \end{array} \right]^{-1}
\end{equation}
with $\Omega = \Sigma \otimes I$, 
$\sigma_{ij} = 0 \; \forall \; i \neq j$ and
$\sigma_{ii} = E \left( u_i' u_i \right)$.

%%%%%%%%%%%%%%%%%%%%%%%%%%%%%%%%%%%%%%%%%%%%%%%%%%
\subsubsection{Restricted WLS estimation}

The WLS estimator restricted by $R \beta^0 = q$ can be obtained by
\begin{equation}
   \left[ \begin{array}{c}
      \widehat{\beta}^0_{WLS} \\ \widehat{\lambda}
   \end{array} \right]
   =
   \left[ \begin{array}{cc}
      X' \Omega^{-1} X & R' \\ 
      R & 0
   \end{array} \right]^{-1}
   \cdot
   \left[ \begin{array}{c}
      X' \Omega^{-1} y \\ q 
   \end{array} \right]
\end{equation}
with $\Omega = \Sigma \otimes I$, 
$\sigma_{ij} = 0 \; \forall \; i \neq j$ and
$\sigma_{ii} = E \left( u_i' u_i \right)$.
The covariance matrix of the estimated parameters is
\begin{equation}
   \Cov 
   \left[ \begin{array}{c}
      \widehat{\beta}^0_{WLS} \\ \widehat{\lambda}
   \end{array} \right] 
   = 
   \left[ \begin{array}{cc}
      X' \Omega^{-1} X & R' \\ 
      R & 0
   \end{array} \right]^{-1}
\end{equation}

%%%%%%%%%%%%%%%%%%%%%%%%%%%%%%%%%%%%%%%%%%%%%%%%%%
\subsubsection{Restricted SUR estimation}

The SUR estimator restricted by $R \beta^0 = q$ can be obtained by
\begin{equation}
   \left[ \begin{array}{c}
      \widehat{\beta}^0_{SUR} \\ \widehat{\lambda}
   \end{array} \right]
   =
   \left[ \begin{array}{cc}
      X' \Omega^{-1} X & R' \\ 
      R & 0
   \end{array} \right]^{-1}
   \cdot
   \left[ \begin{array}{c}
      X' \Omega^{-1} y \\ q 
   \end{array} \right]
\end{equation}
with $\Omega = \Sigma \otimes I$ and
$\sigma_{ij} = E \left( u_i' u_j \right)$.
The covariance matrix of the estimated parameters is
\begin{equation}
   \Cov 
   \left[ \begin{array}{c}
      \widehat{\beta}^0_{SUR} \\ \widehat{\lambda}
   \end{array} \right] 
   = 
   \left[ \begin{array}{cc}
      X' \Omega^{-1} X & R' \\ 
      R & 0
   \end{array} \right]^{-1}
\end{equation}

%%%%%%%%%%%%%%%%%%%%%%%%%%%%%%%%%%%%%%%%%%%%%%%%%%
\subsubsection{Restricted 2SLS estimation}

The 2SLS estimator restricted by $R \beta^0 = q$ can be obtained by
\begin{equation}
   \left[ \begin{array}{c}
      \widehat{\beta}^0_{2SLS} \\ \widehat{\lambda}
   \end{array} \right]
   =
   \left[ \begin{array}{cc}
      \widehat{X}' \widehat{X} & R' \\ 
      R & 0
   \end{array} \right]^{-1}
   \cdot
   \left[ \begin{array}{c}
      \widehat{X}' y \\ q 
   \end{array} \right]
   \label{eq:beta2SLSr}
\end{equation}
If the whole system is treated as one single equation,
the covariance matrix of the estimated parameters is
\begin{equation}
   \Cov 
   \left[ \begin{array}{c}
      \widehat{\beta}^0_{2SLS} \\ \widehat{\lambda}
   \end{array} \right] 
   = \sigma^2 
   \left[ \begin{array}{cc}
      \widehat{X}' \widehat{X} & R' \\ 
      R & 0
   \end{array} \right]^{-1}
\end{equation}
with $\sigma^2 = E \left( u' u \right)$.
If the disturbance terms of the individual equations
are allowed to have different variances, 
the covariance matrix of the estimated parameters is
\begin{equation}
   \Cov 
   \left[ \begin{array}{c}
      \widehat{\beta}^0_{2SLS} \\ \widehat{\lambda}
   \end{array} \right] 
   = 
   \left[ \begin{array}{cc}
      \widehat{X}' \Omega^{-1} \widehat{X} & R' \\ 
      R & 0
   \end{array} \right]^{-1}
\end{equation}
with $\Omega = \Sigma \otimes I$, 
$\sigma_{ij} = 0 \; \forall \; i \neq j$ and
$\sigma_{ii} = E \left( u_i' u_i \right)$.


%%%%%%%%%%%%%%%%%%%%%%%%%%%%%%%%%%%%%%%%%%%%%%%%%%
\subsubsection{Restricted W2SLS estimation}

The W2SLS estimator restricted by $R \beta^0 = q$ can be obtained by
\begin{equation}
   \left[ \begin{array}{c}
      \widehat{\beta}^0_{W2SLS} \\ \widehat{\lambda}
   \end{array} \right]
   =
   \left[ \begin{array}{cc}
      \widehat{X}' \Omega^{-1} \widehat{X} & R' \\ 
      R & 0
   \end{array} \right]^{-1}
   \cdot
   \left[ \begin{array}{c}
      \widehat{X}' \Omega^{-1} y \\ q 
   \end{array} \right]
\end{equation}
with $\Omega = \Sigma \otimes I$, 
$\sigma_{ij} = 0 \; \forall \; i \neq j$ and
$\sigma_{ii} = E \left( u_i' u_i \right)$.
The covariance matrix of the estimated parameters is
\begin{equation}
   \Cov 
   \left[ \begin{array}{c}
      \widehat{\beta}^0_{W2SLS} \\ \widehat{\lambda}
   \end{array} \right] 
   = 
   \left[ \begin{array}{cc}
      \widehat{X}' \Omega^{-1} \widehat{X} & R' \\ 
      R & 0
   \end{array} \right]^{-1}
\end{equation}


%%%%%%%%%%%%%%%%%%%%%%%%%%%%%%%%%%%%%%%%%%%%%%%%%%
\subsubsection{Restricted 3SLS estimation}

The standard 3SLS estimator restricted by $R \beta^0 = q$ can be obtained by
\begin{equation}
   \left[ \begin{array}{c}
      \widehat{\beta}^0_{3SLS} \\ \widehat{\lambda}
   \end{array} \right]
   =
   \left[ \begin{array}{cc}
      \widehat{X}' \Omega^{-1} \widehat{X} & R' \\ 
      R & 0
   \end{array} \right]^{-1}
   \cdot
   \left[ \begin{array}{c}
      \widehat{X}' \Omega^{-1} y \\ q 
   \end{array} \right]
   \label{eq:3slsGlsR}
\end{equation}
with $\Omega = \Sigma \otimes I$ and
$\sigma_{ij} = E \left( u_i' u_j \right)$.
The covariance matrix of this estimator is
\begin{equation}
   \Cov 
   \left[ \begin{array}{c}
      \widehat{\beta}^0_{3SLS} \\ \widehat{\lambda}
   \end{array} \right] 
   = 
   \left[ \begin{array}{cc}
      \widehat{X}' \Omega^{-1} \widehat{X} & R' \\ 
      R & 0
   \end{array} \right]^{-1}
   \label{eq:cov3slsr}
\end{equation}
The 3SLS-IV estimator restricted by $R \beta^0 = q$ can be obtained by
\begin{equation}
   \left[ \begin{array}{c}
      \widehat{\beta}^0_{3SLS-IV} \\ \widehat{\lambda}
   \end{array} \right]
   =
   \left[ \begin{array}{cc}
      \widehat{X}' \Omega^{-1} X & R' \\ 
      R & 0
   \end{array} \right]^{-1}
   \cdot
   \left[ \begin{array}{c}
      \widehat{X}' \Omega^{-1} y \\ q 
   \end{array} \right]
   \label{eq:3slsIvR}
\end{equation}
with
\begin{equation}
   \Cov 
   \left[ \begin{array}{c}
      \widehat{\beta}^0_{3SLS-IV} \\ \widehat{\lambda}
   \end{array} \right] 
   = 
   \left[ \begin{array}{cc}
      \widehat{X}' \Omega^{-1} \widehat{X} & R' \\ 
      R & 0
   \end{array} \right]^{-1}
\end{equation}
The restricted 3SLS-GMM estimator can be obtained by
\begin{equation}
   \left[ \begin{array}{c}
      \widehat{\beta}^0_{3SLS-GMM} \\ \widehat{\lambda}
   \end{array} \right]
   =
   \left[ \begin{array}{cc}
      X' H \left( H' \Omega H \right)^{-1} H' X & R' \\ 
      R & 0
   \end{array} \right]^{-1}
   \cdot
   \left[ \begin{array}{c}
      X' H \left( H \Omega H \right)^{-1} H' y \\ q 
   \end{array} \right]
   \label{eq:3slsGmmR}
\end{equation}
with
\begin{equation}
   \Cov 
   \left[ \begin{array}{c}
      \widehat{\beta}^0_{3SLS-GMM} \\ \widehat{\lambda}
   \end{array} \right] 
   = 
   \left[ \begin{array}{cc}
      X' H \left( H' \Omega H \right)^{-1} H' X & R' \\ 
      R & 0
   \end{array} \right]^{-1}
\end{equation}
The restricted 3SLS estimator based on the suggestion of
\cite{schmidt90} is:
\begin{equation}
   \left[ \begin{array}{c}
      \widehat{\beta}^0_{3SLS-Schmidt} \\ \widehat{\lambda}
   \end{array} \right]
   =
   \left[ \begin{array}{cc}
      \widehat{X}' \Omega^{-1} \widehat{X} & R' \\ 
      R & 0
   \end{array} \right]^{-1}
   \cdot
   \left[ \begin{array}{c}
      \widehat{X}' \Omega^{-1} H \left( H' H \right)^{-1} H' y \\ q 
   \end{array} \right]
   \label{eq:3slsSchmidtR}
\end{equation}
with
\begin{eqnarray}
   \Cov 
   \left[ \begin{array}{c}
      \widehat{\beta}^0_{3SLS-Schmidt} \\ \widehat{\lambda}
   \end{array} \right] 
   & = & 
   \left[ \begin{array}{cc}
      \widehat{X}' \Omega^{-1} \widehat{X} & R' \\ 
      R & 0
   \end{array} \right]^{-1}
   \\
   & & \cdot
   \left[ \begin{array}{cc}
      \widehat{X}' \Omega^{-1} H \left( H' H \right)^{-1} H' \Omega
      H \left( H' H \right)^{-1} H' \Omega^{-1} \widehat{X} & 0' \\ 
      0 & 0
   \end{array} \right]^{-1}
   \nonumber \\
   & & \cdot
   \left[ \begin{array}{cc}
      \widehat{X}' \Omega^{-1} \widehat{X} & R' \\ 
      R & 0
   \end{array} \right]^{-1}
   \nonumber
\end{eqnarray}
The econometrics software EViews calculates the restricted 3SLS estimator by:
\begin{equation}
   \left[ \begin{array}{c}
      \widehat{\beta}^0_{3SLS-EViews} \\ \widehat{\lambda}
   \end{array} \right]
   =
   \left[ \begin{array}{cc}
      \widehat{X}' \Omega^{-1} \widehat{X} & R' \\ 
      R & 0
   \end{array} \right]^{-1}
   \cdot
   \left[ \begin{array}{c}
      \widehat{X}' \Omega^{-1} \left( y - X \widehat{\beta}^0_{2SLS} \right)
      \\ q 
   \end{array} \right]
   \label{eq:3slsEViewsR}
\end{equation}
where $\widehat{\beta}^0_{2SLS}$ is the restricted 2SLS estimator calculated
by equation (\ref{eq:beta2SLSr}). 
To calculate the covariance matrix
EViews uses the standard formula of the restricted 3SLS
estimator~(\ref{eq:cov3slsr}).


If the same instrumental variables are used in all equations 
($H_1 = H_2 = \ldots = H_G$), 
all the above mentioned approaches lead to identical parameter estimates
and identical covariance matrices of the estimated parameters.

%%% Local Variables: 
%%% mode: latex
%%% TeX-master: "systemfit"
%%% End: 
