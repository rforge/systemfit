\documentclass[12pt,english]{scrartcl}
\usepackage{lmodern}
\usepackage[T1]{fontenc}
\usepackage[latin1]{inputenc}
\usepackage{babel}
\usepackage{geometry}
\geometry{tmargin=0.75in, bmargin=0.9in, lmargin=0.8in,
   rmargin=0.8in, headheight=0.1cm, headsep=0.1cm, footskip=0.4in }
\usepackage{setspace}
\onehalfspacing
\usepackage[authoryear]{natbib}
\usepackage{url}
\usepackage{csquotes}
\MakeOuterQuote{�}

% avoid club lines / orphans
\clubpenalty=10000
% avoid widow lines
\widowpenalty=10000
\displaywidowpenalty=10000


\begin{document}
\begin{center}
{\Large\textbf{\textsf{
systemfit: A Package for Estimating\\[0.3em]
Systems of Simultaneous Equations in \proglang{R} (JSS~224)%
\footnote{
The original title was: �systemfit: A Package to Estimate
Simultaneous Equation Systems in \proglang{R}�.}
}}}

\bigskip

{\Large\textbf{\textsf{Responses to Reviewer A}}}
\end{center}


We thank this reviewer for her/his very helpful comments and suggestions
that assisted us to considerably improve the software as well as the paper.
The reviewer has made one overall comment, 9 major comments and
several minor comments.
We have structured our reply accordingly.

\section{Overall Comment}
% ########################################################
% ## Review for                                         ##
% ##   systemfit: A Package to Estimate Simultaneous    ##
% ##   Equation Systems in R                            ##
% ##   Arne Henningsen, Jeff D. Hamann                  ##
% ## submitted for publication in                       ##
% ##   Journal of Statistical Software                  ##
% ########################################################
% 
% The paper describes the CRAN package "systemfit" for fitting simultaneous
% equation models in R. Estimation procedures include standard techniques in
% econometrics such as SUR, 2SLS and 3SLS.
% The paper can be divided roughly into three parts: Sections 1/2 explain
% the underlying theory (~12 pages), Sections 3/4 describe the code in systemfit
% and how to use it (~10 pages) and Section 5 reproduces results from various
% textbooks (~8 pages).
% The paper and software are both interesting and mostly well-written: they
% deserve publication in JSS. However, some major and minor comments have to
% be addressed before it can be accepted. In particular, the revision has to
% streamline exposition and notation of standard results, and to combine the
% (established) theory with the implementation more closely (otherwise the reader
% has to wait more than 10 pages before s/he learns something about the package).
% I suggest to eliminate Section 2 in its current form and to re-write the
% description of the theory in a more general and unified way while weaving
% it with the translation into computational tools.
We appreciate this recommendation of the reviewer very much,
but we consider the presentation of the theoretical background
in a separate section
as an advantage rather than as a drawback.
As the formulas to calculate the estimators and their covariances
are not essential to \emph{use} \pkg{systemfit},
we think that the formulas and the usage should be presented separately.
Furthermore, our structure allows readers
who already know the theoretical background
(or who do not want to know it)
to skip this entire section
and to go immediately into the section they are interested in
(e.g.\ section �Using systemfit�).
Hence, they do not have to scan a large composite section
just to figure out a few pieces of information
(e.g.\ how to use the commands of \pkg{systemfit}).
Readers who have skipped the �Statistical background�
can easily go back to this section
if they are interested in a particular part of the theory,
because all parts in section �Using systemfit�
have references to their theoretical background
in section �Statistical background�.

% As for the software itself,
% several modifications are necessary to make it more intuitive for R users
% and to improve how the underlying concepts are turned into computational 
% tools. More details are described below.
As suggested by this reviewer,
the software package \pkg{systemfit} has been considerably modified
to make it more intuitive for \proglang{R} users.
We tried to make the usage of \pkg{systemfit}
as close as possible to the primary regression tools in \proglang{R}
(e.g.\ \code{lm}).
More details of theses modifications are described below.

% 
% 
\section{Major Points}
% MAJOR POINTS:
\subsection{Title}
%   o title
% 
%     The title sounds a bit awkward in English: first, "a package *for*
%     estimat*ing*" would be more appropriate, secondly "systems of
%     equations" seems to be preferred over "equation systems" in English.
%
We have changed the title according to these suggestions.


\subsection{Abstract/Introduction}
%   o abstract/introduction
% 
%     Both, abstract and introduction, talk rather generally about
%     systems of equations reviewing rather well-known theoretical
%     results. I would suggest to revise both, focusing on what
%     *can* be done *in R* rather than giving a tutorial what *should*
%     be done *in general*. Of course, it is ok to review some theory,
%     however, practitioners should know what they are doing and the
%     paper should mainly teach them how to do it in R.
%
Both, abstract and introduction, now focus much more
on the capabilities of the \pkg{systemfit} package
rather than on the general theory of estimating systems of equations.


\subsection{Estimation Methods (2.1)}
%   o estimation methods (2.1)
% 
%     The description of the estimation methods can be shortened considerably
%     and should focus on the unifying properties of the different estimators.
%     All methods from OLS through to standard 3SLS comply with formulas
%     (12) and (13), only with differing choices for Omega (s*I, diagonal or
%     full Omega) and X (original X vs \hat X from instruments). The exposition
%     could be streamlined considerably if this structure were exploited.
% 
The reviewer is right that it is possible to shorten
the description of the estimation methods
by exploiting the common properties of the different estimators.
However, in our opinion this shortening would considerably worsen
the readability of this section.
Furthermore, a major purpose of this section is
that a user of the \pkg{systemfit} package
can take a look at the theory (e.g.\ formulas) of a specific estimation method.
Therefore, we fully describe each estimation method
in a single (subsub)section.

%     Terminology for the different estimators also varies between various
%     communities using systems of equations models. The paper should address
%     this. For example, in econometrics the GLS estimator would be an estimator
%     with known Omega while FGLS (feasible GLS) is the estimator based on
%     an estimated Omega (see Greene, 2003). The current discussion makes this
%     even more problematic because Omega is treated as known throughout
%     Section 2. The discussion of the estimators needs to be combined with
%     discussion of the estimators for Omega.
%
The paper no longer treats $\Omega$ as known in sections~2.1 and~2.2.
The fact that $\Sigma$ and $\Omega$ are unknown and have to be estimated
is no longer �hidden� in section 2.3,
but is mentioned in sections~2.1 and~2.2 now.

%     For the varying 3SLS estimators, it would help to point out why/when
%     these are needed at the beginning rather than at the end of the 3SLS
%     section. 
%
We think that it is better
to present the review of the varying 3SLS estimators
at the end rather than at the beginning of the 3SLS section,
because it is easier for the reader to understand this review
if the compared estimation methods have been described before.


\subsection{Imposing Linear Restrictions (2.2)}
%   o imposing linear restrictions (2.2)
% 
%     I found this section somewhat confusing. At first, it should be made
%     clear that this is about estimating the coefficients under linear
%     restrictions (and not inference or something else). Then, it needs
%     to be explained more concisely that all linear restrictions are of
%     type (33) and that the rest of the discussion is about different
%     computational algorithms for obtaining the estimates under restriction.
%     Again, the exposition on pages 7-9 can be improved (and shortened)
%     if the unifying structure is exploited. Also the notation can be
%     streamlined concerning beta, \hat beta, beta^0, beta^* etc.
%
First, it has been made clearer
that this section is about estimating coefficients under linear restrictions
(and not about inference).
Second, it has been explained more concisely,
how the two methods to impose linear restrictions are implemented
in \pkg{systemfit},
i.e.\ that the rest of this (sub)section belongs to the second method
(equation 31, former equation 33).
Finally, we have streamlined the notations of the $\beta$s.


\subsection{systemfitClassic (4.3)}
%   o systemfitClassic (4.3)
% 
%     First, this function deserves a better name: it is long, unintuitive
%     and does not reflect the underlying conceptual tools. If the requirement
%     of using the same formula for all equations would be relaxed (but
%     still supported) a conceivable name would be sur() for SUR type 
%     models. But it would also be conceivable to stick to the current
%     requirement of a single formula (then, still a better name would
%     be needed) and to set up a second wrapper sur() etc. In any case,
%     the function (its name, arguments and output) should reflect what
%     is conceptually intuitive for this kinds of models (for the communities
%     using it).
The function \code{systemfitClassic} has been removed.
%Its functionality has been taken over by \code{systemfit}.
A (classical) �Seemingly Unrelated Regression� analysis
with panel-like data
can now be done by \code{systemfit} itself.
If the argument \code{data} is an object of class \code{pdata.frame}
(�panel data frame�, created with the function \code{pdata.frame}
from the \proglang{R} package \pkg{plm} \citep{r-plm-0.1-2}),
\code{systemfit} estimates the provided formula for each
(cross-section) individual.

In our opinion,
it is not desirable to relax the requirement
of using the same formula for all equations
in (classical) �Seemingly Unrelated Regression� analyses
with panel-like data.
First, it would be rather cumbersome for the user to specify,
which formula should be applied to which (cross-section) individual.
Second, this type of analysis can be easily done
by reshaping the data (e.g.\ with \code{reshape})
and then using (standard) \code{systemfit}.
Third, we think that this feature would rarely be used.

%     Second, the interface should also be re-designed: the standard in R is
%     that the formula/s is/are the first argument (which would also be more
%     convenient for systemfit()), e.g.,
%       function(formula, method = "FGLS", data = ..., ...)
%     Currently, the data argument seems to be mandatory (opposed to the
%     paper documentation). The specification of eqnVar and timeVar via
%     strings seems to be uncommon and suboptimal. Probably some kind of
%     formula would be preferable. Maybe a formula similar to those used in
%     nlme could be employed? If more than one formula is specified, one has
%     to pay attention that no problems are used by different patterns of
%     missingness in differing equations.
%
All remaining concerns regarding \code{systemfitClassic}
(order of arguments;
argument \code{data} being mandatory;
arguments \code{eqnVar} and \code{timeVar} being strings)
have been resolved by its removal.


\subsection{Testing Linear Restrictions (4.4)}
%   o testing linear restrictions (4.4)
% 
%     The package introduces several functions that are somewhat orthogonal
%     to exisiting methodology. A single method (with "test" argument) for
%     the linear.hypothesis() function from "car" might seem more appropriate
%     here. The Wald test can already now be reproduced using the default
%     linear.hypothesis() method (except for formula formatting)
%       linear.hypothesis(fitsur, Rmat, qvec)    
%     If the coefficient naming would be modified (see below), the convenience
%     interface for specifying the restrictions in linear.hypothesis() could
%     also be re-used.    
%     Also two of the functions look like methods to the generics lrtest() and
%     waldtest() in package "lmtest", although this is not the case. Furthermore,
%     some more general method of comparing various fitted models would be nice,
%     e.g., an anova() method or methods to waldtest()/lrtest() although the 
%     latter two seem to be under development.
%     The output formatting of the current function is also not very elegant,
%     a new class along with its print method is introduced for each test. It
%     would be much easier to re-use "htest" or (probably better) "anova" as
%     classes for this.
We have modified the tools for testing linear hypothesis
according to the suggestions of this reviewer.
The user can test linear restrictions by a Wald/$\chi^2$ or F test
using the generic function \code{linear.hypothesis} now.
The method \code{linear.hypothesis.systemfit} uses
\code{linear.}\hspace{0pt}\code{hypothesis.}\hspace{0pt}\code{default}
of the \pkg{car} package
to compute the Wald/$\chi^2$ statistic.
The $F$ statistic is computed in the \pkg{systemfit} package internally,
but the returned object is of the same class and structure
as the result of \code{linear.hypothesis.default}.

The method \code{lrtest.systemfit} is a wrapper function
to \code{lrtest.default} of the \pkg{lmtest} package now.
Its main task is to specify the names for the compared objects.

%
\subsection{Organize Internals of systemfit}
%   o organize internals of systemfit
% 
%     The internals of a fitted "systemfit" object could be re-organized. Currently,
%     there are a lot of single letter elements - more standard naming as in "lm"
%     objects etc. would be more desirable.
We have renamed many elements of a fitted \code{systemfit} object
to make the naming more standard.
A comparison with the elements returned by \code{lm}
is available in the paper (appendix~A).

%     Also, computing the partial t tests
%     should/could be done in the summary() method, avoiding that certain parameters
%     (not affecting the fit itself, such as degrees of freedom) necessarily have to
%     be specified for model fitting.
Furthermore, we have followed the advice of this referee
and moved the computation of the partial $t$ statistic
from the \code{systemfit()} function to its \code{summary()} method.

% 
%     The technical control parameters could also be collected in a list of arguments, e.g.,
%       control = systemfit.control()
%     that sets suitable defaults but can also easily be changed. If possible `...'
%     arguments could be passed to systemfit.control() assuring backward compatibility
%     and facilitating interactive usage.
Following the suggestion of this referee,
the technical control parameters of \code{systemfit()} are collected
in a single argument \code{control} now.
This argument is a list that contains the technical control parameters,
where suitable default values are set by the function
\code{systemfit.control()}.

%     The names of various
%     arguments should be re-thought, e.g. "tx", "R.restr" and "q.rests" seem rather
%     arbitrary. Similar comments hold for other arguments, see also below for some
%     more comments.
The names of several arguments have been improved.
The arguments to impose parameter restrictions by $R \beta = q$
(former \code{R.restr} and \code{q.restr})
have been renamed to \code{restrict.matrix} and \code{restrict.rhs},
respectively
(following the arguments \code{hypothesis.matrix} and \code{rhs}
of the generic function \code{linear.hypothesis}).
The argument to impose parameter restrictions
by modifying (post-multiplying) the regressor matrix
(former \code{TX}) has been renamed to \code{restrict.regMat}.
Finally, the arguments to specify the methods
for the 3SLS estimation and to calculated the residual covariance matrix
(former \code{formula3sls} and \code{rcovformula})
have been renamed to \code{method3sls} and \code{methodRCov},
respectively.

%     As noted above, the standard in R seems to supply the formula(s)
%     as the first argument and to have then some `method' argument.
We have reversed the order of arguments \code{method} and \code{eqns}
of \code{systemfit()}.
It is no longer necessary to reverse the order of the arguments
of \code{systemfitClassic()},
because this function has been removed.

% 
%     Some more standard elements of a model fit should also be kept, in particular
%     the call so that update() can work with "systemfit" objects, e.g.,
%       fitsur <- systemfit("SUR", list(demand = eqDemand, supply = eqSupply))
%       fitsur$call <- call("systemfit", "SUR", list(demand = eqDemand, supply = eqSupply))
%       fitsurRmat <- update(fm, R.restr = Rmat, q.restr = qvec)
%     The call can easily be captured via match.call().
The call of \code{systemfit} is stored in the returned object now.
This allows the application of \code{update()} to \code{systemfit} objects.

%     Similarly, one could think 
%     about how the terms and formula can be preserved and possibly made available
%     in terms() and formula() calls. Other standard methods would also be nice to have,
%     e.g., df.residual() or logLik().
Following the suggestions of this referee,
the �model terms� (including the formulas) of each equation are preserved now.
We have added methods \code{terms.systemfit},
\code{terms}\hspace{0pt}\code{.systemfit}\hspace{0pt}\code{.equation},
\code{formula.systemfit}, and \code{formula.systemfit.equation}
to extract the �model terms� and formulas
of the system and of a single equation.
The new method \code{logLik} computes the log-likelihood value
of a fitted \code{systemfit} object.
Furthermore, the default method of \code{df.residual}
can extract the degrees of freedom of the residuals,
both of the system and of a single equation now.

%     (By the way: The row/column naming in fitted()
%     and residuals() should match, probably re-using equation and observation names.)
The data frames returned by \code{residuals.systemfit} and
\code{fitted.systemfit} have the same row names and column names now.

% 
%     The coefficient naming should be improved, e.g., "<eqname>_<varname>"
%     where <eqname> can be set by default to "eq1", "eq2" etc. but should use the
%     equation names (if supplied). Spaces in names should be avoided.
%     The coefficient naming in the systemfitClassic() interface could also be 
%     improved, see e.g. the output on p.27, l.546: "eq 2 capital.Chrysler" redundantly
%     codes the firm (both by eq and name) and "Chrysler_capital" or "Chryler.capital"
%     would be preferable.
The naming of the coefficients has been improved
according to the suggestions of this referee:
they are now \code{<eqname>}\verb!_!\code{<varname>},
where \code{<eqname>} is the equation name,
which is set to \code{"eq1"}, \code{"eq2"} etc.\
if no equation names are supplied by the user.
In the case of a (classical) �Seemingly Unrelated Regression� analysis
with panel-like data,
all redundant parts of the coefficient names have been removed:
the coefficient names in the vector of all coefficients are
\code{<individual>}\verb!_!\code{<varname>}
and the coefficient names in the vector of the coefficients of a single
individual are just \code{<varname>}.

% 
%     It would also be nice to have some more control about the amount of output of
%     the summary() method. Currently, the output can become very long easily when the
%     number of equations increases. One could want to control the equations and
%     correlations used, e.g., by something like
%       summary(sfobject, equation = ..., cor = ...)
The user has more control about the amount of output of
\code{summary()} now.
We have added arguments \code{residCov} and \code{equations}
to function \code{print.summary.systemfit()}.
Argument \code{residCov} indicates
whether the residual correlation matrix,
the residual covariance matrix, and its determinant are printed;
argument \code{equations} indicates
whether summary results of each equation are printed
(or if just the coefficients are printed).
As users often use \code{summary()} without \code{print()},
we have added arguments \code{printResidCov} and \code{printEquations}
to \code{summary()},
which set default values for arguments \code{residCov} and \code{equations}
of function \code{print.summary.systemfit()}.

% 
%
\subsection{Computations in systemfit}
%   o computations in systemfit()
% 
%     A few computations within the systemfit() functions can be further streamlined.
%     A simple modification is to use
%       method %in% c("OLS", "WLS", ...)
%     instead of
%       method == "OLS" | method == "WLS" | method == ...
We have streamlined the source code of the \pkg{systemfit} package.
For instance, we have followed the suggestion of this referee
and replaced all
�\code{method == "OLS" | method == "WLS" | method == ...}�
by �\code{method \%in\% c("OLS", "WLS", ...)}�.

%     Furthermore, various marix computations can be made even more stable. Currently,
%     the code still has    
%       solve(crossprod(X), crossprod(X,Y))    
%       solve(crossprod(X))
%     which looks very similar to the code presented in Bates (2004). However, Bates
%     (2004) uses the "Matrix" package, which has specialized solve() for certain
%     matrix types. To perform this `by hand' without Matrix, one way would be re-using
%     lm.fit() for the first command. Alternatively, the QR or Cholesky decomposition
%     could be carried out using qr() or chol(). The latter command could be replaced
%     by
%      chol2inv(qr.R(qr(X)))
%      chol2inv(chol(crossprod(X)))
%     where chol() should be faster but qr() might be more accurate.
%
\subsection{Replication of Textbook Results (5)}
%   o Replication of textbook results (5)
% 
%     Having a section like this is excellent and should help teachers and students
%     for using systemfit in their lectures. Furthermore, it encourages reproducible
%     research which always deserves support. However, I would suggest to streamline
%     the section to make it more readable. The section should discuss its aims (where
%     I would add replication to reliability) and summarize the most important results
%     and experiences from the replication. Then, I would pick a `typical' example and
%     discuss it in some more detail with full R output. For the remaining examples,
%     the input suffices as long as there are no deviations of the systemfit results
%     from the published numbers.
% 
%     In addition to Table 14.1 and 14.2, I tried to replicate Table 14.3 with
%     ML estimates (using iterated FGLS) in Greene (2003) via 
%       systemfitClassic("WSUR", invest ~ value + capital, "firm", "year",
%                        data = GrunfeldGreene, rcovformula = 0, pooled = TRUE, maxiter = 1000)    
%     for the pooled model which yielded rather similar results; and via
%       systemfitClassic("SUR", invest ~ value + capital, "firm", "year",
%                        data = GrunfeldGreene, rcovformula = 0, maxiter = 1000)    
%     for the full model which is not unsimilar to Greene's results but also not very
%     close either. Is this a problem of systemfit or of Greene or of me fitting the
%     wrong model?
% 
%     Having two Grunfeld data sets is confusing as the data used by Theil is simply
%     a subset of the data used by Greene:
%       subset(GrunfeldGreene, firm %in% c("General Electric", "Westinghouse"))
%     Of course, there are different versions of this data set, but neither the paper
%     nor the manual pages of GrunfeldGreene and GrunfeldTheil appropriately discuss
%     the problem. From the manual and the paper, the reader is only able to feel
%     vaguely insecure about the appropriateness of the selected subsets. Also note
%     that a more complete (and slightly different) version of the data set is
%     available in package "Ecdat".
% 
%
\section{Minor Points and Typos}
We have corrected the paper according to all of these comments
except for �p.3, l.45� and the second part of �p.3, l.52�.

% MINOR POINTS & TYPOS
% 
%   o p.1, l.3. Is a "theoretical model that is econometrically estimated"
%     a well-defined term?
% 
%   o p.2, Eq.2. The \cdot between the regressor matrix and its coefficients
%     should be omitted (since this is not a scalar multiplication).
% 
%   o p.2, l.42. For symmetry between the two index types, it might be
%     more convenient to denote the equation by i/j and the observation
%     by s/t rather than t/t*, in particular as * is also used later for 
%     other purposes.
% 
%   o p.3, l.45. While it is possible to call u "residual" and \hat u an 
%     "estimated residual", the more standard (and less confusing) terminology
%     is to call u "disturbance" or "error term" and the estimated quantity
%     \hat u "residual".
% 
%   o p.3, l.52. The equation $\sigma^2 = E(u'u)$ is wrong because the 
%     right-hand side is a sum over squared errors. This mistake is repeated
%     throughout the paper. Also (as noted above), no distinction between
%     population and sample covariances is made which leads to confusion.
% 
%   o p.3, l.69. "the GLS is" -> "the GLS estimator is"
%                "called ... SUR" -> "called ... SUR estimator"
% 
%   o p.4, l.72/3. The "E[u|X] = 0" type equations imply (but are not
%     equivalent to) the assumptions of uncorrelated disturbances.
% 
% 
%   o p.4, l.79. The formulation "estimated coefficients are biased" is somewhat
%     sloppy, bias is a property of the estimator.
% 
%   o p.6, l.127 / p.10, l.198. The symbol T is used for different quantities
%     here.
% 
%   o p.12, eq.63. One Sigma is lacking a hat. Now M is used for the number
%     of equations, before it was G.
% 
%   o p.13, l.285. Rather than calling the methods by their full name, it
%     is preferred to refer to the predict or coef method for "systemfit"
%     objects.
% 
%   o p. 13, l.289. The remarks about why not the full code is presented
%     sound very clumsy and should be avoided.
% 
%   o p.13, l.299. "interest" -> "interested"
% 
%   o p.13, l.321. "deviding" -> "dividing"
% 
%   o p.14, l.334. values to arguments should also be set in verbatim,
%     e.g., "``OLS''" -> "\code{"OLS"}"
% 
%   o p.14, l.340. "tilde" -> "\verb/~/"
% 
%   o p.14, l.341. Are only "+" signs allowed or the full standard
%     formula language in R?
% 
%   o p.14, eq.68/69/71. "*" -> "\cdot" (and not in verbatim)
% 
%   o p.14, l.346. "dependant" -> "dependent"
% 
%   o p.17, l.387. The creation of tx should be simplified, e.g., the
%     first six lines could be replaced by
%       tx <- cbind(rbind(diag(5), 0, 0), 0)
% 
%   o p.17, l.405. "large" -> "larger", also say that maxiter = 1 is the
%     default.
% 
%   o p.18, Tab.1. I find it a bit confusing that the argument ist called 
%     "rcovformula" but does not expect a formula (in the R sense). Also,
%     type 0 should have an alternative string specification. A similar
%     argument holds for "formula3sls".
% 

\bibliographystyle{jss}
\bibliography{../systemfit}

\end{document}
