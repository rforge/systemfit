\documentclass[12pt,english]{scrartcl}
\usepackage{lmodern}
\usepackage[T1]{fontenc}
\usepackage[latin1]{inputenc}
\usepackage{babel}
\usepackage{geometry}
\geometry{tmargin=0.75in, bmargin=0.9in, lmargin=0.8in,
   rmargin=0.8in, headheight=0.1cm, headsep=0.1cm, footskip=0.4in }
\usepackage{setspace}
\onehalfspacing
\usepackage[authoryear]{natbib}
\usepackage{url}
\usepackage{csquotes}
\MakeOuterQuote{�}

% avoid club lines / orphans
\clubpenalty=10000
% avoid widow lines
\widowpenalty=10000
\displaywidowpenalty=10000


\begin{document}
\begin{center}
{\Large\textbf{\textsf{
systemfit: A Package for Estimating\\[0.3em]
Systems of Simultaneous Equations in \proglang{R} (JSS~224)
}}}

\bigskip

{\Large\textbf{\textsf{Responses to the Reviewers}}}
\end{center}

% -------------------------

\section{First reviewer}

% -------------------------

% It seems that the authors have dealt with my concerns adequately.
% Especially, section 2 has become easier to read. However, the sentence
% starting line 3 and ending line 6 should be changed: the disturbance
% terms are not correlated BECAUSE the models are econometrically
% estimated.

The sentence starting line 3 and ending line 6
has been changed to make this point clearer.
% revision 281

% -------------------------

\section{Second reviewer}

% -------------------------

% The paper and software improves significantly on the initial version
% submitted to JSS. The software looks much better now, I've got only
% a few remaining suggestions for improvement. The paper, however, needs
% some more work and should be streamlined to be more accessible, especially
% for typical JSS readers. More detailed comments are included below.


\subsection{Major points}

\subsubsection{Description of the estimation methods}
%   o I already included this point in my first review:
%       The description of the estimation methods can be shortened considerably
%       and should focus on the unifying properties of the different estimators.
%       All methods from OLS through to standard 3SLS comply with formulas
%       (12) and (13), only with differing choices for Omega (s*I, diagonal or
%       full Omega) and X (original X vs \hat X from instruments). The exposition
%       could be streamlined considerably if this structure were exploited.
%     This has not been changed by the authors, their reply says:
%       A major purpose of this section is that a user of the systemfit package
%       can take a look at the theory (e.g. formulas) of a specific estimation
%       method. Therefore, we fully describe each estimation method in a single
%       (subsub)sectionD
% 
%     I heartily disagree with this. Currently, the unifying properties of the
%     models are not stated explicitely but are brought out by repitition,
%     e.g., formulas 9, 11, and 13 just differ in the subscript of the beta.
%     Also the distinctive properties are not stated explicitely but are always
%     obscured inline in the text. Therefore I would like to re-emphasize my
%     original suggestion. State the general formulas for beta and its covariance
%     once (e.g., in the form of Equations 10 and 11). This gives you explicitely
%     the unifying properties and you just have to state the distinctive properties
%     subsequently, i.e., describe which Omega is used and whether the original X
%     and y are used or projected on the instruments. This could also be done in
%     tabular form giving a quick overview of the different models.
%     Such an exposition would be particularly appropriate in a paper about software
%     because it reveals which structures can be re-used. It would also facilitate
%     the use of the paper for reference purposes.
% 
%     The same comments hold for Section 2.2 which repeats the same formulas over
%     and over again, often just changing the name of the estimator.

The unifying properties of the different estimators are exploited now.
We present the general formulas at the beginning of sections~2.1 and~2.2
and describe which Omega is used for the different estimation methods below.
Section~2.3 (Estimation under linear restrictions)
has been streamlined accordingly.
% rev. 279, 280, 285

\subsubsection{Estimation under linear restrictions}
%   o In my first review I had a comment about Section 2.2 (Imposing linear
%     restrictions) which has only slighly been changed by the authors.
%     Hence, I rephrase my comment:
% 
%     It is still not very clear that this is about estimation: "imposing"
%     does not necessarily imply "estimation" (or "testing"). E.g., in footnore 3,
%     you use "impose (and test)" which is just the wrong terminology: the
%     fundamental concepts in statistics are *estimation* and *testing*!
% 
%     Hence, please say something like
%       "In many empiricaly applications, it is desriable to estimate
%       the coefficients under linear restrictions of the form
%         <Equation 31>
%       For instance, in econometric demand and production analysis, ..."
%     and so on. Thus, also include the footnote within the text.

\subsubsection{Methods for classes ``ystemfit'' and ``systemfit.equation''}
%   o Section 3 starts with a discussion of methods for two classes that
%     haven't been introduced, yet. For readers that have used "systemfit"
%     before this should not be a huge problem (even though still confusing)
%     but for new users this will be very hard. Therefore, please start
%     in Section 3 with a brief description of the core function systemfit()
%     and its usage, e.g.,
%       systemfit(formula, method, data, ...)
%     Briefly indicate what these most important arguments mean and what
%     the classes "systemfit" and "systemfit.equation" are. Then you can
%     go on to explain that all the usual model methods exist from print()
%     over summary() to predict() etc.
% 
%     Note that you should not refer to the methods by their full name
%     (as already indicated in the first review), but as the print() or
%     coef() method for "systemfit" objects etc.
% 
%     Above, I used "formula" as the first argument of systemfit(). The
%     terminology "equations" is non-standard in the S world, so it should
%     be "formulas" or "formula". As there are cases it is indeed a single
%     formula, I would recommend to call it "formula" and just indicate
%     verbally that it is typically a list of several formulas.


\subsection{Minor points}

\subsubsection{Panel-like data}
%   o What is "panel-like data"?
%     I've got no idea what this is and why this term is used. Please use
%     "panel data" and (if necessary) explain what other types of data would
%     also be covered by this.
%     Also Section 4.3 interrupts the flow of Section 4.2 and 4.3
%     and should be postponed.

\subsubsection{Names of arguments of the functions}
%   o Some arguments of the functions in "systemfit" have non-standard
%     names. For example, "returnModelFrame", "returnModelMatrix", etc.
%     Albeit being more verbose, they are much longer and many R users
%     will probably be used to using "model", "x" and "y" instead
%     (which are also much shorter to type). Similarly, "printResidCov"
%     and "printEquations" are extremely long.

\subsubsection{Testing linear restrictions}
%     The new methods for linear.hypothesis() and lrtest() are very useful.
%     But I've got a methodological and a technical question/suggestion:
%       - methodological
%         linear.hypothesis() usually computes Wald tests, this can be
%         done using a finite sample (F) or large sample (Chisq) test.
%         However, the exposition in this paper conveys that the F test
%         is not a Wald test, but something different. What is it?
%         Also the code for test = "F" does something different than in
%         the default linear.hypothesis(). It is not clear to me what
%         this difference should be.
%         Similarly, the residual degrees of freedom for the t test
%         in the summary() are different from df.residual(), is this
%         intended? If so, why? This results in linear.hypothesis.default()
%         and coeftest.default() use different (or wrong?) degrees of
%         freedm than summary().
%       - technical
%         The code for obtaining the object names in systemfit's
%         linear.hypothesis() and lrtest() methods looks quite strange
%         and yields only sub-optimal results. For many of these cases
%         you could simply use the usual deparse(substitute(...)).
%     These points are illustrated by some R code attached at the end
%     of this review.


\subsection{More minor points}

\subsubsection{p.2, l.41, background}
%   o p.2, l.41 "background on that" -> "background that"

\subsubsection{p.3, l.59-60, Kronecker product}
%   o p.3, l.59-60 also introduce the Kronecker product symbol

\subsubsection{p.3-5, notation of Omegas}
%   o p.3-5, the notation for the different Omegas could be more streamlined,
%     e.g., compare p.3, l.68 with p.4, l.71 and p.4, l.79.

\subsubsection{p.4, l.89, parameter/coefficient estimates}
%   o p.4, l.89 with "parameter estimates" you mean "coefficient estimates"?
%     "parameter estimates" is somewhat confusing because you include the
%     comment immediately after the covariance (rather than coefficient)
%     estimator.

\subsubsection{p.5, eq.15, $y$ vs.\ $\hat{y}$}
%   o p.5, eq.15 "y" should be the projected "\textbackslash hat y", right?
%     This would also apply to the other instrument-based formulas below.

\subsubsection{p.15, l.371, book ``car''}
%   o p.15, l.371 I would also include a reference to the book "car",
%     not only the package.

\subsubsection{p. 16, l.414, different computations times}
%   o p. 16, ll.414 The R examples illustrating the different computations
%     times are nice but I would defer them to an appendix. In the main
%     text, I would just verbally explain which option is preferred for
%     which size of data.

\subsubsection{p.18, l.464, regression formulas}
%   o p.18, l.464 (+footnote 7) This description of the formulas is even
%     more confusing than the original version! Please just say that this
%     these are standard regression formulas. JSS readers should know what
%     this is.

\subsubsection{p.20, l.495+3, error checking for ``inst''}
%   o p.20, l.495+3 When playing around with different "inst", I noticed
%     that the error checking could be improved. If "inst" is a named
%     list (as your "eqSystem"), the names could be checked and matched
%     against the main regression formulas. Furthermore, you could catch
%     if the list is too short.

\subsubsection{p.20, l.502, start section with example from p.21, l.528}
%   o p.20, l.502 You should really start this section with the example
%     from p.21, l.528. This is clearly the most intelligible and very
%     easy to use. The other methods should be included below pointing
%     out the differences to the first most convenient interface.
Done.

\subsubsection{p. 22, Table 1, markup for code}
%   o p. 22, Table 1 and paragraphs above and below.
%     The markup for code changes several times here. Table 1 has single
%     directed quotes (although the authors probably wanted undirected
%     quotes). The paragraph above has no quotes by a typewriter font.
%     The paragraph below has typewriter plus directed double quotes.
%     I would suggest to do away with all the quotes and always use
%     typewriter fonts.
%     This also applies to the whole paper...it just became moste apparent
%     here.
The markup for code has been made more consistent now.
Quotes are used for character strings only;
all these quotes are undirected double quotes in typewriter font.

\subsubsection{p.23, l.591, opposite/contrast}
%   o p.23, l.591. "In opposite" -> "In contrast"

Done.

\subsubsection{p.24, code below l.600, example ``GrunfeldGreene''}
%   o p.24, code below l.600
%     The quotation of the arguments of data() and library() changes.
%     The code could be made even more readable by using named arguments.
%       data("GrunfeldGreene")
%       library("plm")
%       gr <- pdata.frame(GrunfeldGreene, id = "firm", time = "year")
%       greeneSur <- systemfit(invest ~ value + capital, method = "SUR",
%         data = gr)
%     Both comments apply more generally to the whole paper.
%     (Note that using pdata.frame() without assignment is possible
%     but really bizarre. I wouldn't use this kind of syntax.)

Done.

\subsubsection{p.26, l.622, test}
%   o p.26, l.622 "tests on the" -> "tests of the"

Done.

\subsubsection{p.26, l.626, doubling}
%   o p.26, l.626 "restricted by ... restrictions": avoid doubling

Done.

\subsubsection{p.26, l.634, printCoefmat}
%   o p.26, l.634 and code below
%     The coefficients should be printed by printCoefmat() to obtain
%     the usual formatting.

Done.
% rev. 497 of the systemfit package

\subsubsection{p.27, l.640, hyphenation}
%   o p.27, l.640 and code below
%     omit hyphenation in "F-test" and "Wald-test"

Done.
% rev. 496 of the systemfit package

\subsubsection{p.28, l.646, directed quotes vs.\ double quotes}
%   o p.28, l.646 don't use directed quotes around "F" and "Chisq".
%     For character strings, include the double quotes within the
%     typewriter font, e.g.,
%       \code{"F"} and \code{"Chisq"}

Done.

\subsubsection{p.29, l.673, references for the errata to Greene (2003)}
%   o p.29, l.673 Please add an item in the references for the
%     errata to Greene (2003). Instead of repeating the URL, use
%     the correct citation then. Also note that the URL
%       http://pages.stern.nyu.edu/~wgreene/Text/econometricanalysis.htm
%     now points to the 6th edition of Greene's book!

Done.

\subsubsection{p.37, Table 3, typewriter markup}
%   o p.37, Table 3. Use typewriter markup for the code elements in
%     the table.

Done.

\end{document}
