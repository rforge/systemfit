\documentclass[12pt,english]{scrartcl}
\usepackage{lmodern}
\usepackage[T1]{fontenc}
\usepackage[latin1]{inputenc}
\usepackage{babel}
\usepackage{geometry}
\geometry{tmargin=0.75in, bmargin=0.9in, lmargin=0.8in,
   rmargin=0.8in, headheight=0.1cm, headsep=0.1cm, footskip=0.4in }
\usepackage{setspace}
\onehalfspacing
\usepackage[authoryear]{natbib}
\usepackage{url}
\usepackage{csquotes}
\MakeOuterQuote{�}

% avoid club lines / orphans
\clubpenalty=10000
% avoid widow lines
\widowpenalty=10000
\displaywidowpenalty=10000


\begin{document}
\begin{center}
{\Large\textbf{\textsf{
systemfit: A Package for Estimating\\[0.3em]
Systems of Simultaneous Equations in \proglang{R} (JSS~224)%
\footnote{
The original title was: �systemfit: A Package to Estimate
Simultaneous Equation Systems in \proglang{R}�.}
}}}

\bigskip

{\Large\textbf{\textsf{Responses to Reviewer B}}}
\end{center}


We thank this reviewer for a very helpful set of comments and suggestions
for improving the paper.
The reviewer has made one overall comment, 9 specific comments and
several smaller comments.
We have structured our reply accordingly.

\section{Overall Comment}
% Review of the manuscript 'systemfit: A package to estimate
% simultaneous equation systems in R'
% This manuscript describes a new R package systemfit for inference in multi-equation models. The
% main parts are section 2 and 4. Section 2 gives the statistical background for the analysis tools
% available in systemfit while section 4 describes how to apply systemfit. The theory is adequately
% explained and the syntax in systemfit is nice and logical allowing complicated models to be quite
% easily fitted.
% My only real concern is that systemfit relies on classic econometric (least squares) developed 40
% years ago. Since then computational power has increased dramatically and superior methods may be
% available. First of all, full likelihood methods are available in existing software for structural
% equation modelling. In particular, R has a package called SEM. I am not sure what is gained with
% systemfit compared to this. Also it is a weakness that systemfit only allows analysis of complete
% cases.

It was our intention to create a package
for fitting simultaneous equation models that could
be verified against published results before moving on to include more
sophisticated fitting methods. 

The reasons for doing this were two-fold.  Our first reason was to
simply introduce an econometrics package where the main objective of the
package was a unified framework for simultaneous equation models
and not to have a few specialized
routines, as is the case with the \code{tsls()} function in the \pkg{sem}
package. While there are definitely links among structural equation
models, IV methods and experimental design, the focus of this package
was econometrics; more specifically simultaneous equation methods.   

The second reason for creating the package was to create a �forum� for
more advanced fitting methods as described in Section 6 (Summary and
Outlook). It is our intention to implement these more advanced methods
(unbalanced data sets, FIML, GMM, etc.) over time, but we felt that
having a �basic� package that was accepted as part of a peer-review
process would create an environment that would reduce resistance to
acceptance in the future since the more sophisticated methods can be
subject to the difficulties beyond simple numerical routines as were
used in the \pkg{systemfit} package so far.

Encouraged by the question of this referee,
we have added a comparison of \code{systemfit()} and \code{sem()}
(see section~3.4 and appendix~B).


\section{Specific comments}
% Specific comments

\subsection{Instrumental variable estimation}
% Instrumental variable estimation has received much attention lately
% in biostatistics. Maybe it would help 'sell' systemfit if this was
% emphasized at bit more. For instance, instrumental variables could
% be mentioned directly in the introduction and included as a key
% word.

We have mentioned �instrumental variables� in the abstract, included
three examples of topics that use IV methods and have added
�instrumental variables� to the keywords.

\subsection{Section 2}
% Section 2 gives a brief overview of the statistical background. This
% may be a bit hard to follow for people outside econometrics. It
% would help to include more references, especially in section 2.1.

We have added several references to literature (e.g.\ econometrics textbooks)
that provides a detailed description of simultaneous equations systems.


\subsection{Number of observations}
% Page 2: 'observations' should be explained earlier together with the presentation of the data
% structure. The number of observations should be stated.

We have included a brief statement that the number of observations for
each equation in the system is the same. 

\subsection{x and y variables}
% Page 2: It should be clarified whether y-variables in one equation can be x-variables in another.

This is stated in the introduction (page 2, lines 14-15).

\subsection{Instrumental variables}
% Page 4, line 82: 'may' should be 'must'. Here the difference is extremely important.

According to the comment of this referee,
we have replaced �may� by �must�.

% Maybe the text
% could emphasize more clearly what is required of a variable if it is to be used as an instrumental
% variable. What is a good instrumental variable and what happens if the requirements are not
% fulfilled?

The selection of instrument variables is discussed in more detail in
other references and may quite unique to each field. 
We chose not to focus on details such as selection of instrument
variables, variance-covariance structures and other specifics of
simultaneous equations as this article was written to present the
basic equations, fitting methods and usage of the software package.
While we desired to include a wide variety of topics, such as
structure of the residual variance-covariance matrix, many of these
were included in the package to compare, and finally match, the
results from other software packages.
For guidance on selection of instrument variables, the reader is
referred to the citations in the article.

\subsection{Equation 20 and 23}
% The difference between equation 20 and 23 is so smaller that maybe it should be noted in the text
% what it is.

We have added a note to make the reader aware of the difference
between these two equations. 

\subsection{Residual covariance matrix (Sigma)}
% Line 193-194: Language is confusing. It should be explained that calculation of some of the
% estimators earlier introduced require Sigma to be known. Often Sigma is not known so an estimator
% is used instead. However, then statistical properties may changed compared to the theory described
% earlier.

We have described this issue more clearly now.
For instance in sections~2.1 and~2.2, the residual covariance matrix
is no longer treated as known.
Hence, the fact that $\Sigma$ and $\Omega$ are unknown and have to be estimated
is no longer �hidden� in section 2.3.


\subsection{First step OLS}
% Line 196: It should be explained what is meant by a first step OLS.

We have explained the two-step procedure
to obtain the estimated residual covariance matrix
more clearly now.


\subsection{Efficiency and maximum likelihood}
% Section 2.3: Somewhere in this section or (2.1) it should be noted that the efficiency results of
% section 2.1 depend on Sigma being known. And that a full maximum likelihood analysis may be
% superior.

\textbf{[Arne, I'm not sure how to answer this one. I'm not sure this is an
important issue for the manuscript. We mention that FIML is a more
sophisticated approach in the summary and outlook section, but we
don't really mention FIML in the paper. Should we?]}


\subsection{Alternative approach}
% Line 215-219: Alternative approach should be explained more clearly.

\textbf{[Arne, can we simply not mention the "alternative method" in the
paper? I think it should stay as is, but I need a little more
understanding of the reasoning behind the method and can't defend
it. I just don't have any exposure to it yet.]}


\subsection{Syntax for instrumental variables}
% Line 375-379: Syntax on programming of instrumental variables should be made more clear. What
% is a one-sided formula? Does the code given imply that the sum of income, farmPrice and trend was
% used as and instrumental variable? If so, it should be stated in the text.

We have included the proper citation for the R manuals which describes
the syntax for formulas. 

\subsection{Section 2.7}
% Section 2.7: Given the theory of the previous sections, I was confused about the Hausman test.
% Under the alternative hypothesis, independent variables are correlated with error terms. In this
% section it is stated that this leads to inconsistency of 3SLS. However, on page 5 (top) the opposite is
% stated? Would it not make more sense to compare SUR to 3SLS?

The Hausman test is somewhat unique from other detection tests in that
the test is designed "backwards" in the sense that the model that is
being tested is correctly specified and not whether a predetermined 
value is sufficiently close to the parameter estimate. 

\textbf{[I'm tired and will have to get back to this]}

\section{Smaller comments}
% Smaller corrections:
% Sometimes ':' is used just before an equation other times not.
% Page 7, line 147: change 'doesn't' to 'does not'.
% Line 135: '\beta^*' should be '\widehat{\beta^*}'
% Line 209: first \widehat{\Sigma} should be \Sigma
% Equation 63: '\Sigma' should be '\widehat{\Sigma}
% Equation 65: Why two hats?
% Line 299: 'interest' should be 'interested'
% Page 14: 'dependant' should be 'dependent'
We have corrected the paper according to all of these comments.

% \bibliographystyle{}
% \bibliography{}

\end{document}
