\documentclass[12pt,english]{scrartcl}
\usepackage{lmodern}
\usepackage[T1]{fontenc}
\usepackage[latin1]{inputenc}
\usepackage{babel}
\usepackage{geometry}
\geometry{tmargin=0.75in, bmargin=0.9in, lmargin=0.8in,
   rmargin=0.8in, headheight=0.1cm, headsep=0.1cm, footskip=0.4in }
\usepackage{setspace}
\onehalfspacing
\usepackage[authoryear]{natbib}
\usepackage{url}
\usepackage{csquotes}
\MakeOuterQuote{�}

% avoid club lines / orphans
\clubpenalty=10000
% avoid widow lines
\widowpenalty=10000
\displaywidowpenalty=10000


\begin{document}
\begin{center}
{\Large\textbf{
systemfit: A Package to Estimate
Simultaneous Equation Systems in R (JSS~224)}}

\bigskip

{\Large\textbf{Responses to Reviewer A}}
\end{center}


We thank this reviewer for a very helpful set of comments and suggestions
for improving the paper.
The reviewer has made one overall comment, 9 specific comments and
several smaller comments.
We have structured our reply accordingly.

\section{Overall Comment}
% Review of the manuscript 'systemfit: A package to estimate
% simultaneous equation systems in R'
% This manuscript describes a new R package systemfit for inference in multi-equation models. The
% main parts are section 2 and 4. Section 2 gives the statistical background for the analysis tools
% available in systemfit while section 4 describes how to apply systemfit. The theory is adequately
% explained and the syntax in systemfit is nice and logical allowing complicated models to be quite
% easily fitted.
% My only real concern is that systemfit relies on classic econometric (least squares) developed 40
% years ago. Since then computational power has increased dramatically and superior methods may be
% available. First of all, full likelihood methods are available in existing software for structural
% equation modelling. In particular, R has a package called SEM. I am not sure what is gained with
% systemfit compared to this. Also it is a weakness that systemfit only allows analysis of complete
% cases.

\section{Specific comments}
% Specific comments

\subsection{Instrumental variable estimation}
% Instrumental variable estimation has received much attention lately in biostatistics. Maybe it would
% help 'sell' systemfit if this was emphasized at bit more. For instance, instrumental variables could
% be mentioned directly in the introduction and included as a key word.

\subsection{Section 2}
% Section 2 gives a brief overview of the statistical background. This may be a bit hard to follow for
% people outside econometrics. It would help to include more references, especially in section 2.1.

\subsection{Number of observations}
% Page 2: 'observations' should be explained earlier together with the presentation of the data
% structure. The number of observations should be stated.

\subsection{x and y variables}
% Page 2: It should be clarified whether y-variables in one equation can be x-variables in another.

\subsection{'may' should be 'must'}
% Page 4, line 82: 'may' should be 'must'. Here the difference is extremely important. Maybe the text
% could emphasize more clearly what is required of a variable if it is to be used as an instrumental
% variable. What is a good instrumental variable and what happens if the requirements are not
% fulfilled?

\subsection{Equation 20 and 23}
% The difference between equation 20 and 23 is so smaller that maybe it should be noted in the text
% what it is.

\subsection{Sigma}
% Line 193-194: Language is confusing. It should be explained that calculation of some of the
% estimators earlier introduced require Sigma to be known. Often Sigma is not known so an estimator
% is used instead. However, then statistical properties may changed compared to the theory described
% earlier.

\subsection{First step OLS}
% Line 196: It should be explained what is meant by a first step OLS.

\subsection{Efficiency and maximum likelihood}
% Section 2.3: Somewhere in this section or (2.1) it should be noted that the efficiency results of
% section 2.1 depend on Sigma being known. And that a full maximum likelihood analysis may be
% superior.
% 

\subsection{Alternative approach}
% Line 215-219: Alternative approach should be explained more clearly.

\subsection{Syntax for instrumental variables}
% Line 375-379: Syntax on programming of instrumental variables should be made more clear. What
% is a one-sided formula? Does the code given imply that the sum of income, farmPrice and trend was
% used as and instrumental variable? If so, it should be stated in the text.

\subsection{Section 2.7}
% Section 2.7: Given the theory of the previous sections, I was confused about the Hausman test.
% Under the alternative hypothesis, independent variables are correlated with error terms. In this
% section it is stated that this leads to inconsistency of 3SLS. However, on page 5 (top) the opposite is
% stated? Would it not make more sense to compare SUR to 3SLS?

\section{Smaller comments}
% Smaller corrections:
% Sometimes ':' is used just before an equation other times not.
% Page 7, line 147: change 'doesn't' to 'does not'.
% Line 135: '\beta^*' should be '\widehat{\beta^*}'
% Line 209: first \widehat{\Sigma} should be \Sigma
% Equation 63: '\Sigma' should be '\widehat{\Sigma}
% Equation 65: Why two hats?
% Line 299: 'interest' should be 'interested'
% Page 14: 'dependant' should be 'dependent'

% \bibliographystyle{}
% \bibliography{}

\end{document}
