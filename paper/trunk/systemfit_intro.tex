
%       $Id$    


%%%%%%%%%%%%%%%%%%%%%%%%%%%%%%%%%%%%%%%%%%%%%%%%%%
\section{Introduction}
%%%%%%%%%%%%%%%%%%%%%%%%%%%%%%%%%%%%%%%%%%%%%%%%%%

Many theoretical models that are econometrically estimated
consist of more than one equation.
In this case, the disturbance terms of these equations are likely
to be contemporaneously correlated,
because some unconsidered factors
that influence the disturbance term in one equation
probably influence the disturbance terms in other equations
of these model, too.
Ignoring this contemporaneous correlation
and estimating these equations separately
leads to inefficient parameter estimates.
However, estimating all equations simultaneously,
taking the covariance structure of the residuals into account,
leads to efficient estimates.
This estimation procedure is generally called
�Seemingly Unrelated Regression� (SUR) \citep{zellner62}.
Another reason to estimate an equation system simultaneously are
cross-equation parameter restrictions.%
\footnote{
Especially the economic theory suggests many cross-equation parameter
restrictions (e.g.\ the symmetry restriction in demand models).
}
These restrictions can be tested and/or imposed only in a simultaneous
estimation approach.

Furthermore, these models can contain variables
that appear on the left-hand side in one equation
and on the right-hand side of another equation.
Ignoring the endogeneity of these variables can lead to inconsistent
parameter estimates.
This simultaneity bias can be circumvented by applying
a �Two-Stage Least Squares� (2SLS) or �Three-Stage Least Squares� (3SLS)
estimation of the equation system.


% For all of the methods developed in the package, the disturbances of
% the individual equations are assumed to be independent and identically
% distributed (iid).  
% In the future, we would like to add the ability to fit equations were
% the disturbances are serially correlated (wikins 1969).

The \pkg{systemfit} package provides the capability to estimate
linear equation systems in \proglang{R}
\citep{r-project}.
Although linear equation systems can be estimated
with several other statistical and econometric software packages
(e.g.\ \proglang{SAS}, \proglang{EViews}, \proglang{TSP}),
\pkg{systemfit} has several advantages.
First, all estimation procedures are publicly available in the source code.
Second, the estimation algorithms can be easily modified to meet specific
requirements.
Third, the (advanced) user can control estimation details generally
not available in other software packages by overriding reasonable defaults.

This paper is organized as follows:
In the following Section~\ref{sec:statistics} we
introduce the statistical background of estimating equation systems.
The implementation of the statistical procedures in \proglang{R}
is shortly explained in Section~\ref{sec:code}.
Section~\ref{sec:Usage} demonstrates how to run
\pkg{systemfit} and especially how the features presented in the previous
section can be used.
In Section~\ref{sec:reliability} the reliability of the results from
\pkg{systemfit} are tested.
Finally, a summary and outlook are presented in
Section~\ref{sec:Summmary}.


%%% Local Variables: 
%%% mode: latex
%%% TeX-master: "systemfit"
%%% End: 
