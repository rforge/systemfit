
%       $Id$    


%%%%%%%%%%%%%%%%%%%%%%%%%%%%%%%%%%%%%%%%%%%%%%%%%%
\section{Introduction}
%%%%%%%%%%%%%%%%%%%%%%%%%%%%%%%%%%%%%%%%%%%%%%%%%%

Many theoretical models consist of more than one equation.
These models can contain variables that appear on the left-hand side in one
equation and on the right-hand side of another equation (simultaneity-bias).
The system can have non-zero off-diagonal covariance elements resulting from
related disturbances (contemporaneous correlation).
Ignoring contemporaneous correlation leads to inefficient parameter
estimates \citep{zellner62} and ignoring simultaneity-bias can lead to
inconsistent parameter estimates.

% If these models are econometrically estimated, it may not be
% appropriate to estimate the equations individually, but all together,
% because the disturbance terms of these equations are likely to be
% contemporaneously correlated.  Estimating all equations
% simultaneously, taking the covariance structure of the residuals into
% account, leads to efficient estimates.  Ignoring this correlation
% leads to an inefficient parameter estimation \citep{zellner62}.

Another reason to estimate an equation system simultaneously are
cross-equation parameter restrictions.
These restrictions can be tested and/or imposed only in a simultaneous
estimation approach.
Especially the economic theory suggests many cross-equation parameter
restrictions (e.g.\ the symmetry restriction in demand models).

% (But we don't say why this is important...) Are we
% imposing restrictions in the parameters, or the variance-covariance
% matrix or both and for what pupose?

% For all of the methods developed in the package, the disturbances of
% the individual equations are assumed to be independent and identially
% distributed (iid).  
% In the future, we would like to add the ability to fit equations were
% the disturbances are serially correlated (wikins 1969).

The \pkg{systemfit} package provides the capability to estimate
linear and non-linear equation systems in \proglang{R}
\citep{r-project}.
Although linear and non-linear equation systems can be estimated
with several other statistical and econometric software packages
(e.g.\ \proglang{SAS}, \proglang{EViews}, \proglang{TSP}),
\pkg{systemfit} has several advantages.
First, all estimation procedures are publicly available in the source code.
Second, the estimation algorithms can be easily modified to meet specific
requirements.
Third, the (advanced) user can control estimation details generally
not available in other software packages by overriding reasonable defaults.

This paper is organized as follows: In Section~\ref{sec:Estimation} we
introduce the mathematics of estimating equation systems.
Section~\ref{sec:Restrictions} shows how linear restrictions can be
imposed.
Section~\ref{sec:Usage} demonstrates how to run
\pkg{systemfit}, especially how the features presented in the previous
sections can be used.
All other relevant issues are discussed in Section~\ref{sec:Other}.
Finally, a summary and outlook are presented in
Section~\ref{sec:Summmary}.


%%% Local Variables: 
%%% mode: latex
%%% TeX-master: "systemfit"
%%% End: 
