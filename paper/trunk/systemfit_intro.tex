
%       $Id$    


%%%%%%%%%%%%%%%%%%%%%%%%%%%%%%%%%%%%%%%%%%%%%%%%%%
\section{Introduction}
%%%%%%%%%%%%%%%%%%%%%%%%%%%%%%%%%%%%%%%%%%%%%%%%%%

Many theoretical models that are econometrically estimated
consist of more than one equation.
The disturbance terms of these equations are likely
to be contemporaneously correlated,
because unconsidered factors
that influence the disturbance term in one equation
probably influence the disturbance terms in other equations, too.
Ignoring this contemporaneous correlation
and estimating these equations separately
leads to inefficient estimates of the coefficients.
However, estimating all equations simultaneously
with a �Generalized Least Squares� (GLS) estimator,
which takes the covariance structure of the residuals into account,
leads to efficient estimates.
This estimation procedure is generally called
�Seemingly Unrelated Regression� (SUR) \citep{zellner62}.
Another reason to estimate a system of equations simultaneously are
cross-equation restrictions on the coefficients.%
\footnote{
Especially the economic theory suggests many cross-equation
restrictions on the coefficients (e.g.\ the symmetry restriction in demand models).
}
Estimating the coefficients under cross-equation restrictions
and testing these restrictions
requires a simultaneous estimation approach.

Furthermore, these models can contain variables
that appear on the left-hand side in one equation
and on the right-hand side of another equation.
Ignoring the endogeneity of these variables can lead to inconsistent
estimates.
This simultaneity bias can be corrected for by
applying a �Two-Stage Least Squares� (2SLS) estimation
to each equation.
Combining this estimation method with the SUR method results
in a simultaneous estimation of the system of equations
by the �Three-Stage Least Squares� (3SLS) method
\citep{zellner62b}.

% For all of the methods developed in the package, the disturbances of
% the individual equations are assumed to be independent and identically
% distributed (iid).  
% In the future, we would like to add the ability to fit equations were
% the disturbances are serially correlated (wikins 1969).

The \pkg{systemfit} package provides the capability to estimate
systems of linear equations in \proglang{R}
\citep{r-project}.
Currently, the estimation methods
�Ordinary Least Squares� (OLS),
�Weighted Least Squares� (WLS),
�Seemingly Unrelated Regression� (SUR),
�Two-Stage Least Squares� (2SLS),
�Weighted Two-Stage Least Squares� (W2SLS), and
�Three-Stage Least Squares� (3SLS)
are implemented.%
\footnote{
In this context,
the term �weighted� in �Weighted Least Squares� (WLS) and
�Weighted Two-Stage Least Squares� (W2SLS) means
that the \emph{equations} might have different weights
and \emph{not}
that the \emph{observations} have different weights.
}
The WLS, SUR, W2SLS, and 3SLS estimates can be based
either on one-step (OLS or 2SLS) (co)variances
or these estimations can be iterated,
where the (co)variances are calculated from the estimates of the previous step.
Furthermore,
the \pkg{systemfit} package
provides statistical tests for restrictions on the coefficients
and for testing the consistency of the 3SLS estimation.

Although systems of linear equations can be estimated
with several other statistical and econometric software packages
(e.g.\ \proglang{SAS}, \proglang{EViews}, \proglang{TSP}),
\pkg{systemfit} has several advantages.
First, all estimation procedures are publicly available in the source code.
Second, the estimation algorithms can be easily modified to meet specific
requirements.
Third, the (advanced) user can control estimation details generally
not available in other software packages by overriding reasonable defaults.

% This paper is organized as follows:
In Section~\ref{sec:statistics} we introduce the statistical
background of estimating equation systems.
The implementation of the statistical procedures in \proglang{R} is
briefly explained in Section~\ref{sec:code}.
Section~\ref{sec:Usage} demonstrates how to run
\pkg{systemfit} and how some of the features presented in the second
section can be used.
In Section~\ref{sec:reliability} we replicate several textbook results
with the \pkg{systemfit} package.
Finally, a summary and outlook are presented in
Section~\ref{sec:Summmary}.


%%% Local Variables: 
%%% mode: latex
%%% TeX-master: "systemfit"
%%% End: 
