
%       $Id$    


%%%%%%%%%%%%%%%%%%%%%%%%%%%%%%%%%%%%%%%%%%%%%%%%%%
\section{Introduction}
%%%%%%%%%%%%%%%%%%%%%%%%%%%%%%%%%%%%%%%%%%%%%%%%%%

Many theoretical models consist of more than one equation. If these
models are econometrically estimated, it may not be appropriate to
estimate the equations individually, but all together, because the
disturbance terms of these equations are likely to be
contemporaneously correlated \citep{theil71} (make sure this is the
correct paper). Estimating all equations simultaneously, taking the
covariance structure of the residuals into account, leads to efficient
estimates. Ignoring this correlation leads to an inefficient parameter
estimation \citep{zellner62}.

Another reason to estimate an equation system simultaneously are
cross-equation parameter restrictions. These restrictions can be
tested and/or imposed only in a simultaneous estimation approach
especially the economic theory suggests many cross-equation
restrictions \textbf{(is this statement really true?)}.

The \pkg{systemfit} package provides the capability to estimate linear
and non-linear equation systems in \proglang{R}.  Although linear and
non-linear equation systems can be estimated with several other
statistical and econometric software packages (e.g. \proglang{SAS},
\proglang{EViews}, \proglang{TSP}), \pkg{systemfit} has several
advantages.  First, all estimation procedures are publicly available
in the source code.  Second, the estimation algorithms can be easily
modified to meet specific requirements.  Third, the (advanced) user
can control many important details of the estimation procedure that
are generally not available in other software packages.  On the other
hand, these user options have reasonable defaults that also beginners
can easily use the provided functions.

This paper is organized as follows: In section~\ref{sec:Estimation} we
introduce the mathematics of estimating equation systems.
Section~\ref{sec:Restrictions} shows how linear restrictions can be
imposed.  Section~\ref{sec:Usage} demonstrates how to run
\pkg{systemfit}, especially how the features presented in the previous
sections can be used. All other relevant issues are discussed in
section~\ref{sec:Other}. Finally, a summary and outlook are presented
in section~\ref{sec:Summmary}.


%%% Local Variables: 
%%% mode: latex
%%% TeX-master: "systemfit"
%%% End: 
